\chapter{Classical Relations}
\label{app:clRel}

In this appendix we list examples of classical relations as calculated by a Mathematica package available at:
\url{https://github.com/willzeng/GroupoidHomRelations}
\begin{align*}
\begin{tabular}{lcl}
Classical relations $\mathbb{Z}_3\to\mathbb{Z}_3$:  & \hspace{60pt} & Classical relations $\mathbb{Z}_4\to\mathbb{Z}_4$:  \\[-10pt]
\begin{tabular}{l}
\{(0,0),   (0,1),   (0,2)\} \\
\{(0,0),(1,1),(2,2)\} \\
\{(0,0),(1,2),(2,1)\}
\end{tabular} & &
\begin{tabular}{l}
\\
\{(0,0),(0,1),(0,2),(0,3)\} \\
\{(0,0),(1,1),(2,2),(3,3)\} \\
\{(0,0),(2,1),(0,2),(2,3)\} \\
\{(0,0),(3,1),(2,2),(1,3)\}
\end{align*}
\end{tabular}
\end{tabular}
\end{align*}

\noindent The classical relations from  $\mathbb{Z}_2\oplus\mathbb{Z}_2\to\mathbb{Z}_2\oplus\mathbb{Z}_2$ are:
\begin{align*}
\begin{tabular}{cc}
\{(0,2),(2,2),(1,3),(3,3)\} &\qquad \{(0,0),(1,1),(2,2),(3,3)\} \\
\{(0,2),(2,2),(1,3),(2,3)\} &\qquad \{(0,0),(1,1),(2,2),(2,3)\} \\
\{(0,2),(2,2),(0,3),(3,3)\} &\qquad \{(0,0),(0,1),(2,2),(3,3)\} \\
\{(0,2),(2,2),(0,3),(2,3)\} &\qquad \{(0,0),(0,1),(2,2),(2,3)\} \\
\{(2,0),(3,1),(0,2),(1,3)\} &\qquad \{(0,0),(2,0),(1,1),(3,1)\} \\
\{(2,0),(3,1),(0,2),(0,3)\} &\qquad \{(0,0),(2,0),(1,1),(2,1)\} \\
\{(2,0),(2,1),(0,2),(1,3)\} &\qquad \{(0,0),(2,0),(0,1),(3,1)\} \\
\{(2,0),(2,1),(0,2),(0,3)\} &\qquad \{(0,0),(2,0),(0,1),(2,1)\} \\
\end{tabular}
\end{align*}

\section{Mathematica Code}
\begin{lstlisting}
(* This code generate groupoid homomorphisms in the category of relations *)
(* by William Zeng 2015 *)
(* william.zeng@cs.ox.ac.uk *)

Groupoid Definitions

These multiplication tables define whatever groupoid homomorphisms you are considering. It is prepopulated with some usual small ones 

(* Define the cyclic group multiplication *)
(* INPUT: Integers 1,1 *)
(* OUTPUT: Integer 2*)
z3[x_, y_] := ({
      {0, 1, 2},
      {1, 2, 0},
      {2, 0, 1}
     })[[x + 1]][[y + 1]];
(* This defines multiplication on a set of inputs *)
(* INPUT: Integer sets {1,2},{2}*)
(* OUTPUT: Integer set {0,2}*)
z3Set[X_, Y_] := Union[Flatten[
    Table[
     Table[
      z3[X[[x]], Y[[y]]]
      , {x, 1, Length[X]}]
     , {y, 1, Length[Y]}]
    ]];
z4[x_, y_] := ({
      {0, 1, 2, 3},
      {1, 2, 3, 0},
      {2, 3, 0, 1},
      {3, 0, 1, 2}
     })[[x + 1]][[y + 1]];
z4Set[X_, Y_] := Union[Flatten[
    Table[
     Table[
      z4[X[[x]], Y[[y]]]
      , {x, 1, Length[X]}]
     , {y, 1, Length[Y]}]
    ]];

(* a is undefined *)
z2z2[x_, y_] := ({
      {0, 1, {}, {}},
      {1, 0, {}, {}},
      {{}, {}, 2, 3},
      {{}, {}, 3, 2}
     })[[x + 1]][[y + 1]];
z2z2Set[X_, Y_] := Union[Flatten[
    Table[
     Table[
      z2z2[X[[x]], Y[[y]]]
      , {x, 1, Length[X]}]
     , {y, 1, Length[Y]}]
    ]];

z2z2z2[x_, y_] := ({
      {0, 1, {}, {}, {}, {}},
      {1, 0, {}, {}, {}, {}},
      {{}, {}, 2, 3, {}, {}},
      {{}, {}, 3, 2, {}, {}},
      {{}, {}, {}, {}, 4, 5},
      {{}, {}, {}, {}, 5, 4}
     })[[x + 1]][[y + 1]];
z2z2z2Set[X_, Y_] := Union[Flatten[
    Table[
     Table[
      z2z2z2[X[[x]], Y[[y]]]
      , {x, 1, Length[X]}]
     , {y, 1, Length[Y]}]
    ]];

(* Relational evaluation method *)
(* INPUT: Relation R = {{0,0},{1,1},{2,2},{2,3}}; and argument x=2*)
(* OUTPUT: List of Integers {2,3}*)
eval[R_, x_] := Map[#[[2]] &, Select[R, #[[1]] == x &]];
(* All possible relations *)
(*allRelations = Subsets[Tuples[{0,1,2,3},2]];*)

Choose Groupoids

(*Set the size of the domain and coDomain*)
nD = 4; (*Size of the domain *)
ncoD = 4; (*Size of the coDomain *)

(* YOU NEED TO SET THE GROUPOIDS HERE*)
(* Test to see if monoid multiplication is preserved. *)
(* Need this and unitality condition to define a groupoid homomorphism*)
(* INPUT: Relation R = {{0,0},{1,1},{2,2}};*)
(* OUTPUT: True or False *)
isMonHom[R_] := If[Length[Position[Flatten[
      Table[
       Table[
        (*Specify which groupoids are domain and coDomain here*)
        (*Use the multiplication version that is defined on sets of elements \
for the coDomain*)
        eval[R, z2z2[x, y]] == z2z2Set[eval[R, x], eval[R, y]]
        , {y, 0, nD - 1}]
       , {x, 0, nD - 1}]]
     , False]] == 0, True, False]

Print["Be careful. A full search requires checking ", 2^(nD*ncoD), 
  " relations."];

(* R = {{0,0},{1,1},{2,2}};*)
(* check which relations are comonoid homs*)
(*passed =Select[allRelations\[LeftDoubleBracket]1;;1000\[RightDoubleBracket],
isCoHom[#]&
];
Grid[%]*)

Be careful. A full search requires checking 65536 relations.

Search for Monoid Homomorphisms

Print["Note that this outputs a Monoid Homomorphism relation.
        If you are interested in a classical relations (comonoid homomorphism \
relations), you will need to take the relational 
        converse of this output.  See arXiv:1503.05857"];

(* All possible pairs of elements one in domain and one in coDomain *)
pairings = Tuples[{Range[0, nD - 1], Range[0, ncoD - 1]}];

(* Specify the range of relations to search over *)
(* They are defined as binary strings from 0 to 2^(nD*ncoD) and*)
(* searched sequentially*)
(* BE SURE TO HARD CODE IN THE UNITALITY CONDITION *)
start = 0;
end = 2^(nD*ncoD); (* MAX is 2^(nD*ncoD)*)

results = {};
For[i = start, i < end, i++,
 (* We use binary numbers to index the different possible relations *)
 (* where a 0 or a 1 indicates if one of the nD*ncoD pairings is present *)
 binaryIndex = PadLeft[IntegerDigits[i, 2], nD*ncoD];
 relation = Map[pairings[[#]] &, Flatten[Position[binaryIndex, 1]]];
 
 (* Check the unitality condition *)
 (* BE SURE TO HARD CODE IN THE UNITALITY CONDITION HERE*)
 (* EXAMPLES *)
 (* Z3 -> 3 should be unitalCondition=Union[eval[relation,0]]\[Equal]{0};*)
 (* Z4 -> Z4 should be unitalCondition=Union[eval[relation,0]]\[Equal]{0};*)
 (* Z2Z2 -> Z2Z2 should be unitalCondition=Union[eval[relation,0],eval[relation,2]]\[Equal]{0,2}; *)
 unitalCondition = Union[eval[relation, 0], eval[relation, 2]] == {0, 2};
 
 (* Return True if both the unitality condition and preservation of monoid
mult are met*)
 If[unitalCondition && isMonHom[relation],
  Print[relation];
  results = Append[results, relation];
  ];
 ]

Note that this outputs a Monoid Homomorphism relation.
        If you are interested in a classical relations (comonoid homomorphism relations), you will need to take the relational 
        converse of this output.  See arXiv:1503.05857

Tests - If you want to dive deeper into an example

(* Prints the multiplication table for relation R *)
(* BE SURE TO SPECIFY THE SAME GROUPOIDS*)
(* INPUT: Relation R = {{0,0},{1,0}}; *)
(* OUTPUT: {{0,0},True} {{0,1},True}    {{0,2},True}    {{0,3},True}
{{1,0},True}    {{1,1},True}    {{1,2},True}    {{1,3},True}
{{2,0},True}    {{2,1},True}    {{2,2},True}    {{2,3},True}
{{3,0},True}    {{3,1},True}    {{3,2},True}    {{3,3},True}

 *)
mTable[R_] := Grid[Table[
   Table[
    (* Specify groupoids as in isMonHom *)
    {{x, y}, eval[R, z2z2[x, y]] == z2z2z2Set[eval[R, x], eval[R, y]]}
    , {y, 0, nD - 1}]
   , {x, 0, nD - 1}]]

(* Pick a relation to test and see its multiplication table*)
R = {{0, 0}, {0, 4}, {1, 0}, {1, 4}, {2, 2}, {3, 2}};
mTable[R]
Union[eval[R, 0], eval[R, 2]] == {0, 2, 4}

\end{lstlisting}

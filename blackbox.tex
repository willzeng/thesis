\chapter{Quantum Blackbox Algorithms}

\todo{\chapabstract{}}

\section{\color{blue} The abstract structure of unitary oracles}


\subsection{Complementarity via unitarity}
A pair of symmetric dagger-Frobenius algebras can be used to build a linear map in the following way:
\begin{equation}
\label{eq:generalizedcnot}
\sqrt{\ud(A)}\,\,
\begin{pic}[xscale=\tikzxscale, yscale=\tikzyscale]
\node (b) [graydot] at (0,0) {};
\node (w) [whitedot] at (1,1) {};
\draw (-0.75,2) to [out=down, in=left] (b.center);
\draw (b.center) to [out=right, in=left] (w.center);
\draw (w.center) to (1,2);
\draw (b.center) to (0,-1);
\draw (w.center) to [out=right, in=up] (1.75,-1);
\end{pic}
\end{equation}
Here we have assumed that we operate in a category where square roots of scalars exist.  The two algebras are complementary exactly when this composite is unitary, as we show in the following theorem.

\begin{theorem}[Complementarity via a unitary]
\label{thm:complementarityunitary}
  In a dagger symmetric monoidal category, two symmetric dagger-Frobenius algebras are complementary if and only if the composite \eqref{eq:generalizedcnot} is unitary.
\end{theorem}
\begin{proof}
  Composing ~\eqref{eq:generalizedcnot} with its adjoint in one order, we obtain the following:
\begin{equation}
\label{eq:generalizedcnotunitaryproof}
\begin{aligned}
  \begin{tikzpicture}[xscale=1.4*\tikzxscale, yscale=1.4*\tikzyscale]
  \node at (-1.6,-2.6) {$\ud (A)$};
  \node (A) at (0,0) {};
  \node (B) at (1.75,0) {};
  \node (b1) [graydot] at (0,-1) {};
  \node (w1) [whitedot] at (1,-2) {};
  \node (w2) [whitedot] at (1,-3) {};
  \node (b2) [graydot] at (0,-4) {};
  \node (C) at (0,-5) {};
  \node (D) at (1.75,-5) {};
  \draw (A.center) to (b1.center);
  \draw (b1.center) to [out=right, in=left] (w1.center);
  \draw (w1.center) to (w2.center);
  \draw (w2.center) to [out=left, in=right] (b2.center);
  \draw (b2.center) to (C.center);
  \draw (w2.center) to [out=right, in=up] (D.center);
  \draw (w1.center) to [out=right, in=down] (B.center);
  \draw (b1.center) to [out=left, in=left] (b2.center);
  \end{tikzpicture}
  \end{aligned}
  \stackrel{\mbox{\small Thm}~\ref{thm:spider}}{=}
  \begin{aligned}
  \begin{tikzpicture}[xscale=1.4*\tikzxscale, yscale=1.4*\tikzyscale]
  \node at (-2.1,-2.6) {$\ud (A)$};
  \node (A) at (-1.75,0) {};
  \node (B) at (0.5,0) {};
  \node (w1) [whitedot] at (0.5,-1.0) {};
  \node (w2) [whitedot] at (1.25,-3) {};
  \node (w3) [whitedot] at (0,-2) {};
  \node (b1) [graydot] at (-1,-2) {};
  \node (b2) [graydot] at (0,-4) {};
  \node (b3) [graydot] at (-0.5,-1) {};
  \node (C) at (0,-5) {};
  \node (D) at (2,-5) {};
  \draw (A.center) to [out=down, in=left] (b1.center);
  \draw (w1.center) to [out=right, in=up] (w2.center);
  \draw (w2.center) to [out=left, in=right] (b2.center);
  \draw (b2.center) to (C.center);
  \draw (w2.center) to [out=right, in=up] (D.center);
  \draw (w1.center) to [out=up, in=down] (B.center);
  \draw (b1.center) to [out=down, in=left] (b2.center);
  \draw (b1.center) to [out=right, in=left] (b3.center);
  \draw (b3.center) to [out=right, in=left] (w3.center);
  \draw (w3.center) to [out=right, in=left] (w1.center);
  \end{tikzpicture}
  \end{aligned}
  \stackrel{\eqref{eq:monoid}}{=}
  \begin{aligned}
  \begin{tikzpicture}[xscale=1.4*\tikzxscale, yscale=1.4*\tikzyscale]
  \node at (-2,-3.1) {$\ud (A)$};  
  \node (A) at (-1,0) {};
  \node (B) at (1.5,0) {};
  \node (w1) [whitedot] at (1.5,-1.0) {};
  \node (w2) [whitedot] at (1.0,-2) {};
  \node (w3) [whitedot] at (0.5,-3) {};
  \node (b1) [graydot] at (0.5,-4) {};
  \node (b2) [graydot] at (0,-5) {};
  \node (b3) [graydot] at (0,-2) {};
  \node (C) at (0,-6) {};
  \node (D) at (2.5,-6) {};
  \draw (A.center) to [out=down, in=left, in looseness=0.6] (b2.center);
  \draw (w1.center) to [out=left, in=up] (w2.center);
  \draw (w2.center) to [out=right, in=right] (b1.center);
  \draw (b2.center) to (C.center);
  \draw (w1.center) to [out=right, in=up, out looseness=0.6] (D.center);
  \draw (w1.center) to [out=up, in=down] (B.center);
  \draw (b1.center) to [out=down, in=right] (b2.center);
  \draw (b1.center) to [out=left, in=left] (b3.center);
  \draw (b3.center) to [out=right, in=left] (w3.center);
  \draw (w2.center) to [out=left, in=right] (w3.center);
  \end{tikzpicture}
  \end{aligned}  

\end{equation}
  If the complementarity condition~\eqref{eq:complementarity} holds then this is clearly the identity on \mbox{$A \otimes A$}. The other composite can be shown to be the identity in a similar way, and so~\eqref{eq:generalizedcnot} is unitary.

  Conversely, suppose~\eqref{eq:generalizedcnot} is unitary. Then the final expression of~\eqref{eq:generalizedcnotunitaryproof} certainly equals the identity on $A \otimes A$:
\begin{equation}
\begin{aligned}
\begin{tikzpicture}[xscale=1.4*\tikzxscale, yscale=1.1*\tikzyscale]
  \draw (0.5,-5) to (0.5,1);
  \draw (-1.5,-5) to (-1.5,1);
\end{tikzpicture}
\end{aligned}
\quad=\quad
\begin{aligned}
\begin{tikzpicture}[xscale=1.4*\tikzxscale, yscale=1.1*\tikzyscale]
  \node at (-1.9,-3.1) {$\ud (A)$};  
  \node (A) at (-1,0) {};
  \node (B) at (1.5,0) {};
  \node (w1) [whitedot] at (1.5,-1.0) {};
  \node (w2) [whitedot] at (1.0,-2) {};
  \node (w3) [whitedot] at (0.5,-3) {};
  \node (b1) [graydot] at (0.5,-4) {};
  \node (b2) [graydot] at (0,-5) {};
  \node (b3) [graydot] at (0,-2) {};
  \node (C) at (0,-6) {};
  \node (D) at (2.5,-6) {};
  \draw (A.center) to [out=down, in=left, in looseness=0.6] (b2.center);
  \draw (w1.center) to [out=left, in=up] (w2.center);
  \draw (w2.center) to [out=right, in=right] (b1.center);
  \draw (b2.center) to (C.center);
  \draw (w1.center) to [out=right, in=up, out looseness=0.6] (D.center);
  \draw (w1.center) to [out=up, in=down] (B.center);
  \draw (b1.center) to [out=down, in=right] (b2.center);
  \draw (b1.center) to [out=left, in=left] (b3.center);
  \draw (b3.center) to [out=right, in=left] (w3.center);
  \draw (w2.center) to [out=left, in=right] (w3.center);
\end{tikzpicture}
\end{aligned}

\end{equation}
 Composing with the black counit at the top-left and the white unit at the bottom-right then gives back complementarity condition~\eqref{eq:complementarity} as required:
\begin{equation}
  \begin{aligned}
  \begin{tikzpicture}[xscale=1.4*\tikzxscale, yscale=1.1*\tikzyscale]
  \node (w) [whitedot] at (-0.5,-4) {};
  \node (b) [graydot] at (-1.5,0) {}; 
  \draw (-0.5,-4) to (-0.5,1);
  \draw (-1.5,-5) to (-1.5,0);
  \end{tikzpicture}
  \end{aligned}
  \,\,\,\,= 
  \begin{aligned}
  \begin{tikzpicture}[xscale=1.4*\tikzxscale, yscale=1.1*\tikzyscale]
  \node at (-1.9,-3.1) {$\ud (A)$}; 
  \node (w) [graydot] at (-1,0) {};
  \node (b) [whitedot] at (2.5,-6) {}; 
  \node (A) at (-1,0) {};
  \node (B) at (1.5,0) {};
  \node (w1) [whitedot] at (1.5,-1.0) {};
  \node (w2) [whitedot] at (1.0,-2) {};
  \node (w3) [whitedot] at (0.5,-3) {};
  \node (b1) [graydot] at (0.5,-4) {};
  \node (b2) [graydot] at (0,-5) {};
  \node (b3) [graydot] at (0,-2) {};
  \node (C) at (0,-6) {};
  \node (D) at (2.5,-6) {};
  \draw (A.center) to [out=down, in=left, in looseness=0.6] (b2.center);
  \draw (w1.center) to [out=left, in=up] (w2.center);
  \draw (w2.center) to [out=right, in=right] (b1.center);
  \draw (b2.center) to (C.center);
  \draw (w1.center) to [out=right, in=up, out looseness=0.6] (D.center);
  \draw (w1.center) to [out=up, in=down] (B.center);
  \draw (b1.center) to [out=down, in=right] (b2.center);
  \draw (b1.center) to [out=left, in=left] (b3.center);
  \draw (b3.center) to [out=right, in=left] (w3.center);
  \draw (w2.center) to [out=left, in=right] (w3.center);
  \end{tikzpicture}
  \end{aligned}
  \,\,= \,\,  \mathrm{d}(A) \;
  \begin{pic}[xscale=1.4*\tikzxscale, yscale=1.4*\tikzyscale]
\draw (-0.5,0.25) to (-0.5,1) node [graydot] {} to [out=left, in=right] (-1,2) node [graydot] {} to [out=left, in=right] (-1.5,1.5) node [whitedot] {} to [out=left, in=down] (-2,2) to [out=up, in=left] (-0.75,3) node (a) [whitedot] {} to [out=right, in=right] (-0.5,1);
\draw (a.center) to +(0,0.75);
\end{pic}

\end{equation}
This completes the proof.
\end{proof}

\subsection{Families of unitary oracles}

This pair of complementary observables automatically gives rise to a much larger family of unitaries, one for each self-conjugate comonoid homomorphism onto one of the classical structures in the pair. See equation~\eqref{eq:comonoidhomomorphismselfconjugate} for the definition of the self-conjugacy property. Lemma~\ref{lem:comonoidhomomorphismselfconjugate} demonstrated that in \cat{FHilb}, every comonoid homomorphism of classical structures is self-conjugate.
\begin{defn}[Oracle]
\label{oracle}
In a symmetric monoidal dagger-category, given a dagger-Frobenius comonoid $\blackcomonoid{A}$, a pair of complementary symmetric dagger-Frobenius comonoids \graycomonoid{B} and \whitecomonoid{B}, and a self-conjugate comonoid homomorphism $f : \blackcomonoid{A} \to \graycomonoid{B}$, the \emph{oracle} is defined to be the following endomorphism of $A \otimes B$:
\begin{equation}
\label{eq:oracle}
\sqrt{\ud(A)}
\begin{pic}[xscale=1.8*\tikzxscale, yscale=1.3*\tikzyscale]
    \node (dot) [blackdot] at (0,1) {};
    \node (f) [morphism, wedge] at (0.7,2) {$f$};
    \node (m) [whitedot] at (1.4,3) {};
\draw (0,0.25)
        node [below] {$A$}
    to (0,1)
    to [out=left, in=south] (-0.7,2)
    to (-0.7,3.75)
        node [above] {$A$};
\draw (0,1)
    to [out=right, in=south] (f.south);
\draw  (f.north)
    to [out=up, in=left] (1.4,3)
    to [out=right, in=up] +(0.7,-1)
    to (2.1,0.25)
        node [below] {$B$};;
\draw (m.center) to +(0,0.75) node [above] {$B$};
\end{pic}

\end{equation}
\end{defn}

\begin{theorem}
\label{thm:familyofunitaries}
Oracles are unitary.
\end{theorem}
\begin{proof}
To demonstrate that the oracle \eqref{eq:oracle} is unitary, we must compose it with its adjoint on both sides and show that we get the identity in each case. In one case, we obtain the following, making use of the Frobenius laws, self-conjugacy of $f$, associativity and coassociativity, the fact that $f$ preserves comultiplication, the complementarity condition, the fact that $f$ preserves the counit, and the unit and counit laws:
\begin{align*}
&\ud(A)
\begin{aligned}
\begin{tikzpicture}[xscale=2*\tikzxscale, yscale=1.2*\tikzyscale]
\node (A) at (0,2) {};
\node (B) at (1.75,0) {};
\node (b1) [blackdot] at (0,1) {};
\node (w1) [whitedot] at (1,-1) {};
\node (w2) [whitedot] at (1,-2) {};
\node (b2) [blackdot] at (0,-4) {};
\node (C) at (0,-5) {};
\node (D) at (1.75,-5) {};
\node (f1) [morphism, wedge] at (0.5,-3) {$f$};
\node (f2) [morphism, wedge, hflip] at (0.5,0) {$f$};
\draw (A.center) to (b1.center);
\draw (b1.center) to [out=right, in=up] (f2.north);
\draw (f2.south) to [out=down, in=left] (w1.center);
\draw (w1.center) to (w2.center);
\draw (w2.center) to [out=left, in=up] (f1.north);
\draw (b2.center) to (C.center);
\draw (w2.center) to [out=right, in=up] (1.5,-3 |- f1.north) to (1.5,-5);
\draw (w1.center) to [out=right, in=down] (1.5,1 |- f2.south) to (1.5,2);
\draw (b1.center) to [out=left, in=up] (-0.5,0 |- f2.north) to (-0.5,0 |- f1.south) to [out=down, in=left] (b2.center);
\draw (f1.south) to [out=down, in=right] (b2.center);
\end{tikzpicture}
\end{aligned}
=\,\,
\ud(A)
\hspace{-3pt}
\begin{aligned}
\begin{tikzpicture}[xscale=2.3*\tikzxscale, yscale=1.4*\tikzyscale]
\node (f1) [morphism, wedge] at (0.5,-4) {$f$};
\node (f2) [morphism, wedge, hflip] at (-0.5,-2) {$f$};
\node (A) at (-2,0) {};
\node (B) at (0.5,0) {};
\node (w1) [whitedot] at (0.5,-2.0) {};
\node (w2) [whitedot] at (1.0,-3) {};
\node (w3) [whitedot] at (-0.125,-3) {};
\node (b1) [blackdot] at (-1.5,-3) {};
\node (b2) [blackdot] at (0,-5) {};
\node (b3) [blackdot] at (-0.75,-1) {};
\node (C) at (0,-6) {};
\node (D) at (1.5,-6) {};
\draw (A.center) to +(0,-1.5) to [out=down, in=left] (b1.center);
\draw (w1.center) to [out=right, in=up] (w2.center);
\draw (w2.center) to [out=left, in=up] (f1.north);
\draw (f1.south) to [out=down, in=right] (b2.center);
\draw (b2.center) to (C.center);
\draw (w2.center) to [out=right, in=up] (D.center |- f1.north) to (D.center);
\draw (w1.center) to [out=up, in=down] (B.center);
\draw (b1.center) to [out=down, in=left] (b2.center);
\draw (b1.center) to [out=right, in=left] (b3.center);
\draw (f2.south) to [out=down, in=left] (w3.center);
\draw (w3.center) to [out=right, in=left] (w1.center);
\draw (b3.center) to [out=right, in=up] (f2.north);
\end{tikzpicture}
\end{aligned}
=\,\,
\ud(A)
\hspace{-3pt}
\begin{aligned}
\begin{tikzpicture}[xscale=2.3*\tikzxscale, yscale=1.4*\tikzyscale]
\node (f1) [morphism, wedge] at (0.5,-4) {$f$};
\node (f2) [morphism, wedge] at (-1,-2) {$f$};
\node (A) at (-2,0) {};
\node (B) at (0.68,0) {};
\node (w1) [whitedot] at (0.68,-2.0) {};
\node (w2) [whitedot] at (1.0,-3) {};
\node (w3) [whitedot] at (0.25,-3) {};
\node (b1) [blackdot] at (-1.5,-3) {};
\node (b2) [blackdot] at (0,-5) {};
\node (b3) [graydot] at (-0.5,-1) {};
\node (C) at (0,-6) {};
\node (D) at (1.5,-6) {};
\draw (A.center) to +(0,-1.5) to [out=down, in=left] (b1.center);
\draw (w1.center) to [out=right, in=up] (w2.center);
\draw (w2.center) to [out=left, in=up] (f1.north);
\draw (f1.south) to [out=down, in=right] (b2.center);
\draw (b2.center) to (C.center);
\draw (w2.center) to [out=right, in=up] (D.center |- f1.north) to (D.center);
\draw (w1.center) to [out=up, in=down] (B.center);
\draw (b1.center) to [out=down, in=left] (b2.center);
\draw (w3.center) to [out=left, in=right] (b3.center);
\draw (f2.south) to [out=down, in=right] (b1.center);
\draw (w3.center) to [out=right, in=left] (w1.center);
\draw (b3.center) to [out=left, in=up] (f2.north);
\end{tikzpicture}
\end{aligned}
\\
&
\hspace{2cm}
= \,\,\ud(A)
\begin{aligned}
\begin{tikzpicture}[xscale=2.6*\tikzxscale, yscale=1.4*\tikzyscale]
\node (f1) [morphism, wedge] at (-0.25,-3) {$f$};
\node (f2) [morphism, wedge] at (1.25,-3) {$f$};
\node (A) at (-1,0) {};
\node (B) at (1.5,0) {};
\node (w1) [whitedot] at (1.5,-1.0) {};
\node (w2) [whitedot] at (1.0,-2) {};
\node (w3) [whitedot] at (0.5,-2.5) {};
\node (b1) [blackdot] at (0.5,-4) {};
\node (b2) [blackdot] at (0,-5) {};
\node (b3) [graydot] at (0,-2) {};
\node (C) at (0,-6) {};
\node (D) at (2.25,-6) {};
\draw (A.center) to [out=down, in=left, in looseness=0.59] (b2.center);
\draw (w1.center) to [out=left, in=up] (w2.center);
\draw (w2.center) to [out=right, in=up] (f2.north);
\draw (b2.center) to (C.center);
\draw (w1.center) to [out=right, in=up, out looseness=0.45] (D.center);
\draw (w1.center) to [out=up, in=down] (B.center);
\draw (b1.center) to [out=down, in=right] (b2.center);
\draw (b1.center) to [out=left, in=down] (f1.south);
\draw (b3.center) to [out=right, in=left] (w3.center);
\draw (w2.center) to [out=left, in=right] (w3.center);
\draw (f1.north) to [out=up, in=left] (b3.center);
\draw (f2.south) to [out=down, in=right] (b1.center);
\end{tikzpicture}
\end{aligned}
= \,\,\ud(A)
\begin{aligned}
\begin{tikzpicture}[xscale=2.6*\tikzxscale, yscale=1.4*\tikzyscale]
\node (f) [morphism, wedge] at (0.5,-4) {$f$};
\node (A) at (-0.5,0) {};
\node (B) at (1.5,0) {};
\node (w1) [whitedot] at (1.5,-1.0) {};
\node (w2) [whitedot] at (1.0,-2) {};
\node (w3) [whitedot] at (0.5,-2.5) {};
\node (b1) [graydot] at (0.5,-3.25) {};
\node (b2) [blackdot] at (0,-5) {};
\node (b3) [graydot] at (0,-2) {};
\node (C) at (0,-6) {};
\node (D) at (2,-6) {};
\draw (A.center) to +(0,-4) to [out=down, in=left] (b2.center);
\draw (w1.center) to [out=left, in=up] (w2.center);
\draw (w2.center) to [out=right, in=right] (b1.center);
\draw (b2.center) to (C.center);
\draw  (D.center) to +(0,4) to [out=up, in=right] (w1.center);
\draw (w1.center) to [out=up, in=down] (B.center);
\draw (b1.center) to [out=down, in=up] (f.north);
\draw (f.south) to [out=down, in=right] (b2.center);
\draw (b1.center) to [out=left, in=left] (b3.center);
\draw (b3.center) to [out=right, in=left] (w3.center);
\draw (w2.center) to [out=left, in=right] (w3.center);
\end{tikzpicture}
\end{aligned}
\\
&
\hspace{2cm}
=\hspace{5pt}
\begin{aligned}
\begin{tikzpicture}[xscale=2.6*\tikzxscale, yscale=1.2*\tikzyscale]
\node (f) [morphism, wedge] at (0.5,-4) {$f$};
\node (A) at (-0.5,0) {};
\node (B) at (1.5,0) {};
\node (w1) [whitedot] at (1.5,-1.0) {};
\node (w2) [whitedot] at (1.0,-2) {};
\node (b1) [graydot] at (0.5,-3) {};
\node (b2) [blackdot] at (0,-5) {};
\node (C) at (0,-6) {};
\node (D) at (2,-6) {};
\draw (A.center) to +(0,-4) to [out=down, in=left] (b2.center);
\draw (w1.center) to [out=left, in=up] (w2.center);
\draw (b2.center) to (C.center);
\draw  (D.center) to +(0,4) to [out=up, in=right] (w1.center);
\draw (w1.center) to [out=up, in=down] (B.center);
\draw (b1.center) to [out=down, in=up] (f.north);
\draw (f.south) to [out=down, in=right] (b2.center);
\end{tikzpicture}
\end{aligned}
\hspace{5pt}=\hspace{5pt}
\begin{aligned}
\begin{tikzpicture}[xscale=2.6*\tikzxscale, yscale=1.2*\tikzyscale]
\node (A) at (-0.5,0) {};
\node (B) at (1.5,0) {};
\node (w1) [whitedot] at (1.5,-1.0) {};
\node (w2) [whitedot] at (1.0,-2) {};
\node (b1) [blackdot] at (0.5,-4) {};
\node (b2) [blackdot] at (0,-5) {};
\node (C) at (0,-6) {};
\node (D) at (2,-6) {};
\draw (A.center) to +(0,-4) to [out=down, in=left] (b2.center);
\draw (w1.center) to [out=left, in=up] (w2.center);
\draw (b2.center) to (C.center);
\draw  (D.center) to +(0,4) to [out=up, in=right] (w1.center);
\draw (w1.center) to [out=up, in=down] (B.center);
\draw (b1.center) to [out=down, in=right] (b2.center);
\end{tikzpicture}
\end{aligned}
\hspace{5pt}=\hspace{5pt}
\begin{aligned}
\begin{tikzpicture}[xscale=2*\tikzxscale, yscale=1.2*\tikzyscale]
\draw (0,0) to (0,6);
\draw (1.5,0) to (1.5,6);
\end{tikzpicture}
\end{aligned}
\end{align*}
There is a similar argument that the other composite also gives the identity. \end{proof}

\section{The Deutsch-Jozsa algorithm}
The abstract structure of the Deutsch-Jozsa quantum algorithm is presented and its function is verified abstractly, following~\cite{vicary-tqa}. This proof inspires new ways to implement the DJ algorithm, where Fourier transforms are replaced by more general Hadamard transformations, without effecting the algorithm. We investigate these new implementations numerically and analytically. {\bf Status:} undrafted, though the proof and numerical results have been completed.

\section{Grover's algorithm}
For completeness, we summarize the results of \cite{vicary-tqa}, where the single-shot Grover's algorithm is investigated using categorical techniques. {\bf Status:} undrafted.

\section{The hidden subgroup algorithm}
Similarly, we abstractly demonstrate the function of the hidden subgroup algorithm, which leads us to conjecture that a speedup is possible by altering an input query representation. {\bf Status:} undrafted.

\section{\color{blue} The group homomorphism identification algorithm}

\subsection{The problem}

In this section we construct a new deterministic quantum algorithm to identify  group homomorphisms.  
\begin{defn}[Group homomorphism identification problem]
Given finite groups $G$ and $A$ where $A$ is abelian, and a blackbox function $f:G\to A$ that is promised to be a group homomorphism, identify the homomorphism $f$.
\end{defn}

\noindent
We will define a quantum algorithm that solves the group homomorphism identification problem with a number of queries equal to the number of simple factors of the abelian group $A$.

For comparison, we can consider the obvious classical algorithm for this problem.
\begin{lemma}
Given finite groups $G$ and $A$, where $A$ is abelian and $G$ has a generating set of order $m$, and a blackbox function $f:G\to A$ that is promised to be a group homomorphism, a classical algorithm can determine $f$ with $m$ oracle queries.
\end{lemma}
\begin{proof}
Once we have evaluated $f$ classically on the generating set of $G$, we have fully characterized~$f$. 
\end{proof}

\noindent
We are unable to prove optimality in either the quantum or classical case. However, we note that the query complexities of these quantum and classical algorithms depend of different and unrelated parameters of the problem. Instances where the order of the generating set of $G$ is larger than the number of factors in the target group $A$ will demonstrate a quantum advantage. 

In  the simpler case where $G$ is an abelian group this quantum algorithm was previously described by H\o yer \cite{hoyer1999conjugated}, though his algebraic presentation differs significantly from ours. H\o yer also notes that the algorithm by Bernstein and Vazirani in~\cite{bernstein1993quantum} is an instance of the abelian group identification problem where $G=\mathbb{Z}_n^n$ and $A=\mathbb{Z}_2$. Independently, Cleve et. al.~\cite{cleve1998quantum} also presented an algorithm for the abelian case where $G=\mathbb{Z}_2^n$ and $A=\mathbb{Z}_2^m$.

%Our results give a new approach to the solution of the group homomorphism identification problem that both extends the existing results to the case where $G$ is non-abelian, and clearly connects the structure of our algorithm to that of other black-box quantum algorithms, such as the Deutsch-Jozsa and hidden subgroup algorithms.

We will proceed using the abstract structure defined earlier, but will now work in the dagger-symmetric monoidal category {\bf FHilb}. Any choice of orthonormal basis for  an object $A$ in {\bf FHilb} endows it with a dagger-Frobenius algebra $(A,\tinymult[blackdot],\tinyunit[blackdot])$, whose copying map $d: A \to A\otimes A$ is defined as the linear extension of $d(|i\rangle)=|i\rangle\otimes|i\rangle$. Any finite group $G$ induces a different dagger-Frobenius algebra on an object $A=\mathbb{C}[G]$, the Hilbert space with orthonormal basis given by the elements $G$, with multiplication given by linear extension of the group multiplication; we represent this structure as~$(A, \tinymult[whitedot], \tinyunit[whitedot])$. These two Frobenius algebras are complementary.

\def\Mat{\mathrm{Mat}}
In the case that $G$ is finite, its representations can be characterized as the homomorphisms \mbox{$G \sxto \rho \mbox{Mat}(n)$}. The homomorphism conditions take the following form~\cite[Section~A.7]{vicary-tqa}:
\begin{calign}
\label{eq:rhocopied}
\begin{aligned}
\begin{tikzpicture}[xscale=2*\tikzxscale, yscale=2*\tikzyscale]
\draw (-0.7,-1) node [below] {$G$} to [out=up, in=\swangle] (0,0);
\draw (0.7,-1) node [below] {$G$} to [out=up, in=\seangle] (0,0);
\draw (0,0) to (0,0.75);
\node (m) at (0,0) [whitedot] {};
\node (rho) at (0,0.75) [morphism, wedge, width=0, anchor=south] {$\rho$};
\draw ([xshift=5pt] rho.north) to +(0,0.70);
\draw ([xshift=-5pt] rho.north) to +(0,0.70);
\node at (0,2.25) [anchor=south] {$\Mat(n)$};
\end{tikzpicture}
\end{aligned}
\quad=\quad
\begin{aligned}
\begin{tikzpicture}[xscale=2*\tikzxscale, yscale=2*\tikzyscale]
\node (r1) at (0,1.5) [morphism, wedge] {$\rho$};
\node (r2) at (1.5,1.5) [morphism, wedge] {$\rho$};
\draw (0,0) node [below] {$G$} to (r1.south);
\draw (1.5,0) node [below] {$G$} to (r2.south);
\draw ([xshift=5pt] r1.north) to [out=up, in=up] ([xshift=-5pt] r2.north);
\draw ([xshift=-5pt] r1.north) to [out=up, in=down, in looseness=1] (0.55,3.25);
\draw ([xshift=5pt] r2.north) to [out=up, in=down, in looseness=1] (0.95,3.25);
\node [above] at (0.75,3.25) {$\Mat(n)$};
\end{tikzpicture}
\end{aligned}
&
\begin{aligned}
\begin{tikzpicture}[xscale=2*\tikzxscale, yscale=2*\tikzyscale]
\draw (0,-0.55) node [whitedot] {} to +(0,1) node (r1) [morphism, wedge, anchor=south] {$\rho$};
\draw ([xshift=-5pt] r1.north) to +(0,1);
\draw ([xshift=5pt] r1.north) to +(0,1);
\node at (0.0,2.25) [above] {$\Mat(n)$};
\node [below, white] at (0,-1) {$G$};
\end{tikzpicture}
\end{aligned}
\quad=\quad
\begin{aligned}
\begin{tikzpicture}[xscale=2*\tikzxscale, yscale=2*\tikzyscale]
\draw (0,0) to (0,-1) to [out=down, in=down, looseness=2] (0.5,-1) to (0.5,0);
\node at (0.25,0) [above] {$\Mat(n)$};
\node [below, white] at (0.25,-3.25) {$G$};
\end{tikzpicture}
\end{aligned}

\end{calign}
These will be essential for our proofs below.

\subsection{The algorithm}

The structure of the quantum algorithm that solves the group homomorphism identification problem is given by the topological diagram~\eqref{eq:theAlg} below. Here $\sigma:G\to\mathbb{C}$ is a normalized irreducible representation of $G$, representing the result of the measurement, and $\rho:A\to\mathbb{C}$ is a normalized irreducible representation of $A$. The representation $\rho$ is one-dimensional as $A$ is an abelian group. Physically, we are able to produce the input  state $\rho$ efficiently, using $O(\log n)$ time steps, via the quantum Fourier transform for any finite abelian group~\cite{cleve2000fast}. The measurement  result $\sigma$ arises from  a measurement in the Fourier basis, which can, by a similar procedure for any finite group~\cite{childs2010quantum}, also be implemented efficiently.
\begin{align}
\label{eq:theAlg}
\begin{aligned}
\begin{tikzpicture}[xscale=2*\tikzxscale, yscale=2*\tikzyscale]
    \node (dot) [blackdot] at (0,1) {};
    \node (f) [morphism, wedge] at (0.7,2) {$f$};
    \node (m) [whitedot] at (1.4,3) {};
    \node (topsig) [morphism, fill=white, wedge, anchor=south] at (-0.7,3.6) {$\sigma$};
\draw ([xshift=5pt] topsig.north) to +(0,0.3);
\draw ([xshift=-5pt] topsig.north) to +(0,0.3);     
\draw (0,0.4)
        node [blackdot] {}
        node [anchor=20] {$\frac 1 {\sqrt{|G|}}$}
    to (0,1)
    to [out=left, in=south] (-0.7,2)
    to (topsig.south);   
\draw (0,1)
    to [out=right, in=south] (f.south);
\draw  (f.north)
    to [out=up, in=left] (1.4,3)
    to [out=right, in=up] +(0.7,-1)
    to (2.1,0.4)
        node [morphism, wedge, hflip, anchor=north] {$\rho$};
\draw (m.center) to (1.4,4.4)
        node [above] {};
\draw [thin, lightgray] (-1.25,0.7) to (7.5,0.7);
\draw [thin, lightgray] (-1.25,3.3) to (7.5,3.3);
\node at (3,0) [anchor=west] {Prepare initial states};
\node at (3,2) [anchor=west] {Apply a unitary map};
\node at (3,4) [anchor=west] {Measure the left system};
\node at (-0.7,2) [anchor=east] {$\sqrt{|G|}$};
\end{tikzpicture}
\end{aligned}

\end{align}

We can compare the structure of this algorithm to that of the standard quantum algorithm for the hidden subgroup problem. There, the second system is prepared in a state given by the identity element of the group, corresponding to a uniform linear combination of the irreducible representations.  A later measurement of this second system---which is not a part of the standard hidden subgroup algorithm, but can be done without changing the result of the procedure---would collapse this combination to a classical mixture of these representations. The hidden subgroup algorithm therefore contains an amount of classical nondeterminism in its initial setup. In principle removing this, and selecting the input representation strategically, can only improve performance, and we take advantage of this here.

We analyze the effect of our new algorithm as follows.
\begin{lemma}
The algorithm defined by~\eqref{eq:theAlg} gives output $\sigma$ with probability given by the square norm of~$\sigma\circ f^*\circ\rho^*$.
\end{lemma}
\begin{proof}
Using that $\rho$ is a group homomorphism and simple diagrammatic rewrites defined in~\cite[Section~A.9]{vicary-tqa},
 we show the following, making use of the fact that representations are copyable points for group multiplication:
\begin{align}
\label{simplifyAlg}
\begin{aligned}
\begin{tikzpicture}[xscale=2*\tikzxscale, yscale=2*\tikzyscale]
\draw [use as bounding box, draw=none] (-0.3,0.6) rectangle +(3.45,3.7);
    \node (f) [morphism, wedge] at (1.25,2) {$f$};
    \node (s) [morphism, wedge] at (0,3.5) {$\sigma$};
\node (r) at (2.5,0.75) [morphism, wedge, hflip] {$\rho$};
\draw (f.south) to [out=down, in=right] +(-0.625,-0.5) node (b) [blackdot] {} to [out=left, in=down] +(-0.625, 0.5) to (s.south);
\draw (b.center) to +(0,-0.5) node [blackdot] {};
\draw ([xshift=4pt] s.north) to +(0,0.5);
\draw ([xshift=-4pt] s.north) to +(0,0.5);
\draw (f.north) to [out=up, in=left] +(0.625,0.5) node (w) [whitedot] {} to [out=right, in=up] +(0.625,-0.5) to (r.north);
\draw (w.center) to +(0,1.5);
\end{tikzpicture}
\end{aligned}
\quad=\quad
\begin{aligned}
\begin{tikzpicture}[xscale=2*\tikzxscale, yscale=2*\tikzyscale]
\draw [use as bounding box, draw=none] (-0.3,0.6) rectangle +(3.45,3.7);
\node (s) [morphism, wedge] at (0,3.5) {$\sigma$};
\node (f) [morphism, wedge] at (1.25,2) {$f$}; 
\node (r) at (1.25, 3) [morphism, wedge] {$\rho$};
\node (r2) at (2.5,3.5) [morphism, wedge, hflip] {$\rho$};
\draw (r2.north) to +(0,0.5);
\draw ([xshift=4pt] s.north) to +(0,0.5);
\draw ([xshift=-4pt] s.north) to +(0,0.5);
\draw (r.south) to (f.north);
\draw (f.south) to [out=down, in=right] +(-0.625,-0.5) node (b) [blackdot] {} to [out=left, in=down] +(-0.625, 0.5) to (s.south);
\draw (b.center) to +(0,-0.5) node [blackdot] {};
\end{tikzpicture}
\end{aligned}
\quad=\quad
\begin{aligned}
\begin{tikzpicture}[xscale=2*\tikzxscale, yscale=2*\tikzyscale]
\draw [use as bounding box, draw=none] (-0.66,0.6) rectangle +(2.75,3.7);
\node (r) [morphism, wedge, hflip, vflip] at (0,1.5) {$\rho$};
\node (s) [morphism, wedge] at (0,3.5) {$\sigma$};
\node (f) [morphism, wedge, hflip, vflip] at (0,2.5) {$f$};
\node (r2) at (1.5,3.5) [morphism, wedge, hflip] {$\rho$};
\draw ([xshift=4pt] s.north) to +(0,0.5);
\draw ([xshift=-4pt] s.north) to +(0,0.5);
\draw (s.south) to (f.north);
\draw (f.south) to (r.north);
\draw (r2.north) to +(0,0.5);
\end{tikzpicture}
\end{aligned}

\end{align}
The left hand system is thus in the state $\sigma\circ f^*\circ\rho^*$, and using the Born rule, the squared norm of this state gives the probability of this experimental outcome.
\end{proof}

\begin{lemma}\label{lem:irrep}
The composite $\rho\circ f$ is an irreducible representation of $G$.
\end{lemma}
\begin{proof}
The map $f$ is a homomorphism, so $\rho\circ f:G\to\mathbb{C}$ is a one-dimensional representation of $G$. All one-dimensional representations are irreducible, so $\rho\circ f$ is an irreducible representation.
\end{proof}

% \begin{lemma}
% The fundamental representation of a group is a monomorphism.
% \end{lemma}

\begin{lemma}
\label{lem:equaliso}
One-dimensional representations are equivalent only if they are equal.
\end{lemma}
\begin{proof}
Let $\rho_1,\rho_2:G\to \mathbb{C}$ be irreducible representations of $G$. If they are isomorphic, then there exists a linear map $\mathcal{L}:\mathbb{C}\to\mathbb{C}$, i.e. some complex number, such that $\forall g\in G$
$$\mathcal{L}\rho_1(g) = \rho_2(g)\mathcal{L}.$$
Hence we see that $\forall g\in G$, $\rho_1(g) = \rho_2(g)$.
\end{proof}

\begin{theorem}[Structure theorem for finite abelian groups]
\label{thm:structure}
Every finite abelian group is isomorphic to a direct product of cyclic groups of prime power order.
\end{theorem}
\begin{proof}
See~\cite[Theorem 6.4]{artin-algebra} for a proof of this standard result.
\end{proof}

\begin{theorem}\label{rightCyclic}
For a finite group $G$ and cyclic group of prime power order $\mathbb{Z}_{p^n}$, the algorithm~\eqref{eq:theAlg} identifies a group homomorphism $f:G\to \mathbb{Z}_{p^n}$ in a single query.
\end{theorem}
\begin{proof}
Choose the input representation $\rho$ to be the fundamental representation of $\mathbb{Z}_{p^n}$. This representation is faithful.  This means exactly that 
\[ \rho\circ f = \rho\circ f' \qquad \Leftrightarrow \qquad f=f'. \]
Thus $\rho\circ f$ and $\rho\circ f'$ are different irreducible representations if and only if$f$ and $f'$ are different group homomorphisms.  The single measurement on the state $(\rho\circ f)^*$ is performed by the algorithm in the representation basis of $G$, allowing us to determine $\rho\circ f$ up to isomorphism. Due to Lemma~\ref{lem:equaliso} we know that each equivalence class contains only one representative, and thus we can determine $f$ with a single query.
\end{proof}

\begin{theorem}\label{thm:intoAbThm}
For any two finite groups $G$ and $A$, where $A$ is abelian with $n$ simple factors, the quantum algorithm~\eqref{eq:theAlg} can identify a group homomorphism $f:G \to A$ with $n$ oracle queries.
\end{theorem}
\begin{proof}
We prove the result by induction. 
\newline\newline
\noindent{\bf Base case.} When $A=\mathbb{Z}_{p^n}$ is simple, then by Theorem~\ref{rightCyclic} we can identify the homomorphism with a single query.
\newline\newline
\noindent{\bf Inductive step.} If $A$ is not simple, then we must have $A=H_1\times H_2$ by Theorem~\ref{thm:structure}, where the following hold:
\begin{enumerate}

\item The product $\times$ is the direct product whose projectors ($p_1,p_2$) are homomorphisms.

\item  $H_1$ and $H_2$ are groups with $n_1$ and $n_2$ factors respectively such  that the theorem holds, i.e. homomorphisms of the type $f_{1}:G\to H_1$ and $f_{2}:G\to H_2$  can be identified in $n_1$ and $n_2$ queries respectively.

\end{enumerate} 
Since $p_1\circ f$ and $p_2\circ f$ are homomorphisms, we can run subroutines of the algorithm to determine them. Hence we recover $f$ as
\begin{align*}
f(x) = ( (p_1\circ f)(x),(p_2\circ f)(x) ).
\end{align*}
The first subroutine will require $n_1$ queries and the second will require $n_2$ queries, so the total number of queries will be $n_1+n_2$, which is the number of factors of $H_1\times H_2$.
\end{proof}

\ignore{In practice we run the algorithm once on each $k$-th factor of $A$ to determine homomorphisms 
\[ p_k\circ f = f_k:G\to H_k \]
We then know $f$ as 
\begin{align*}
f(n_0,n_1,...n_{n-1}) = \left(f_0(n_0), f_1(n_1),..., f_{n-1}(n_{n-1})\right)
\end{align*}
we know that each isomorphic class contains only one representation}

\todo{\subsection{Optimality}}

\subsection{Extension to the non-abelian case}
We now consider the more general case where the target group $A$ is non-abelian. We do not know how to extend the algorithm described above to this case.  Nevertheless, it is instructive to analyze this scenario in our graphical approach. 

Irreducible representations of a non-abelian group $A$ are not necessarily one dimensional, though we are still able to compute them via the Fourier transform efficiently \cite{childs2010quantum}. In this case the algorithm has the following structure, where $\psi$ represents the initial state of the right-hand system in the representation space:
\begin{equation}
\label{eq:NonAbAlg}
\begin{aligned}
\begin{tikzpicture}[xscale=2*\tikzxscale, yscale=2*\tikzyscale]
\draw [use as bounding box, draw=none] (-0.3,-0.3) rectangle +(3.45,4.55);
\node (f) [morphism, wedge] at (1.25,2) {$f$};
\node (topsig) [morphism, wedge] at (0,3.5) {$\sigma$};
\node (r) at (2.5,0.75) [morphism, wedge, hflip] {$\rho$};
\draw (f.south) to [out=down, in=right] +(-0.625,-0.5) node (b) [blackdot] {} to [out=left, in=down] +(-0.625, 0.5) to (s.south);
\draw (b.center) to +(0,-0.5) node [blackdot] {} node [anchor=east] {$\frac {1} {\sqrt{G}}$};
\draw ([xshift=4pt] s.north) to +(0,0.5);
\draw ([xshift=-4pt] s.north) to +(0,0.5);
\draw (f.north) to [out=up, in=left] +(0.625,0.5) node (w) [whitedot] {} to [out=right, in=up] +(0.625,-0.5) to (r.north);
\draw (w.center) to +(0,1.5);
\draw ([xshift=4pt] r.south) to +(0,-0.3);
\draw ([xshift=-4pt] r.south) to +(0,-0.3);
\node [morphism, wedge, anchor=north] at ([yshift=-0.3cm] r.south) {$\psi$};
\end{tikzpicture}
\end{aligned}
\quad=\quad
\begin{aligned}
\begin{tikzpicture}[xscale=2*\tikzxscale, yscale=2*\tikzyscale]
\draw [use as bounding box, draw=none] (-0.3,-0.3) rectangle +(3.45,4.55);
\node (s) [morphism, wedge] at (0,3.5) {$\sigma$};
\node (f) [morphism, wedge] at (1.25,1.2) {$f$}; 
\node (r) at (1.25, 2.15) [morphism, wedge] {$\rho$};
\node (r2) at (2.5,3.5) [morphism, wedge, hflip] {$\rho$};
\draw (r2.north) to +(0,0.5);
\draw ([xshift=4pt] s.north) to +(0,0.5);
\draw ([xshift=-4pt] s.north) to +(0,0.5);
\draw (r.south) to (f.north);
\draw (f.south) to [out=down, in=right] +(-0.625,-0.5) node (b) [blackdot] {} to [out=left, in=down] +(-0.625, 0.5) to (s.south);
\draw (b.center) to +(0,-0.5) node [blackdot] {};
\node (psi) [morphism, wedge, anchor=north] at (2.5,0.2) {$\psi$};
\draw ([xshift=4pt] psi.north) to ([xshift=4pt] r2.south);
\draw ([xshift=-4pt] psi.north) to ([xshift=-4pt] r.north -| r2.south) to [out=up, in=up] ([xshift=4pt] r.north);
\draw ([xshift=-4pt] r.north) to [out looseness=1.3, out=up, in=down, in looseness=0.8] ([xshift=-4pt] r2.south);
\end{tikzpicture}
\end{aligned}

\end{equation}
We notice two additional features in this case. First, it is clear that the left and right systems are no longer in a product state at the end of the protocol, as they were in the final diagram of \eqref{simplifyAlg}. Second, we now have an additional choice when preparing the input representation $\rho$; in order to construct a state from a representation $\rho$ we also must choose the state $\psi$.

While this provides a clear description of the algorithm in this more general setting, it is not clear that it would identify homomorphisms into non-abelian groups. Complications include the lack of a structure theorem that satisfies the conditions for Theorem~\ref{thm:intoAbThm}, and that Lemma~\ref{lem:irrep} no longer applies.  In this setting it may be useful to make the problem easier by restricting to the identification of homomorphisms up to \emph{natural isomorphism}, i.e. where two homomorphisms $f_1,f_2:G\to H$ are considered equivalent when there exists some $\eta\in H$ such that, for all $g\in G$, we have $\eta f_1(g) \eta^{-1} = f_2(g)$.

\section{\color{blue} The hidden shift problem}
{\bf Status:} UNDRAFTED.

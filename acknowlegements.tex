\begin{acknowledgementslong}
Research is humbling work, and I mean this in the proudest way. I am deeply grateful for the time I've been given to pursue it.  In no way would this have been possible without the dedication, encouragement, and support of many.

I begin with thanks to Michel Devoret and Rob Schoelkopf at Yale, and Andreas Wallraff at ETH Zurich, all of whom, during my undergraduate years, made quantum computation a reality for me.  Prof. Devoret has my particular thanks, as, with considerable patience and insight, he witnessed my (at times bouncing) trajectory over the field since I first wrestled with the abelian hidden subgroup algorithm in his office in the summer of 2008. This is where my fascination with the structure of quantum algorithms began.

My mature study of the subject is with the tools and language taught to me in huge part, by Samson Abramsky, Bob Coecke, Chris Heunen, Aleks Kissinger, and Jamie Vicary. The support and self-directed agency that my supervisors Bob and Jamie encouraged in my studies were crucial. Bob has an independent sense, which, while rare enough as it is, is all the rarer with his match of openness and generosity; and I don't just mean at Club 13, Beijing. Jamie's eye for detail and deliberate approach were especially invaluable in guiding my initial transition from student-with-ideas to researcher. Our first work directly together (Section~\ref{sec:blackbox} in this thesis) gave me the initial confidence that has culminated in this thesis.  Special thanks also goes to Stefano Gogioso. Even though we only have been working together for a year, he has been an important catalyst, and the backlog of our ideas is ever growing.  How long would my working papers (esp. on Mermin non-locality) have sat alone in my office desk if he and Andre Ranchin had not continued to ask over them. Indeed this sense of shared ideas is one of the aspects I am most grateful for. 

The approach to the framework presented here - for me an initially broad leap of mathematical background - would not have been in any way possible without my cohort in the Quantum Group at Oxford, especially Miriam Backens, Katriel Cohn-Gordon, Brendan Fong, Amar Hadzihasanovic, Dan Marsden, Shane Mansfield, Nadish de Silva, and Vladimir Zamdzhiev, who were learning alongside me. Other researchers in our group, including Niel de Beaudrap, Matty Hoban, Clare Horsman, Kohei Kishida, and Ray Lal have made useful sounding boards for the work developed here, and I thank them for this.

There are many others young researchers that I have met at conferences and workshops during the course of my thesis study, who have helped me situate and clarify the ideas presented here. A, by no means exhaustive list, includes Ning Bao, Alex Kubica, Shaun Maguire, Raghu Mahajan, Sandra Rankovic, Jonathan Skowera, Nathaniel Thomas, and Michael Walter, and I am happy to count them as my friends. I would also like to especially thank Stephen Jordan and Ronald de Wolf for their suggestions and interest in the quantum algorithms work presented in Section~\ref{sec:blackbox}.

My work in the final chapter of this thesis, on structural connections between quantum computation and a model in natural language processing, required quick acclimatization in a new field. For this I thank Stephen Clark, Dimitri Kartsaklis, Tamara Polajnar, Mehrnoosh Sadrzadeh and everyone else in the DisCoCat project for their openness and inclusiveness.

My academic life has benefited from the support of many other dear friends throughout my time at Oxford.  My particular gratitude is to Merritt Moore, Ankur Desai, David Furlong, Andrew Lanham, Victor Pontis, Bary Pradelski, Max Roser, Spencer Salovaara, Lucas Zwirner, my entire OUBC family, my Rhodie family, Wire \& String attendees and many others who I have had the privilege to invite to dinner. May I leave you un-puzzled, MM. 

Finally, I am also glad to thank the students that I have had throughout the years as they have taught me so much. 

\begin{quote}
\textit{
And every science, when we understand it not as an instrument of power and domination but as an adventure in knowledge pursued by our species across the ages, is nothing but this harmony, more or less vast, more or less rich from one epoch to another, which unfurls over the course of generations and centuries, by the delicate counterpoint of all the themes appearing in turn, as if summoned from the void.}
\end{quote}
\hspace{250pt} - Alexandre Grothendieck
\\\\
Thank you all, as I start to play my small part.

\end{acknowledgements}

\begin{acknowledgements}
Research is humbling work, and I mean this in the proudest way. I am deeply grateful for the time I have been given to pursue it.  In no way would this have been possible without the dedication, encouragement, and support of many.

I begin with thanks to Michel Devoret and Rob Schoelkopf at Yale, and Andreas Wallraff at ETH Zurich, all of whom, during my undergraduate years, made quantum computation a reality for me.  Prof. Devoret has my particular thanks, as he witnessed my (at times bouncing) trajectory over the field since I first wrestled with the abelian hidden subgroup algorithm in his office in the summer of 2008. This is where my fascination with the structure of quantum algorithms began.

My mature study of the subject is with the tools and language taught to me in large part, by Samson Abramsky, Bob Coecke, Chris Heunen, Aleks Kissinger, and Jamie Vicary. The support and self-directed agency that my supervisors Bob and Jamie encouraged in my studies were crucial. Further, this learning - for me an initially broad leap of mathematical background - would not have been possible without my other colleagues in the Quantum Group at Oxford, especially Miriam Backens, Brendan Fong, Stefano Gogioso, and Vladimir Zamdzhiev. I would also like to thank the students that I have had throughout the years as they have taught me much. 

\textit{And every science, when we understand it not as an instrument of power and domination but as an adventure in knowledge pursued by our species across the ages, is nothing but this harmony, more or less vast, more or less rich from one epoch to another, which unfurls over the course of generations and centuries, by the delicate counterpoint of all the themes appearing in turn, as if summoned from the void.}

\hspace{230pt} - Alexandre Geothendieck

Thank you all, as I start to play my small part.

\end{acknowledgements}

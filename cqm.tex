\chapter{Categorical Quantum Mechanics}
\label{chap:cqm}

\todo{\chapabstract{}}

In this chapter we further develop the notion of abstract process theories in $\dagger$-SMC's. The chapter is called Categorical Quantum Mechanics as many of the properties developed here, while still general, are extensions of ideas from quantum theory, e.g. complementarity.  Abstract process theories that posses these properties can be considered quantum-like.

\section{Phases}
Phases were introduced in \cite{coecke2011interacting}.

\begin{defn}
\label{def:phases}
A \emph{phase state} for a Frobenius algebra $\whitecomonoid{A}$ is a state $\ket{\alpha}$ such that:
\begin{equation}
\label{eqn:zphasestate}
\begin{pic}[xscale=\tikzxscale, yscale=\tikzyscale]
\node [style=state] (0) at (0, -1) {$a$};
\node [style=whitedot] (1) at (0, -0) {};
\node [style=state, hflip] (2) at (-1, 1) {$a$};
\node (3) at (1, 1) {};
\node (4) at (1, 2) {};                

\draw (0) to (1);
\draw (3.center) to (4);
\draw [bend left=45, looseness=1.00] (1) to (2);
\draw [bend right=45, looseness=1.00] (1) to (3.center);
\end{pic}
\;=\;
\begin{pic}[xscale=\tikzxscale, yscale=\tikzyscale]
\draw (0,0) to (0,2);
\node [whitedot] at (0,0) {};
\end{pic}
\;=\;
\begin{pic}[xscale=-1*\tikzxscale, yscale=\tikzyscale]
\node [style=state] (0) at (0, -1) {$a$};
\node [style=whitedot] (1) at (0, -0) {};
\node [style=state, hflip] (2) at (-1, 1) {$a$};
\node (3) at (1, 1) {};
\node (4) at (1, 2) {};                

\draw (0) to (1);
\draw (3.center) to (4);
\draw [bend left=45, looseness=1.00] (1) to (2);
\draw [bend right=45, looseness=1.00] (1) to (3.center);
\end{pic}

\end{equation}
Each phase state corresponds to a \emph{phase} that is a map in the following form:
\begin{equation}
\label{eqn:zphase}
\begin{tikzpicture}[xscale=1.4*\tikzxscale, yscale=1.4*\tikzyscale]
\node [whitedot] (0) at (-1, -0) {$\alpha$};
\node (1) at (-1, -1) {};
\node (2) at (-1, 1) {};
\node (3) at (0, -0) {$:=$};
\node [whitedot] (4) at (1, -0) {};
\node (5) at (1, -1) {};
\node (6) at (1, 1) {};
\node [state] (7) at (2, -0.5) {$\alpha$};

\begin{pgfonlayer}{background}
\draw [bend right=45, looseness=0.75] (7) to (4);
\draw (5.center) to (4);
\draw (4) to (6.center);
\draw (1.center) to (0);
\draw (0) to (2.center);
\end{pgfonlayer}
\end{tikzpicture}

\end{equation}
\end{defn}

\begin{proposition}[Phase groups]
Given a dagger Frobenius algebra $\whitemonoid$ in a monoidal dagger category, its phases form a group under the following multiplication:
\begin{equation}
\begin{pic}[xscale=\tikzxscale, yscale=\tikzyscale]
\draw (0,0) to (0,2);
\node [state] at (0,0) {$ab$};
\end{pic}
\;=\;
\begin{pic}[xscale=1.2*\tikzxscale, yscale=1.1*\tikzyscale]
\node [style=whitedot] (0) at (0, 0) {};
\node [style=state] (1) at (-1, -1) {$a$};
\node [style=state] (2) at (1, -1) {$a$};
\node (3) at (0, 1) {};
\draw [bend right=45, looseness=1.00] (0) to (1);
\draw [bend left=45, looseness=1.00] (0) to (2);
\draw (0) to (3.center);
\end{pic}
\end{equation}
with unit $\tinyunit[whitedot]$. When the Frobenius algebra is part of a classical structure in a $\dager$-SMC, then its phases form an abelian group.
\end{proposition}
\begin{proof}
\todo{Include proof or cite CQM notes?}
\end{proof}

\section{Complementarity}
\begin{defn}[Complementarity]
\label{def:complementarity}
In a symmetric monoidal dagger-category, two special symmetric dagger-Frobenius comonoids \whitecomonoid{A} and \graycomonoid{A} are \emph{complementary} when the following equation holds:
\begin{equation}
\label{eq:complementarity}
\ud(A) \,\,
\begin{tikzpicture}[xscale=1.4*\tikzxscale, yscale=1.4*\tikzyscale]
\draw (-0.5,0.25) to (-0.5,1) node [graydot] {} to [out=left, in=right] (-1,2) node [graydot] {} to [out=left, in=right] (-1.5,1.5) node [whitedot] {} to [out=left, in=down] (-2,2) to [out=up, in=left] (-0.75,3) node (a) [whitedot] {} to [out=right, in=right] (-0.5,1);
\draw (a.center) to +(0,0.75);
\end{tikzpicture}
\quad=\quad\,\,\,
\begin{tikzpicture}[xscale=1.4*\tikzxscale, yscale=1.4*\tikzyscale]
\draw (0,0.25) to (0,1) node [graydot] {};
\draw (0,3) node [whitedot] {} to (0,3.75);
\end{tikzpicture}
\end{equation}
\end{defn}
\noindent
Note that this is not a symmetric condition between the gray and white structures. However, thanks to the symmetric property of the dagger-Frobenius algebras, it is equivalent to the following alternative condition:
\begin{equation}
\ud(A) \,\,
\begin{tikzpicture}[xscale=1.4*\tikzxscale, yscale=-1.4*\tikzyscale]
\draw (-0.5,0.25) to (-0.5,1) node [whitedot] {} to [out=left, in=right] (-1,2) node [whitedot] {} to [out=left, in=right] (-1.5,1.5) node [graydot] {} to [out=left, in=down] (-2,2) to [out=up, in=left] (-0.75,3) node (a) [graydot] {} to [out=right, in=right] (-0.5,1);
\draw (a.center) to +(0,0.75);
\end{tikzpicture}
\quad=\quad\,\,\,
\begin{tikzpicture}[xscale=1.4*\tikzxscale, yscale=1.4*\tikzyscale]
\draw (0,0.25) to (0,1) node [graydot] {};
\draw (0,3) node [whitedot] {} to (0,3.75);
\end{tikzpicture}
\end{equation}
The daggers of these equations give rise to two further equivalent conditions.

\begin{defn}\label{def_StrongComplementarity}
A pair of classical structures \whitecomonoid{A} and \blackcomonoid{A} is \emph{strongly complementary} if it satisfies the following \emph{bialgebra equation} (\ref{eqn:bialgebraEqns}) and \emph{coherence equations} (\ref{eqn:unitCopyEqns}):
\begin{equation}
\label{eqn:bialgebraEqns}
\begin{tikzpicture}[xscale=\tikzxscale, yscale=\tikzyscale]

\node (center) {};

\node (algebraTop) [whitedot]   
  [above of = center, yshift = -5mm]{};
\node (Hout) [above of = algebraTop, xshift = -5mm] {};
\node (Tout) [above of = algebraTop, xshift = +5mm] {};

\node (algebraBot) [blackdot]  
  [below of = center, yshift = +5mm]{};
\node (Hin) [below of = algebraBot, xshift = -5mm] {};
\node (Tin) [below of = algebraBot, xshift = +5mm] {};

\begin{pgfonlayer}{background}
\draw[-,out=90,in=270] (algebraBot) to (algebraTop);
\draw[-,out=135,in=270] (algebraTop) to (Hout);
\draw[-,out=45,in=270] (algebraTop) to (Tout);
\draw[-,out=90,in=225] (Hin) to (algebraBot);
\draw[-,out=90,in=315] (Tin) to (algebraBot);
\end{pgfonlayer}

\node (equals) [right of = center, xshift = 0mm]{$=$};

\node (center) [right of = equals, xshift = 0mm] {};

\node (algebraTop) [blackdot]   
  [above of = center, yshift = -5mm]{};
\node (Hout) [above of = algebraTop, xshift = 0mm] {};
\node (timemult) [blackdot] 
  [right of = algebraTop, xshift = 0mm] {}; 
\node (Tout) [above of = timemult, xshift = 0mm] {};

\node (algebraBot) [whitedot]  
  [below of = center, yshift = +5mm]{};
\node (Hin) [below of = algebraBot, xshift = 0mm] {};
\node (timediag) [whitedot] 
  [right of = algebraBot, xshift = 0mm] {}; 
\node (Tin) [below of = timediag, xshift = 0mm] {};

\begin{pgfonlayer}{background}
\draw[-,out=135,in=225] (algebraBot) to (algebraTop);
\draw[-,out=90,in=270] (algebraTop) to (Hout);
\draw[-,out=90,in=270] (Hin) to (algebraBot);
\draw[-,out=90,in=270] (Tin) to (timediag);
\draw[-,out=90,in=270] (timemult) to (Tout);
\draw[-,out=45,in=315] (timediag) to (timemult);
\draw[-,out=135,in=315] (timediag) to (algebraTop);
\draw[-,out=45,in=225] (algebraBot) to (timemult);
\end{pgfonlayer}

\end{tikzpicture}

\end{equation}
\begin{equation}
\label{eqn:unitCopyEqns}
\hbox{\begin{tikzpicture}[xscale=\tikzxscale, yscale=\tikzyscale]

\node (center) {};

\node (algebraTop) [whitedot]   
        [above of = center, yshift = -4mm]{};
\node (Hout) [above of = algebraTop, xshift = -3mm,yshift = 0mm] {};
\node (Tout) [above of = algebraTop, xshift = +3mm,yshift = 0mm] {};

\node (algebraBot) [blackdot]  
        [below of = center, yshift = +8mm]{};

\begin{pgfonlayer}{background}
\draw[-,out=90,in=270] (algebraBot) to (algebraTop);
\draw[-,out=135,in=270] (algebraTop) to (Hout);
\draw[-,out=45,in=270] (algebraTop) to (Tout);
\end{pgfonlayer}

\node (equals) [right of = center, xshift = -1mm,yshift = 5mm]{$=$};

\node (center) [right of = equals, xshift = -3mm,yshift = -5mm] {};

\node (algebraTop) [blackdot]   
        [below of = center, yshift = +10mm]{};
\node (Hout) [above of = algebraTop, yshift = +2mm] {};
\node (timemult) [blackdot] 
        [right of = algebraTop, xshift = -4mm] {}; 
\node (Tout) [above of = timemult, yshift = +2mm] {};

\begin{pgfonlayer}{background}
\draw[-,out=90,in=270] (algebraTop) to (Hout);
\draw[-,out=90,in=270] (timemult) to (Tout);
\end{pgfonlayer}


\node (spacer) [right of = center, xshift = 3mm,yshift = 5mm]{};

\node (center) [right of = spacer, yshift = -5mm] {};


\node (algebraTop) [blackdot]   
        [below of = center, yshift = +14mm]{};
\node (Hout) [below of = algebraTop, xshift = -3mm] {};
\node (Tout) [below of = algebraTop, xshift = +3mm] {};

\node (algebraBot) [whitedot]  
        [above of = center, yshift = +2mm]{};

\begin{pgfonlayer}{background}
\draw[-,out=270,in=90] (algebraBot) to (algebraTop);
\draw[-,out=225,in=90] (algebraTop) to (Hout);
\draw[-,out=315,in=90] (algebraTop) to (Tout);
\end{pgfonlayer}

\node (equals) [right of = center, xshift = -1mm,yshift = 5mm]{$=$};

\node (center) [right of = equals, xshift = -3mm, yshift = -7mm] {};

\node (algebraTop)  
        [below of = center, yshift = +8mm]{};
\node (Hout) [whitedot]  
        [above of = algebraTop, yshift = 2mm] {};
\node (timemult) 
        [right of = algebraTop, xshift = -4mm] {}; 
\node (Tout) [whitedot] 
        [above of = timemult, yshift = 2mm] {};

\begin{pgfonlayer}{background}
\draw[-,out=90,in=270] (algebraTop) to (Hout);
\draw[-,out=90,in=270] (timemult) to (Tout);
\end{pgfonlayer}

\end{tikzpicture}
}
\end{equation}
\end{defn}

\section{Duality}

\begin{defn}
In a monoidal category 
\end{defn}

\begin{defn}
\todo{Compact categories?} (closed follows)
\end{defn}

\begin{defn}
\todo{dagger compact categories?} (closed follows)
\end{defn}

There is a correspondence between morphisms and states given by the following.

\begin{defn}
Given a morphism $f:A\to B$ in a monoidal category and a Frobenius algebra $\blackmonoid{A}$, the \emph{name} of $f$ is given by:
\begin{equation}
\begin{pic}[xscale=\tikzxscale, yscale=\tikzyscale]
\node at (0,0) {$\name{f}$};
\node (L) [morphism, xscale=2.5] at (0,0) {};
\draw ([xshift=-1cm] L.north) to ([xshift=-1cm, yshift=1.5cm] L) node [above] {$\mathbb{C}^n$};
\draw ([xshift=1cm] L.north) to ([xshift=1cm, yshift=1.5cm] L) node [above] {$\mathbb{C} ^n$};
\end{pic}
\quad:=\quad\hspace{-5pt}
\begin{pic}[xscale=\tikzxscale, yscale=-1*\tikzyscale]
\draw (0,-1.5) node [above] {$\mathbb{C} ^n$} to (0,0) to [out=up, in=\swangle] (0.7,1);
\draw (1.4,0) to [out=up, in=\seangle] (0.7,1);
\draw (0.7,1) to (0.7,1.75) node [blackdot] {};
\node [blackdot] at (0.7,1) {};
\node (L) [morphism, anchor=south] at (1.4,0) {$f$};
\draw (L.north) to (1.4,-1.5) node [above] {$\mathbb{C} ^n$};
\end{pic}
\end{equation}
\end{defn}


\section{Measurements and Representation Theory}

\section{Quantum protocols}

\section{The Fourier Transform}

\section{Unitary oracles}


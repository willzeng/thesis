\chapter{Categorical Quantum Mechanics}
\label{chap:cqm}

\todo{\chapabstract{}}

\begin{defn}
\todo{Phases}
\end{defn}

\begin{defn}
\todo{Phase groups}
\end{defn}

\section{Complementarity}
\begin{defn}[Complementarity]
\label{def:complementarity}
In a symmetric monoidal dagger-category, two special symmetric dagger-Frobenius comonoids \whitecomonoid{A} and \graycomonoid{A} are \emph{complementary} when the following equation holds:
\begin{equation}
\label{eq:complementarity}
\ud(A) \,\,
\begin{tikzpicture}[xscale=1.4*\tikzxscale, yscale=1.4*\tikzyscale]
\draw (-0.5,0.25) to (-0.5,1) node [graydot] {} to [out=left, in=right] (-1,2) node [graydot] {} to [out=left, in=right] (-1.5,1.5) node [whitedot] {} to [out=left, in=down] (-2,2) to [out=up, in=left] (-0.75,3) node (a) [whitedot] {} to [out=right, in=right] (-0.5,1);
\draw (a.center) to +(0,0.75);
\end{tikzpicture}
\quad=\quad\,\,\,
\begin{tikzpicture}[xscale=1.4*\tikzxscale, yscale=1.4*\tikzyscale]
\draw (0,0.25) to (0,1) node [graydot] {};
\draw (0,3) node [whitedot] {} to (0,3.75);
\end{tikzpicture}
\end{equation}
\end{defn}
\noindent
Note that this is not a symmetric condition between the gray and white structures. However, thanks to the symmetric property of the dagger-Frobenius algebras, it is equivalent to the following alternative condition:
\begin{equation}
\ud(A) \,\,
\begin{tikzpicture}[xscale=1.4*\tikzxscale, yscale=-1.4*\tikzyscale]
\draw (-0.5,0.25) to (-0.5,1) node [whitedot] {} to [out=left, in=right] (-1,2) node [whitedot] {} to [out=left, in=right] (-1.5,1.5) node [graydot] {} to [out=left, in=down] (-2,2) to [out=up, in=left] (-0.75,3) node (a) [graydot] {} to [out=right, in=right] (-0.5,1);
\draw (a.center) to +(0,0.75);
\end{tikzpicture}
\quad=\quad\,\,\,
\begin{tikzpicture}[xscale=1.4*\tikzxscale, yscale=1.4*\tikzyscale]
\draw (0,0.25) to (0,1) node [graydot] {};
\draw (0,3) node [whitedot] {} to (0,3.75);
\end{tikzpicture}
\end{equation}
The daggers of these equations give rise to two further equivalent conditions.

\begin{defn}
\todo{Strong Complementarity}
\end{defn}

\section{Measurements and Representation Theory}

\section{Quantum protocols}

\section{Strong Complementarity and the Fourier Transform}

\section{Unitary oracles}

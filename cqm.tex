\chapter{Structures in Generalized Compositional Theories}
\label{chap:cqm}

\chapabstract{In this chapter we introduce the remaining background on categorical structures for GCTs. From this background we define an operational generalized compositional theory (oGCT) that comes equipped with an operational interpretation through a generalized Born rule. Many of the properties developed here, while still general, are extensions of ideas from quantum theory, e.g. complementarity.  oGCTs that posses these properties can be considered quantum-like, and have diagrammatic representations that can be used as more powerful extensions of quantum circuits.}

\section{The Dagger}
In this section concepts are introduced that expand categorical diagrams beyond quantum circuits. We see that abstract linear algebra can be introduced by adding a so-called \emph{dagger functor}.  In the case of \cat{FHilb} this corresponds to the familiar notion of adjoint (complex-transpose).  The addition of the dagger also allows us to take an operational perspective on symmetric monoidal categories (not just \cat{FHilb}). 

\begin{defn}
\label{defn:dagger}
A \emph{dagger functor} on a category \cat{C} is an involutive contravariant functor $\dagger:\cat{C}\to\cat{C}$ that is the identity on objects. A \emph{dagger category} is a category equipped with a dagger functor.
\end{defn}
Spelling out this definition, the dagger functor has the following properties for $f,g:\arrs{C}$ with suitable types and $A\in\objs{C}$:
\begin{equation}
\left(f^{\dagger}\right)^{\dagger} = f 
\end{equation}
\begin{equation}
(g\circ f)^{\dagger} = f^{\dagger}\circ g^{\dagger}
\end{equation}
\begin{equation}
\idm{A}^{\dagger} = \idm{A}
\end{equation}
Thus for any $f:A\to B$ in a dagger category, its dagger or \emph{adjoint} $f^{\dagger}:B\to A$ also exists in that category. We use the shorthand $\dagger$-SMC for a dagger symmetric monoidal category.

The quantum setting of \cat{FHilb} is a dagger category whose canonical dagger is the adjoint, and generalizing it in the manner of Definition \ref{defn:dagger} allows us to generalize many familiar terms, following \cite{abramsky2004categorical}:
\begin{defn}
A morphism $f:A\to B$ in a dagger category is:
\begin{itemize}
\item \emph{self-adjoint} when $f^{\dagger} = f$;
\item a \emph{projector} when self-adjoint and $f\circ f = f$
\item an \emph{isometry} when $f \circ \dag{f} = \idm{A}$;
\item \emph{unitary} when both $f \circ \dag{f} = \idm{A}$ and $f\circ\dag{f} = \idm{B}$;
\item \emph{positive} when $f = \dag{g}\circ g$ for some morphism $g:H\to K$.
\end{itemize}
\end{defn}

\noindent These concepts both have meaning as mathematical objects in linear algebra and as information theoretic concepts in a process theory.  For example, a projector is some process where multiple sequential applications have the same effect as a single application in the broadest possible sense.

Further, the dagger can be intuitively extended to an operation on diagrams: it flips the picture upside-down around the horizontal axes.  As a visual aid, morphisms can now be drawn with broken symmetry as follows:
\begin{equation}
\label{eq:daggerPics}
\begin{aligned}
\begin{tikzpicture}[xscale=\tikzxscale, yscale=\tikzyscale]

\node at (0.5, -2) {$A$};
\node (0) at (0, -2) {};
\node (1) [brnode] at (0, 0) {$f$};
\node (2) at (0, 2) {};
\node at (0.5, 2) {$B$};

\draw (0.center) to (1.south);
\draw (1.north) to (2.center);

\end{tikzpicture}
\end{aligned}
\qquad \mapsto \qquad 
\begin{aligned}
\begin{tikzpicture}[xscale=\tikzxscale, yscale=\tikzyscale]

\node at (0.5, -2) {$B$};
\node (0) at (0, -2) {};
\node (1) [trnode] at (0, 0) {$f$};
\node (2) at (0, 2) {};
\node at (0.5, 2) {$A$};

\draw (0.center) to (1.south);
\draw (1.north) to (2.center);

\end{tikzpicture}
\end{aligned}
\::=\:
\begin{aligned}
\begin{tikzpicture}[xscale=\tikzxscale, yscale=\tikzyscale]

\node at (0.5, -2) {$B$};
\node (0) at (0, -2) {};
\node (1) [morphism] at (0, 0) {$\dag{f}$};
\node (2) at (0, 2) {};
\node at (0.5, 2) {$A$};

\draw (0.center) to (1.south);
\draw (1.north) to (2.center);

\end{tikzpicture}
\end{aligned}

\end{equation}

Thus the unitarity condition becomes:
\begin{equation}
\label{eq:unitarityPics}
\begin{aligned}
\begin{tikzpicture}[xscale=\tikzxscale, yscale=\tikzyscale]

\node (0) at (0, -1.5) {};
\node (1) [brnode] at (0, 0) {$f$};
\node (2) [trnode] at (0, 2) {$f$};
\node (3) at (0, 3.5) {};

\draw (0.center) to (1.south);
\draw (1.north) to (2.south);
\draw (2.north) to (3.center);

\end{tikzpicture}
\end{aligned}
\; = \;
\begin{aligned}
\begin{tikzpicture}[xscale=\tikzxscale, yscale=\tikzyscale]

\node (0) at (0, -1.5) {};
\node (3) at (0, 3.5) {};

\draw (0.center) to (3.center);

\end{tikzpicture}
\end{aligned}
\qquad\qquad\qquad
\begin{aligned}
\begin{tikzpicture}[xscale=\tikzxscale, yscale=\tikzyscale]

\node (0) at (0, -1.5) {};
\node (1) [trnode] at (0, 0) {$f$};
\node (2) [brnode] at (0, 2) {$f$};
\node (3) at (0, 3.5) {};

\draw (0.center) to (1.south);
\draw (1.north) to (2.south);
\draw (2.north) to (3.center);

\end{tikzpicture}
\end{aligned}
\; = \;
\begin{aligned}
\begin{tikzpicture}[xscale=\tikzxscale, yscale=\tikzyscale]

\node (0) at (0, -1.5) {};
\node (3) at (0, 3.5) {};

\draw (0.center) to (3.center);

\end{tikzpicture}
\end{aligned}

\end{equation}

The next two definitions, that of \emph{scalars} and \emph{effects}, further our understanding of $\dagger$-SMC's with a generalized version of the Born rule that provides a basic notion of measurement.

\begin{defn}
\label{defn:scalar}
A \emph{scalar} in a SMC is a morphism $a:I\to I$. As these are maps from the empty diagram to the empty diagram they are unsurprisingly represented as:
\begin{equation}
\begin{tikzpicture}[xscale={\tikzxscale}, yscale={\tikzyscale}]
\node [whitedot] (0) at (-1, -0) {$a$};
\end{tikzpicture}
\end{equation}
\end{defn}

This categorical view of scalars was made explicit in \cite{abramsky2004categorical}. To gain perspective on why these properly represent scalars, we consider two facts.  Firstly, these scalars form a commutative monoid under categorical composition \cite[Prop 6.1]{kelly1980coherence}, as is graphically expressed by:
\begin{equation}
\begin{aligned}
\begin{tikzpicture}[xscale={\tikzxscale}, yscale={\tikzyscale}]
\node [whitedot] (0) at (-1, -0) {$a$};
\node [whitedot] (0) at (-1, -2) {$b$};
\end{tikzpicture}
\end{aligned}
\;=\;
\begin{aligned}
\begin{tikzpicture}[xscale={\tikzxscale}, yscale={\tikzyscale}]
\node [whitedot] (0) at (-1, -0) {$b$};
\node [whitedot] (0) at (-1, -2) {$a$};
\end{tikzpicture}
\end{aligned}
\end{equation}
\noindent Secondly, in $\cat{FHilb}$ they correspond exactly to the complex numbers, as linear maps~$\mathbb{C}\to\mathbb{C}$.

\begin{defn}
\label{def:effect}
In a $\dagger$-SMC, \emph{effects} on an object $A$ are morphisms $\bra{\psi}:A\to I$.
\end{defn}
\noindent There can certainly be morphisms of this type in any monoidal category, but the dagger gives them an operational interpretation: just as some preparation $p:I\to A$ prepares system $A$ in that state, the effect $\dag{p}:A\to I$ eliminates the system $A$ by the process $\dag{p}$.  This duality presents generalized \emph{inner products} as compositions of states and effects that are reminiscent of the usual Dirac notation~\cite{abramsky2004categorical}:
\begin{equation}
\left(
\begin{aligned}
\begin{tikzpicture}[xscale=\tikzxscale, yscale=\tikzyscale]
\node (1) [state] at (0, -0.75) {$\psi$};
\node (2) at (0, 0.75) {};
\draw (1.center) to (2.center);
\end{tikzpicture}
\end{aligned}\right)^{\dagger}
=
\begin{aligned}
\begin{tikzpicture}[xscale=\tikzxscale, yscale=\tikzyscale]
\node (1) at (0, -0.75) {};
\node (2) [state, hflip] at (0, 0.75) {$\psi$};
\draw (1.center) to (2.center);
\end{tikzpicture}
\end{aligned}
\qquad\qquad
|\phi\rangle\circ\langle\psi| = \langle\phi|\psi\rangle =
\begin{aligned}
\begin{tikzpicture}[xscale=\tikzxscale, yscale=\tikzyscale]
\node (1) [state] at (0, -0.75) {$\psi$};
\node (2) [state, hflip] at (0, 0.75) {$\phi$};
\draw (1.center) to (2.center);
\end{tikzpicture}
\end{aligned}
\end{equation}

\section{The generalized Born rule}
\label{sec:bornrule}
Measurement in these dagger generalized compositional theories comes from a statement connecting probabilities and inner products. For a state $X:I\to A$ and an effect $Y:A\to I$ in a $\dagger$-SMC, the amplitude of outcome $X$ given preparation $Y$ is:
\begin{equation}
a = X\circ Y:I\to I.
\end{equation}

%Diagrammatically this is:
%\begin{equation}
%\mbox{Prob}(X|Y) \;=\; 
%\begin{aligned}
%\begin{tikzpicture}[xscale=\tikzxscale, yscale=\tikzyscale]
%\node (1) [state] at (0, -0.5) {$Y$};
%\node (2) [state, hflip] at (0, 0.5) {$X$};
%\draw (1.center) to (2.center);
%\node (3) [state] at (0, 3.5) {$Y^{*}$};
%\node (4) [state, hflip] at (0, 4.5) {$X^{*}$};
%\draw (3.center) to (4.center);
%\end{tikzpicture}
%\end{aligned}
%\;=\;
%\begin{aligned}
%\begin{tikzpicture}[xscale=\tikzxscale, yscale=\tikzyscale]
%\node (1) [state] at (0, -0.5) {$Y$};
%\node (2) [state, hflip] at (0, 0.5) {$X$};
%\draw (1.center) to (2.center);
%\node (3) [state] at (3, -0.5) {$Y^{*}$};
%\node (4) [state, hflip] at (3, 0.5) {$X^{*}$};
%\draw (3.center) to (4.center);
%\end{tikzpicture}
%\end{aligned}
%\end{equation}

\noindent We could then define the operational probability Prob$(X|Y)=|a|^2$. While this clearly makes sense in \cat{FHilb}, where amplitudes are complex numbers, one might ask under what general conditions the scalars $I\to I$ will square to our usual notion of probability. An answer is found in the following theorem:
\begin{theorem}{\cite[Thm 4.2]{vicary2011completeness}}
In a monoidal dagger-category with simple tensor unit, which has all finite dagger-limits and for which the self-adjoint scalars are Dedekind-complete, the scalars have an involution-preserving embedding into the complex numbers.
\end{theorem}
\noindent For any category satisfying these conditions, this embedding can be used, along with appropriate normalization, to extract the real-valued probabilities. Still, even if the scalars do not embed in the complex numbers one can generally handle the needed structure of complex conjugation to obtain a generalized Born rule. There is a way to define conjugation in any $\dagger$-SMC which has duals, as we describe in this section.\footnote{These are sometimes called autonomous categories in the literature \cite{joyal1993braided,selinger2011survey}.}

Duals in a generalized compositional theory help us define both entanglement and conjugation.

\begin{defn}
\label{def:dual}
In a $\dagger$-SMC, a system $A$ has a\footnote{In general, there can be separate left and right duals for an object in any category, where the definition given here corresponds to a left dual. In a \dsmc(in fact in any braided monoidal category with left duals), these are necessarily equal, so we need only speak of one dual~\cite[Prop.~7.2]{joyal1993braided}. } \emph{dual} system $A^*$ if there exist processes 
\begin{align}
e_A:I\to A^*\otimes A \quad \mbox{and}\quad d_A:A\otimes A^*\to I
\end{align}
such that the following equations hold:
\begin{align}
(d_A\otimes \idm{A})\circ(\idm{A}\otimes e_A) = \idm{A} \qquad (\idm{A^*}\otimes d_A)\circ(e_A\otimes \idm{A^*})=\idm{A^*}
\end{align}
\end{defn}

When duals are introduced, we can denote them by placing arrows on the objects:
\begin{equation}
\idm{A} = \;
\begin{pic}[xscale={\tikzxscale}, yscale=2*\tikzyscale]
\node (0) at (0,0) {};
\node (1) at (0,2) {};
\draw [arrow=.5] (0) to (1);
\end{pic}
\qquad \qquad
\idm{A^*} = \;
\begin{pic}[xscale={\tikzxscale}, yscale=2*\tikzyscale]
\node (0) at (1,0) {};
\node (1) at (1,2) {};
\draw [reverse arrow=.5] (0) to (1);
\end{pic}
\qquad \qquad
g:A^*\to B^* = \;
\begin{pic}[xscale={\tikzxscale}, yscale=2*\tikzyscale]
\node (0) at (0,0) {};
\node (1) at (0,2) {};
\node [morphism] (2) at (0,1) {$g$};
\draw [reverse arrow=.5] (0) to (2.south);
\draw [reverse arrow=.5] (2.north) to (1);
\end{pic}

\end{equation}

Thus the duality maps, commonly called ``cups" and ``caps", and their ``snake equations" have the following diagrammatic form:
\begin{align}
e_A = \,
\begin{pic}[xscale={\tikzxscale}, yscale={\tikzyscale}]
\node (0) at (-1, -0) {};
\node (1) at (1, -0) {};
\node (2) at (0, -1) {};
\draw [bend right=45, looseness=1.00] (0.center) to (2.center);
\draw [bend right=45, looseness=1.00, ->] (2.center) to (1.center);
\end{pic}
\qquad \qquad 
d_A = \,
\begin{pic}[xscale={\tikzxscale}, yscale={-\tikzyscale}]
\node (0) at (-1, -0) {};
\node (1) at (1, -0) {};
\node (2) at (0, -1) {};
\draw [bend right=45, looseness=1.00] (0.center) to (2.center);
\draw [bend right=45, looseness=1.00, ->] (2.center) to (1.center);
\end{pic}

\end{align}
\begin{equation}
\label{eq:snake}
\begin{pic}[xscale={1*\tikzxscale}, yscale={1*\tikzyscale}]
\node (0) at (-1, -0) {};
\node (1) at (1, -0) {};
\node (2) at (0, -1) {};
\node (3) at (2, 1) {};
\node (4) at (1, -0) {};
\node (5) at (3, -0) {};
\node (6) at (3, -1) {};
\node (7) at (-1, 1) {};
\draw [bend right=45, looseness=1.00] (0.center) to (2.center);
\draw [bend right=45, looseness=1.00, arrow=.99] (2.center) to (1.center);
\draw [bend left=45, looseness=1.00] (4.center) to (3.center);
\draw [bend left=45, looseness=1.00, arrow=.99] (3.center) to (5.center);
\draw (6.center) to (5.center);
\draw [reverse arrow] (0.center) to (7.center);
\end{pic}
\; = \; 
\begin{pic}[xscale={1*\tikzxscale}, yscale=1.3*\tikzyscale]
\node (0) at (0,0) {};
\node (1) at (0,2) {};
\draw [reverse arrow=.5] (0) to (1);
\end{pic}
\qquad \qquad
\begin{pic}[xscale={1*\tikzxscale}, yscale={-1*\tikzyscale}]
\node (0) at (-1, -0) {};
\node (1) at (1, -0) {};
\node (2) at (0, -1) {};
\node (3) at (2, 1) {};
\node (4) at (1, -0) {};
\node (5) at (3, -0) {};
\node (6) at (3, -1) {};
\node (7) at (-1, 1) {};
\draw [bend right=45, looseness=1.00] (0.center) to (2.center);
\draw [bend right=45, looseness=1.00, arrow=.99] (2.center) to (1.center);
\draw [bend left=45, looseness=1.00] (4.center) to (3.center);
\draw [bend left=45, looseness=1.00, arrow=.99] (3.center) to (5.center);
\draw (6.center) to (5.center);
\draw [reverse arrow] (0.center) to (7.center);
\end{pic}
\; = \; 
\begin{pic}[xscale={1*\tikzxscale}, yscale=-1.3*\tikzyscale]
\node (0) at (0,0) {};
\node (1) at (0,2) {};
\draw [reverse arrow=.5] (0) to (1);
\end{pic}

\end{equation} 
\noindent and are well behaved under the dagger functor, i.e.
\begin{equation}
\left(
\begin{pic}[xscale={\tikzxscale}, yscale={\tikzyscale}]
\node (0) at (-1, -0) {};
\node (1) at (1, -0) {};
\node (2) at (0, -1) {};
\draw [bend right=45, looseness=1.00] (0.center) to (2.center);
\draw [bend right=45, looseness=1.00, ->] (2.center) to (1.center);
\end{pic}
\right)^{\dagger}
\;=\;
\begin{pic}[xscale={\tikzxscale}, yscale={\tikzyscale}]
                \node  (0) at (-1, -0) {};
                \node  (1) at (1, -0) {};
                \node  (2) at (0, 1) {};
                \node  (3) at (0, -1) {};
                \node  (4) at (1, -2) {};
                \node  (5) at (0, -1) {};
                \node  (6) at (-1, -2) {};
                \draw [bend left=45, looseness=1.00] (0.center) to (2.center);
                \draw [reverse arrow=-.01, bend left=45, looseness=1.00] (2.center) to (1.center);
                \draw [arrow=0, bend right=15, looseness=1.00] (0.center) to (3.center);
                \draw [bend left=15, looseness=1.00] (1.center) to (3.center);
                \draw [bend left=15, looseness=1.00] (6.center) to (5.center);
                \draw [bend right=15, looseness=1.00] (4.center) to (5.center);
\end{pic}
\qquad \qquad
\left(
\begin{pic}[xscale={\tikzxscale}, yscale={-1*\tikzyscale}]
\node (0) at (-1, -0) {};
\node (1) at (1, -0) {};
\node (2) at (0, -1) {};
\draw [bend right=45, looseness=1.00] (0.center) to (2.center);
\draw [bend right=45, looseness=1.00, ->] (2.center) to (1.center);
\end{pic}
\right)^{\dagger}
\;=\;
\begin{pic}[xscale={\tikzxscale}, yscale={-1*\tikzyscale}]
                \node  (0) at (-1, -0) {};
                \node  (1) at (1, -0) {};
                \node  (2) at (0, 1) {};
                \node  (3) at (0, -1) {};
                \node  (4) at (1, -2) {};
                \node  (5) at (0, -1) {};
                \node  (6) at (-1, -2) {};
                \draw [bend left=45, looseness=1.00] (0.center) to (2.center);
                \draw [reverse arrow=-0.01, bend left=45, looseness=1.00] (2.center) to (1.center);
                \draw [arrow=0, bend right=15, looseness=1.00] (0.center) to (3.center);
                \draw [bend left=15, looseness=1.00] (1.center) to (3.center);
                \draw [bend left=15, looseness=1.00] (6.center) to (5.center);
                \draw [bend right=15, looseness=1.00] (4.center) to (5.center);
\end{pic}

\end{equation}

We can compose these caps and cups to define the dual of any process $f:A\to B$ as $f^*:B^*\to A^*$:
\begin{equation}
\label{eq:dualmorph}
\begin{pic}[xscale={\tikzxscale}, yscale={2*\tikzyscale}]
\node (0) at (0,0) {};
\node (1) at (0,2) {};
\node [morphism] (2) at (0,1) {$f^*$};
\draw [reverse arrow=.5] (0) to (2.south);
\draw [reverse arrow=.5] (2.north) to (1);
\end{pic}
\;=\;
\begin{pic}[xscale={\tikzxscale}, yscale=1.75*\tikzyscale]
                \node  (0) at (-1, -0) {};
                \node  (1) at (1, -0) {};
                \node  (2) at (0, -1) {};
                \node  (3) at (2, 1) {};
                \node  (4) at (1, -0) {};
                \node  (5) at (3, -0) {};
                \node  (6) at (3, -1) {};
                \node  (7) at (-1, 1) {};
                \node [style=morphism] (8) at (1, -0) {f};
                \draw [bend right=45, looseness=1.00] (0.center) to (2.center);
                \draw [bend right=45, looseness=1.00] (2.center) to (8.south);
                \draw [bend left=45, looseness=1.00] (8.north) to (3.center);
                \draw [bend left=45, looseness=1.00] (3.center) to (5.center);
                \draw [reverse arrow=.5] (6.center) to (5.center);
                \draw [reverse arrow=.5] (0.center) to (7.center);
\end{pic}

\end{equation}

\begin{example}
\label{ex:bellduals}
To get a better handle on what this structure means, we consider the example of \cat{FHilb}. Given some finite dimensional Hilbert space $A\in\objs{FHilb}$, its dual $A^*$ is the usual dual space, i.e. the space of linear functionals $A\to \mathbb{C}$. Note that in this case $A$ is isomorphic to $A^*$, and thus we can canonically map $\bra{i}\mapsto\ket{i}$. Because of this isomorphism, we will often omit the arrows on wires when working in \cat{FHilb}. Using this and given a basis $\{ \ket{i} \}$ for $A$, the cups and the caps are maps:
\begin{align}
e_A &:: 1\mapsto \sum_i\bra{i}\otimes\ket{i}\cong\sum_i\ket{i}\otimes\ket{i}
\\
d_A &:: \sum_i\ket{i}\otimes\bra{i}\cong\sum_i\bra{i}\otimes\bra{i} \mapsto 1
\end{align}
We then immediately recognize $e_A$ as entangled state preparation. In particular, when $A$ is a two-dimensional Hilbert space, $e_A$ is a Bell state preparation of $\ket{\psi_{00}}=\ket{00}+\ket{11}$ and  and $d_A$ is a post-selected Bell measurement for $\ket{\psi_{00}}$. For any process in \cat{FHilb} its dual gives the transpose, which is in general the \emph{upper-star} of $f$.
\end{example}

This provides a motivating example for the following, which acts as a generalized conjugation operation:

\begin{defn}
Given a GCT with duals \cat{C}, the \emph{lower-star} is an involutive map 
\begin{align*}
(-)_*:\arrs{C}&\to\arrs{C} \\
\left(f:A\to B\right) &\mapsto f_*:=(\dag{f})^*=\dag{(f^*)}.
\end{align*}
\end{defn}

The lower-star operation introduces abstract probabilities for the generalized Born rule:
\begin{defn}[Generalized Born Rule]
\label{def:bornrule}
For a state $X:I\to A$ and an effect $Y:A\to I$ in a $\dagger$-SMC, the probability of outcome $X$ given preparation $Y$ is:
\begin{equation}
\mbox{Prob}(X|Y) = (X\circ Y)^*\otimes X\circ Y:I\to I.
\end{equation}
\end{defn}
\noindent It is easy to see that this reduces to the usual Born rule in \cat{FHilb}.

The cup and cap maps also provide an abstract definition for the trace of a process:
\begin{equation}
\label{eq:trace}
\Tr \left(\begin{pic}
\node (0) at (0,0) {};
\node [morphism, wedge] (1) at (0,1) {$f$};
\node (2) at (0,2) {};
\draw (0) to (1.south);
\draw (1.north) to (2);
\end{pic} \right) = 
\begin{pic}
                \node [style=morphism, wedge] (0) at (0, 0) {$f$};
                \node (1) at (0, 0.75) {};
                \node (2) at (0, -0.75) {};
                \node (3) at (-1, 0.75) {};
                \node (4) at (-1, -0.75) {};
                \draw (0.north) to (1.center);
                \draw [bend right=90, looseness=1.50] (1.center) to (3.center);
                \draw (3.center) to (4.center);
                \draw [bend right=90, looseness=1.25] (4.center) to (2.center);
                \draw (2.center) to (0.south);
\end{pic}
,
\end{equation}
whose correspondence with the usual trace is easy to check~\cite{coecke2015generalised}.

\subsection{Operational Generalized Compositional Theories}

The addition of states and a generalized Born rule (with further details in this section), give our generalized compositional theories a notion of measurement and an operational character.

\begin{defn}(Operational Generalized Compositional Theories)
\label{thm:ogct}
A $\dagger$-SMC where all objects have duals is called a \emph{$\dagger$-compact category}. An \emph{operational generalized compositional theory} (oGCT) is a \dcc where objects are interpreted as systems and morphisms are interpreted as processes.
\end{defn}

Duals and their ``wire-straightening" equations give diagrams in an oGCT a lot of power to encode topological equivalences.  This is well expressed in the following graphical coherence theorem, whose mathematical details are shown in~\cite{selinger2011survey}.

\begin{theorem}[Fundamental Theorem of Diagrams~\cite{coecke2015generalised}]
\label{thm:fund}
Two diagrams in an oGCT are considered equal if one can be smoothly transformed to another by bending, stretching, or crossing wires, and by moving boxes around. Given any two oGCTs $\cat{C}$ and $\cat{D}$ and a map $F:\cat{C}\to\cat{D}$, for any two diagrams $d$ and $d'$ in $\cat{C}$, if $d=d'$ as diagrams then $F(d)=F(d')$ in $\cat{D}$.
\end{theorem}

To recap, we have so far introduced the structures of the dagger and duals, allowing us to compute measurement outcomes by manipulation of the diagrams alone.  This contrasts with the usual quantum circuit formalism, which provides a representation of a protocol, but whose implementation must ultimately be understood through the matrix mechanics that accompany it. The use of this difference - the  additional structure of operational generalized compositional theories versus generalized compositional theories - can be illustrated with quantum teleportation, which provides a motivating example and was introduced in this form by~\cite{abramsky2004categorical}.

\begin{example} 
We saw in Example \ref{ex:bellduals} that caps and cups represent preparation and post-selected measurement of one of the Bell states $\ket{\psi_{00}}$. The other Bell states can be prepared and/or measured by the application of a unitary to one of the entangled systems:
\begin{equation}
\begin{pic}[xscale={\tikzxscale}, yscale={\tikzyscale}]
\node (0) at (-1, -0) {};
\node (1) at (1, -0) {};
\node (2) at (0, -1) {};
\node (r) at (1,3) {};
\node (l) at (-1,3) {};
\node [morphism, wedge] (u) at (-1,1) {$U_i$};
\draw (1.center) to (r);
\draw (u.north) to (l);
\draw [bend right=45, looseness=1.00] (u.south) to (2.center);
\draw [bend right=45, looseness=1.00] (2.center) to (1.center);
\end{pic}
\qquad \qquad
\begin{pic}[xscale={\tikzxscale}, yscale={-1*\tikzyscale}]
\node (0) at (-1, -0) {};
\node (1) at (1, -0) {};
\node (2) at (0, -1) {};
\node (r) at (1,3) {};
\node (l) at (-1,3) {};
\node [morphism, wedge, hflip] (u) at (-1,1) {$U_i$};
\draw (1.center) to (r);
\draw (u.south) to (l);
\draw [bend right=45, looseness=1.00] (u.north) to (2.center);
\draw [bend right=45, looseness=1.00] (2.center) to (1.center);
\end{pic}
\end{equation}

A teleportation protocol can then be represented by the process where Alice and Bob share an entangled system that will be used to teleport from Alice to Bob:
\begin{equation}
\begin{pic}[xscale=\tikzxscale, yscale=\tikzyscale]
    \node (0) at (-1, -2) {};
    \node (1) at (3, -2) {};
    \node (2) at (1, -4) {};
    \node [style=morphism, wedge, hflip] (3) at (-3, -1) {$U_{i}$};
    \node (4) at (-2, 1) {};
    \node (5) at (-3, -0) {};
    \node (6) at (-1, -0) {};
    \node (7) at (-3, -4) {};
    \node [style=morphism, wedge] (8) at (3, 2) {$U_i$};
    \node (9) at (3, 3) {};
    \node [draw=black, dotted, minimum width=2cm, minimum height=4cm] at         (-2.5,-0.5) {};
    \node [draw=black, dotted, minimum width=2cm, minimum height=4cm] at         (3.5,-0.5) {};
    \node at (-3.5, 4) {Alice};
    \node at (2.5, 4) {Bob};
    \node at (1.5, -5.5) {Shared Bell state};    
    \draw [bend right=45, looseness=1.00] (0.center) to (2.center);
    \draw [in=-90, out=0, looseness=1.00] (2.center) to (1.center);
    \draw [bend left=45, looseness=1.00] (5.center) to (4.center);
    \draw [in=90, out=0, looseness=1.00] (4.center) to (6.center);
    \draw (7.center) to (3.south);
    \draw (3.north) to (5.center);
    \draw (6.center) to (0.center);
    \draw (8.south) to (1.center);
    \draw (8.north) to (9.center);
\end{pic}

\end{equation}
Here the cup map represents the shared Bell state.  Alice performs a Bell measurement on the source system and her half of the entangled system. Bob then performs a unitary that matches Alice's Bell measurement. We can verify the effect of the protocol using purely diagrammatic equivalences:
\begin{equation}
\label{eq:teleportderive}
\begin{pic}[xscale=0.9*\tikzxscale, yscale=\tikzyscale]
    \node (0) at (-1, -2) {};
    \node (1) at (1, -2) {};
    \node (2) at (0, -3) {};
    \node [style=morphism, wedge, hflip] (3) at (-3, -1) {$U_{i}$};
    \node (4) at (-2, 1) {};
    \node (5) at (-3, -0) {};
    \node (6) at (-1, -0) {};
    \node (7) at (-3, -4) {};
    \node [style=morphism, wedge] (8) at (1, 2) {$U_i$};
    \node (9) at (1, 3) {}; 
    \draw [bend right=45, looseness=1.00] (0.center) to (2.center);
    \draw [in=-90, out=0, looseness=1.00] (2.center) to (1.center);
    \draw [bend left=45, looseness=1.00] (5.center) to (4.center);
    \draw [in=90, out=0, looseness=1.00] (4.center) to (6.center);
    \draw (7.center) to (3.south);
    \draw (3.north) to (5.center);
    \draw (6.center) to (0.center);
    \draw (8.south) to (1.center);
    \draw (8.north) to (9.center);
\end{pic}
\stackrel{\mbox{\small Thm}~\ref{thm:fund}}{=}
\begin{pic}[xscale=0.9*\tikzxscale, yscale=\tikzyscale]
    \node (0) at (-1, -2) {};
    \node (1) at (1, -2) {};
    \node (2) at (0, -3) {};
    \node (3) at (-3, -1) {};
    \node (4) at (-2, 1) {};
    \node (5) at (-3, -0) {};
    \node (6) at (-1, -0) {};
    \node (7) at (-3, -4) {};
    \node [style=morphism, wedge] (8) at (1, 2) {$U_i$};
    \node (9) at (1, 3) {}; 
    \node [style=morphism, wedge, hflip] (u) at (1, 0) {$U_i$};    
    \draw [bend right=45, looseness=1.00] (0.center) to (2.center);
    \draw [in=-90, out=0, looseness=1.00] (2.center) to (1.center);
    \draw [bend left=45, looseness=1.00] (5.center) to (4.center);
    \draw [in=90, out=0, looseness=1.00] (4.center) to (6.center);
    \draw (7.center) to (3.center);
    \draw (3.center) to (5.center);
    \draw (6.center) to (0.center);
    \draw (u.south) to (1.center);
    \draw (u.north) to (8.south);
    \draw (8.north) to (9.center);
\end{pic}
\;\stackrel{\eqref{eq:unitarityPics}}{=}
\begin{pic}[xscale=0.9*\tikzxscale, yscale=\tikzyscale]
    \node (0) at (-1, -2) {};
    \node (1) at (1, -2) {};
    \node (2) at (0, -3) {};
    \node (3) at (-3, -1) {};
    \node (4) at (-2, 1) {};
    \node (5) at (-3, -0) {};
    \node (6) at (-1, -0) {};
    \node (7) at (-3, -4) {};
    \node (8) at (1, 2) {};
    \node (9) at (1, 3) {}; 
    \draw [bend right=45, looseness=1.00] (0.center) to (2.center);
    \draw [in=-90, out=0, looseness=1.00] (2.center) to (1.center);
    \draw [bend left=45, looseness=1.00] (5.center) to (4.center);
    \draw [in=90, out=0, looseness=1.00] (4.center) to (6.center);
    \draw (7.center) to (3.center);
    \draw (3.center) to (5.center);
    \draw (6.center) to (0.center);
    \draw (8.center) to (1.center);
    \draw (8.center) to (9.center);
\end{pic}
\;\stackrel{\eqref{eq:snake}}{=}
\begin{pic}[xscale=0.5*\tikzxscale, yscale=\tikzyscale]
    \node (0) at (-3, -4) {};
    \node (1) at (3, 3) {};
    \draw (0) to (1);
\end{pic}

\end{equation}
This presentation of the teleportation protocol is useful in several ways:
\begin{enumerate}
\item The high level structure of the protocol (the teleportation) is immediately manifest without accompanying calculation.
\item It is obvious how to design equivalent teleportation protocols with multiple parties and Bell measurements in more elaborate arrangements, e.g.
\begin{equation}
\begin{pic}[xscale=2*\tikzxscale, yscale=1.5*\tikzyscale]
    \node  (0) at (-1, -1) {};
    \node  (1) at (1, -0.6) {};
    \node  (2) at (0, -2) {};
    \node  (3) at (1.25, -2.2) {};
    \node  (4) at (2.5, -1) {};
    \node [style=morphism, wedge, hflip] (5) at (0, -1) {$U_i$};
    \node  (6) at (0.5, 0) {};
    \node  (7) at (3.5, -0) {};
    \node  (8) at (0.5, -3) {};
    \node  (9) at (2.5, -0) {};
    \node  (10) at (0, 2.5) {};
    \node  (11) at (2.5, -0) {};
    \node [style=morphism, wedge] (12) at (-2.5, -0) {$U_i$};
    \node  (13) at (-2.25, 2.5) {};
    \node  (14) at (-1, 1) {};
    \node [style=morphism, wedge, hflip] (15) at (-3.5, 1) {$U_i$};
    \node  (16) at (-3.5, -3) {};
    \node  (19) at (3.5, 2.5) {};
    \node [style=morphism, wedge, hflip, vflip] (20) at (-1, -0) {$U_i$};

    \draw [bend right=45, looseness=1.00] (0.center) to (2.center);
    \draw [bend right=45, looseness=1.00] (2.center) to (1.center);
    \draw [bend right=30, looseness=1.00] (5.south) to (3.center);
    \draw [bend right=45, looseness=1.00] (3.center) to (4.center);
    \draw [bend left=45, looseness=1.00] (5.north) to (6.center);
    \draw [bend left=40, looseness=1.00] (6.center) to (1.center);
    \draw [bend right=45, looseness=1.00] (8.center) to (7.center);
    \draw [bend left=45, looseness=1.00] (12.north) to (10.center);
    \draw [bend left=45, looseness=1.00] (10.center) to (11.center);
    \draw (4.center) to (9.center);
    \draw [bend left=45, looseness=1.00] (15.north) to (13.center);
    \draw [bend left=42, looseness=1.0] (13.center) to (14.center);
    \draw (15.south) to (16.center);
    \draw [bend right=45, looseness=1.00] (12.south) to (8.center);
    \draw (14.center) to (20.north);
    \draw (20.south) to (0.center);
    \draw (19.center) to (7.center);
\end{pic}
\quad=\quad
\begin{pic}[xscale=0.5*\tikzxscale, yscale=\tikzyscale]
    \node (0) at (1, -5) {};
    \node (1) at (6, 4) {};
    \draw (0) to (1);
\end{pic}
\begin{pic}

\end{pic}

\end{equation}
\item By specifying the teleportation protocol at the level of oGCTs, we can consider models of teleportation in theories other than quantum theory, i.e. categories other than $\cat{FHilb}$. As an example, the oGCT $\cat{FRel}$ where systems are sets and processes are relations also has entangled states established by duals, where the dagger functor is the relational converse. All the protocol specifications given in this section apply equally well in the $\cat{FRel}$ setting, just as they will in any other oGCT.
\end{enumerate}
\end{example}

Operational generalized compositional theories already give us some power above and beyond the usual quantum circuit formalism, and we add more structures to the toolbox in the next sections.  We have glossed over some of the details of how measurement should be represented in such theories, but the introduction of structures for observables in the next sections allows us to be more exact in Section~\ref{sec:measurements}. 

\section{Observables}
\label{sec:observables}

Within a process theory we can construct objects that have the phenomenology of classical information: copying, deleting, etc. These structures become the observables of oGCTs that allow us to extract classical information via measurements.

\subsection{Monoids and comonoids}

Monoids and comonoids on systems embody our notions of copying and comparing states. As they are a particular kinds of processes we draw them with distinct pictures. For $A\in\objs{C}$ a monoid $(A,*,\mathbbm{1})$ has states of $A$ as elements and maps:
\begin{equation}
(-*-):=\;
\begin{aligned}
\begin{tikzpicture}[xscale=\tikzxscale, yscale=\tikzyscale]
\node at (-0.75, -2) {$A$};
\node (0) at (-0.25, -2) {};
\node (0b) at (1, -2) {};
\node at (1.5, -2) {$A$};
\node (1) [brnode, xscale=1.5] at (0, 0) {};
\node at (0.25,0) {$f$};
\node (2) at (0.25, 2) {};
\node at (0.75, 2) {$A$};

\draw (0.center) to (-0.25,-0.6);
\draw (0b.center) to (1,-0.6);
\draw (0.25,0.6) to (2.center);
\end{tikzpicture}
\end{aligned}
\;=\;
\begin{aligned}
\begin{tikzpicture}[xscale=1.4*\tikzxscale, yscale=1.3*\tikzyscale]
\draw (2,0) to [out=up, in=\seangle] (1.25,1.5);
\draw (0.5,0) to [out=up, in=\swangle] (1.25, 1.5);
\draw (1.25,1.5) to (1.25, 3);
\node [whitedot] at (1.25,1.5) {};
\end{tikzpicture}
\end{aligned}
\qquad\qquad\quad
1:=\;
\begin{aligned}
\begin{tikzpicture}[xscale=\tikzxscale, yscale=\tikzyscale]
\draw (0,0) to (0,2);
\node [state] at (0,0) {$1$};
\end{tikzpicture}
\end{aligned}
\;=\;
\begin{aligned}
\begin{tikzpicture}[xscale=\tikzxscale, yscale=\tikzyscale]
\draw (0,0) to (0,2);
\node [whitedot] at (0,0) {};
\end{tikzpicture}
\end{aligned}

\end{equation}

\begin{defn}
\label{defn:monoid}
In a monoidal category, a \textit{monoid} is a triple \whitemonoid{A} of an object $A$, a morphism $\tinymult[whitedot] : A \otimes A \to A $ called the multiplication, and a state $\tinyunit[whitedot] : I \to A$ called the unit, satisfying associativity and unitality equations:
\begin{equation}
\label{eq:monoid}
\begin{aligned}
\begin{tikzpicture}[xscale=1.4*\tikzxscale, yscale=1.3*\tikzyscale]
\draw (0,0) to [out=up, in=\swangle] (0.5, 1);
\draw (1,0) to [out=up, in=\seangle] (0.5,1);
\draw (2,0) to [out=up, in=\seangle] (1.25,2);
\draw (0.5,1) to [out=up, in=\swangle] (1.25, 2);
\draw (1.25,2) to (1.25, 3);
\node [whitedot] at (0.5,1) {};
\node [whitedot] at (1.25,2) {};
\end{tikzpicture}
\end{aligned}
\quad=\quad
\begin{aligned}
\begin{tikzpicture}[xscale=-1.4*\tikzxscale, yscale=1.3*\tikzyscale]
\draw (0,0) to [out=up, in=\swangle] (0.5, 1);
\draw (1,0) to [out=up, in=\seangle] (0.5,1);
\draw (2,0) to [out=up, in=\seangle] (1.25,2);
\draw (0.5,1) to [out=up, in=\swangle] (1.25, 2);
\draw (1.25,2) to (1.25, 3);
\node [whitedot] at (0.5,1) {};
\node [whitedot] at (1.25,2) {};
\end{tikzpicture}
\end{aligned}
\qquad\qquad\qquad
\begin{aligned}
\begin{tikzpicture}[xscale=1.4*\tikzxscale, yscale=1.3*\tikzyscale]
\draw (0,-1.5) to (0,-0.5) to [out=up, in=\swangle] (0.75,0.5) node [whitedot] {} to (0.75,1.5);
\draw (1.5,-0.5) node [whitedot] {} to [out=up, in=\seangle] (0.75,0.5);
\end{tikzpicture}
\end{aligned}
\quad=\quad
\begin{aligned}
\begin{tikzpicture}[xscale=1.4*\tikzxscale, yscale=1.3*\tikzyscale]
\draw (0,0) to (0,3);
\end{tikzpicture}
\end{aligned}
\quad=\quad
\begin{aligned}
\begin{tikzpicture}[xscale=-1.4*\tikzxscale, yscale=1.3*\tikzyscale]
\draw (0,-1.5) to (0,-0.5) to [out=up, in=\swangle] (0.75,0.5) node [whitedot] {} to (0.75,1.5);
\draw (1.5,-0.5) node [whitedot] {} to [out=up, in=\seangle] (0.75,0.5);
\end{tikzpicture}
\end{aligned}

\end{equation}
\end{defn}

\begin{defn}
\label{defn:comonoid}
In a monoidal category, a \textit{comonoid} is a triple \whitecomonoid{A} of an object $A$, a morphism $\tinycomult[whitedot] : A \to A \otimes A$ called the comultiplication, and an effect $\tinycounit[whitedot] : A \to I$ called the counit, satisfying coassociativity and counitality equations:
\begin{equation}
\label{eq:comonoid}
\begin{aligned}
\begin{tikzpicture}[xscale=1.4*\tikzxscale, yscale=-1.3*\tikzyscale]
\draw (0,0) to [out=up, in=\swangle] (0.5, 1);
\draw (1,0) to [out=up, in=\seangle] (0.5,1);
\draw (2,0) to [out=up, in=\seangle] (1.25,2);
\draw (0.5,1) to [out=up, in=\swangle] (1.25, 2);
\draw (1.25,2) to (1.25, 3);
\node [whitedot] at (0.5,1) {};
\node [whitedot] at (1.25,2) {};
\end{tikzpicture}
\end{aligned}
\quad=\quad
\begin{aligned}
\begin{tikzpicture}[xscale=-1.4*\tikzxscale, yscale=-1.3*\tikzyscale]
\draw (0,0) to [out=up, in=\swangle] (0.5, 1);
\draw (1,0) to [out=up, in=\seangle] (0.5,1);
\draw (2,0) to [out=up, in=\seangle] (1.25,2);
\draw (0.5,1) to [out=up, in=\swangle] (1.25, 2);
\draw (1.25,2) to (1.25, 3);
\node [whitedot] at (0.5,1) {};
\node [whitedot] at (1.25,2) {};
\end{tikzpicture}
\end{aligned}
\qquad\qquad\qquad
\begin{aligned}
\begin{tikzpicture}[xscale=1.4*\tikzxscale, yscale=-1.3*\tikzyscale]
\draw (0,-1.5) to (0,-0.5) to [out=up, in=\swangle] (0.75,0.5) node [whitedot] {} to (0.75,1.5);
\draw (1.5,-0.5) node [whitedot] {} to [out=up, in=\seangle] (0.75,0.5);
\end{tikzpicture}
\end{aligned}
\quad=\quad
\begin{aligned}
\begin{tikzpicture}[xscale=1.4*\tikzxscale, yscale=-1.3*\tikzyscale]
\draw (0,0) to (0,3);
\end{tikzpicture}
\end{aligned}
\quad=\quad
\begin{aligned}
\begin{tikzpicture}[xscale=-1.4*\tikzxscale, yscale=-1.3*\tikzyscale]
\draw (0,-1.5) to (0,-0.5) to [out=up, in=\swangle] (0.75,0.5) node [whitedot] {} to (0.75,1.5);
\draw (1.5,-0.5) node [whitedot] {} to [out=up, in=\seangle] (0.75,0.5);
\end{tikzpicture}
\end{aligned}

\end{equation}
\end{defn}

\noindent We use differently colored dots to represent different monoids on the same object.
\begin{defn}
In a monoidal category with object $A$, a \emph{monoid homomorphism} $f:\whitemonoid{A}\to\blackmonoid{A}$ is a map $f:A\to A$ such that
\begin{equation}
\label{eq:mon_hom}
\begin{pic}[xscale=1.2*\tikzxscale, yscale=1.5*\tikzyscale]
    \node [style=morphism] (0) at (0, 1) {$f$};
    \node [style=whitedot] (1) at (0, -0) {};
    \node (2) at (1, -1) {};
    \node (3) at (-1, -1) {};
    \node (4) at (0, 2) {};

    \draw [bend left=45, looseness=1.00] (3.center) to (1);
    \draw [bend left=45, looseness=1.00] (1) to (2.center);
    \draw (1) to (0.south);
    \draw (0.north) to (4.center);
\end{pic}
\quad = \quad
\begin{pic}[xscale=1.2*\tikzxscale, yscale=1.5*\tikzyscale]
                \node [style=blackdot] (0) at (0, 1) {};
                \node [style=morphism] (1) at (1, -0) {$f$};
                \node [style=morphism] (2) at (-1, -0) {$f$};
                \node (3) at (0, 2) {};
                \node (4) at (1, -1) {};
                \node (5) at (-1, -1) {};
                \draw (4.center) to (1.south);
                \draw (5.center) to (2.south);
                \draw (0) to (3.center);
                \draw [bend left=45, looseness=1.00] (0) to (1.north);
                \draw [bend right=45, looseness=1.00] (0) to (2.north);
\end{pic}

\end{equation}
\end{defn}
\noindent A \emph{comonoid homomorphism} is defined similarly, but with the dagger of the condition in~\eqref{eq:mon_hom}.

We can then ask for the comonoid and monoid to interact in various ways.

\subsection{Abstract observables}

When monoids and comonoids combine under certain rules, we obtain the structure of classical information on systems in an oGCT.

\begin{defn}
\label{def:frobenius}
In a \dsmc the pair of a monoid \whitemonoid{A} and comonoid \whitecomonoid{A} form a \emph{dagger-Frobenius algebra} when the following equation holds:
\begin{equation}
\label{eq:frobenius}
\begin{aligned}
\begin{tikzpicture}[xscale=1.4*\tikzxscale, yscale=1.4*\tikzyscale]
\draw (0,0) to (0,1) to [out=up, in=\swangle] (0.5,2) node [whitedot] {} to (0.5,3);
\draw (0.5,2) to [out=\seangle, in=\nwangle] (1.5,1) node [whitedot] {};
\draw (1.5,0) to (1.5,1) to [out=\neangle, in=down] (2,2) to (2,3);
\end{tikzpicture}
\end{aligned}
\quad = \quad
\begin{aligned}
\begin{tikzpicture}[xscale=-1.4*\tikzxscale, yscale=1.4*\tikzyscale]
\draw (0,0) to (0,1) to [out=up, in=\swangle] (0.5,2) node [whitedot] {} to (0.5,3);
\draw (0.5,2) to [out=\seangle, in=\nwangle] (1.5,1) node [whitedot] {};
\draw (1.5,0) to (1.5,1) to [out=\neangle, in=down] (2,2) to (2,3);
\end{tikzpicture}
\end{aligned}

\end{equation}
\end{defn}

\begin{defn}
\label{def:classicalstruct}
A \emph{classical structure} is a dagger-Frobenius algebra \whitecomonoid{A} satisfying the \emph{specialness} \eqref{eq:special} and \emph{symmetry} \eqref{eq:sym} conditions:
\begin{equation}
\label{eq:special}
\begin{aligned}
\begin{tikzpicture}[xscale={1.5*\tikzxscale}, yscale={1.5*\tikzyscale}]
\draw (0,0.25) to (0,1) node [whitedot] {} to [out=\nwangle, in=down] (-0.5,1.5) to [out=up, in=\swangle] (0,2) node [whitedot] {} to (0,2.75);
\draw (0,1) to [out=\neangle, in=down] (0.5,1.5) to [out=up, in=\seangle] (0,2);
\end{tikzpicture}
\end{aligned}
\quad=\quad
  \begin{aligned}
  \begin{tikzpicture}[xscale={1.5*\tikzxscale}, yscale={1.5*\tikzyscale}]
  \draw (-0.5,0) to (-0.5,3);
  \end{tikzpicture}
  \end{aligned}
  \end{equation}
  
  \vspace{-10pt}
  \begin{equation}
  \label{eq:sym}
\begin{aligned}
\begin{tikzpicture}[xscale={1.5*\tikzxscale}, yscale={1.5*\tikzyscale}]
\draw (0,-0.5) node [whitedot] {} to (0,0.5) node [whitedot] {} to [out=\nwangle, in=down] (-0.5,1.0) to [out=up, in=down] (0.5,2);
\draw (0,0.5) to [out=\neangle, in=down] (0.5,1) to [out=up, in=down] (-0.5,2);
\end{tikzpicture}
\end{aligned}
\quad=\quad
\begin{aligned}
\begin{tikzpicture}[xscale={1.5*\tikzxscale}, yscale={1.5*\tikzyscale}]
\draw (0,-0.5) node [whitedot] {} to (0,0.5) node [whitedot] {} to [out=\nwangle, in=down] (-0.5,1.0) to [out=up, in=down] (-0.5,2);
\draw (0,0.5) to [out=\neangle, in=down] (0.5,1) to [out=up, in=down] (0.5,2);
\end{tikzpicture}
\end{aligned}
\end{equation}
\end{defn}

\begin{defn}
\label{def:copyables}
The set of \emph{classical states} $K_{\circ}$ for a classical structure \whitecomonoid{A} are all states $j:I\to A$ such that:
\begin{equation}
\label{eq:copy}
\begin{pic}[xscale=\tikzxscale, yscale=-0.8*\tikzyscale]
\node [state] at (1.25,3) {$j$};
\draw (2,0) to [out=up, in=\seangle] (1.25,1.5);
\draw (0.5,0) to [out=up, in=\swangle] (1.25, 1.5);
\draw (1.25,1.5) to (1.25, 3);
\node [whitedot] at (1.25,1.5) {};
\end{pic}
\quad=\quad
\begin{pic}[xscale=\tikzxscale, yscale=\tikzyscale]
\node (a) [state] at (2.5,0) {$j$};
\node (b) [state] at (0,0) {$j$};
\draw (a) to (2.5,2);
\draw (b) to (0,2);
\end{pic}
\end{equation}
\end{defn}
\noindent These classical points will be drawn as states that are the same color as the classical structure to which they correspond. It is also easy to determine the behavior of classical points under composition with the (co)unit:
\begin{equation}
\label{eq:unitscalar}
\begin{pic}[xscale=\tikzxscale, yscale=-0.8*\tikzyscale]
\node [state] at (1.25,3) {$j$};
\draw (1.25,0) to (1.25, 3);
\end{pic}
\;\stackrel{\eqref{eq:comonoid}}{=}\;
\begin{pic}[xscale=\tikzxscale, yscale=-0.8*\tikzyscale]
\node [state] at (1.25,3) {$j$};
\draw (2,0) to [out=up, in=\seangle] (1.25,1.5);
\draw (0.5,0) to [out=up, in=\swangle] (1.25, 1.5);
\draw (1.25,1.5) to (1.25, 3);
\node [whitedot] at (1.25,1.5) {};
\node [whitedot] at (0.5,0) {};
\end{pic}
\;\stackrel{\eqref{eq:copy}}{=}\;
\begin{pic}[xscale=\tikzxscale, yscale=\tikzyscale]
\node (a) [state] at (2.5,0) {$j$};
\node (b) [state] at (0,0) {$j$};
\draw (a) to (2.5,2);
\draw (b) to (0,2);
\node [whitedot] at (0,2) {};
\end{pic}
\quad\qquad\Leftrightarrow\qquad\quad
\begin{pic}[xscale=\tikzxscale, yscale=\tikzyscale]

\node (b) [state] at (0,0) {$j$};

\draw (b) to (0,2);
\node [whitedot] at (0,2) {};
\end{pic}
\;=\;1
\end{equation}
\noindent where $1$ is the unit scalar, i.e. $1\circ s = s = s\circ 1$ for all scalars $s$. In this sense the counit ``erases" classical points.

The following theorems characterize the relationship between classical structures and observables. 

\begin{theorem}[{\cite[Cor 4.3]{coecke2013new}}]
Special dagger Frobenius algebras in \cat{FHilb} are precisely finite-dimensional C$^*$-algebras.
\end{theorem}
\noindent Note that this correspondence holds for both symmetric and non-symmetric special dagger Frobenius algebras, which are commutative and non-commutative $C^*$ algebras respectively.

\begin{theorem}[{\cite[Thm 5.1]{coecke2013new}}]
Symmetric dagger Frobenius algebras in FHilb are orthogonal bases.
\end{theorem}
\noindent The additional condition of specialness for classical structures acts as a normalizing condition so that:
\begin{theorem}[{\cite[Sec 6]{coecke2013new}}]
\label{thm:cstructBases}
Classical structures in FHilb are orthonormal bases.
\end{theorem}

\noindent \noindent Classical structures are bases in \cat{FHilb} and so are recognized as generalized \emph{observables} in an oGCT. The copyable states of classical structures  (Definition~\ref{def:copyables}) form the elements of these bases, more specifically the eigenvectors of the observables, though we should be careful that in categories other than $\cat{FHilb}$ they do not necessarily have all the familiar properties of bases for Hilbert spaces.  For example, we usually expect to be able to distinguish maps by testing them on basis elements. In general this only holds for certain classical structures:

\begin{defn}
\label{def:enoughclassicalpoints}
A classical structure has \emph{enough classical states} when for all processes $f,g:A\to B$: 
\begin{align}
f=g \qquad \mbox{iff}\qquad \left(\forall \ket{j}\in K_{\circ}:f\circ\ket{j}=g\circ\ket{j}\right)\; 
\end{align}
\end{defn}

\begin{remark}
The correspondence in Theorem \ref{thm:cstructBases} was modified for the infinite dimensional case in \cite{abramsky2012h}, where it still holds. We will only use the the finite dimensional case in this thesis as we are concerned only with algorithms that run on computers of finite size.
\end{remark}

\begin{defn}
\label{def:dimension}
In an oGCT, the \emph{dimension} $d(A)$ of an object $A$ equipped with a dagger-Frobenius algebra \whitecomonoid{A} is given by the following composite:
\begin{equation}
\label{eq:dim}
\ud(A)
\quad := \quad
\begin{aligned}
\begin{tikzpicture}[xscale=-1.4*\tikzxscale, yscale=1.4*\tikzyscale]
    \node (0) at (0,0) {};
    \node[whitedot] (1) at (0,0.66) {};
    \node (2) at (-0.5,1.2) {};
    \node (3) at (0.5,1.2) {};
    \node (4) at (-0.5,2.0) {};
    \node (5) at (0.5,2.0) {};
    \draw[string] (0.center) to (1.center);
    \draw[string, out=180, in=270] (1.center) to (2.center);
    \draw[string, out=0, in=270] (1.center) to (3.center);
    \draw[string, out=90, in=270] (2.center) to (5.center);
    \draw[string, out=90, in=270] (3.center) to (4.center);;
    \node[whitedot] (6) at (0,2.54) {};        
    \node (7) at (0,3.2) {};
    \draw[string] (0.center) node [whitedot] {} to (1);
    \draw[string] (6.center) to (7.center) node [whitedot] {$$};
    \draw[string, in=left, out=up] (4.center) to node [auto] {$$} (6.center);
    \draw[string, in=right, out=up] (5.center) to node [auto, swap] {$$} (6.center);
\end{tikzpicture}
\end{aligned}
\end{equation}
\end{defn}
\noindent
When the algebra is in fact a classical structure, Equation~\eqref{eq:dim} can be simplified to the composition of the unit and counit:
\begin{equation}
\label{eq:defdim}
d(A) \;= \;
\begin{pic}[xscale={\tikzxscale}, yscale={\tikzyscale}]
\node (t) [whitedot] at (0,1) {};
\node (b) [whitedot] at (0,-1) {};
\draw (t) to (b.center);
\end{pic}
\end{equation}

\begin{remark}
It is important to note that the dimension of a system does not always equal the number of classical states of the associated classical structure. One setting where this does not hold is \cat{FRel}, where this fact plays an important role in its oGCT characterization in Chapter~\ref{chap:mermin}.
\end{remark}

The rules for classical structures can be summarized in a convenient normal form that is in keeping with the sorts of topological equivalences we are used to for diagrams of an oGCT. Let the maps $\tinycomult[whitedot]_n:A\to A^{\otimes n}$ be defined recursively by:
\begin{align}
\tinycomult[whitedot]_0:=\tinycounit[whitedot]
\qquad\qquad
\tinycomult[whitedot]_{n+1}:=\left(\tinycomult[whitedot]_{n}\otimes\idm{A}\right)\circ\tinycomult[whitedot]
\end{align}
Let $\tinymult[whitedot]_n$ be similarly defined.
\begin{theorem}[Spider Theorem]
\label{thm:spider}
Given a classical structure, let \newline$f:A^{\otimes n}\to A^{\otimes m}$  be constructed from $\{\tinymult[whitedot],\tinycomult[whitedot],\tinyunit[whitedot],\tinycounit[whitedot]\}$ such that the diagram is connected. Then $f=\tinycomult[whitedot]_m\circ\tinymult[whitedot]_n$.
\end{theorem}

\noindent This normal form has a neat diagrammatic representation.

\begin{proposition}
\label{prop:spider} 
Given a classical structure on $A$, let
$(\whitedot)_n^m$ denote the `$(n,m)$-legged spider':
\begin{equation}
 \begin{pic}
        \node  (0) at (-2, 1) {};
        \node  (1) at (-0.5, 1) {};
        \node  (2) at (0, 1) {};
        \node  (3) at (1, 1) {};
        \node  (4) at (1.5, 1) {};
        \node  (5) at (-2, 0.75) {};
        \node  (6) at (-1.5, 0.75) {};
        \node  (7) at (-0.5, 0.75) {};
        \node [style=whitedot] (8) at (1.25, 0.75) {};
        \node  (9) at (-1, 0.5) {...};
        \node  (10) at (1, 0.5) {\small \rotatebox[origin=c]{45}{...}};
        \node [style=whitedot] (11) at (0.75, 0.25) {};
        \node [style=whitedot] (12) at (-1.25, 0) {};
        \node [anchor=east] (13) at (0, 0) {$:=$};
        \node [style=whitedot] (14) at (0.75, -0.25) {};
        \node  (15) at (-1, -0.5) {...};
        \node  (16) at (1, -0.5) {\small \rotatebox[origin=c]{-45}{...}};
        \node  (17) at (-2, -0.75) {};
        \node  (18) at (-1.5, -0.75) {};
        \node  (19) at (-0.5, -0.75) {};
        \node [style=whitedot] (20) at (1.25, -0.75) {};
        \node  (21) at (-2, -1) {};
        \node  (22) at (-0.5, -1) {};
        \node  (23) at (0, -1) {};
        \node  (24) at (1, -1) {};
        \node  (25) at (1.5, -1) {};

        \draw [bend left=15] (18.center) to (12);
        \draw (14) to (11);
        \draw (11) to (2.center);
        \draw [style=small braceedge] (22.center) to node[wire label, inner sep=3 pt]{$n$} (21.center);
        \draw (8) to (4.center);
        \draw (23.center) to (14);
        \draw (24.center) to (20);
        \draw (8) to (3.center);
        \draw [bend left=15] (12) to (5.center);
        \draw [bend right=15] (19.center) to (12);
        \draw [bend left=15] (17.center) to (12);
        \draw [bend right=15] (12) to (7.center);
        \draw (25.center) to (20);
        \draw [bend left=15] (12) to (6.center);
        \draw [style=small braceedge] (0.center) to node[wire label, inner sep=3 pt]{$m$} (1.center);
\end{pic}

\end{equation}
then any process $A^{\otimes n}\to A^{\otimes m}$ built from $\{\tinymult[whitedot], \tinycomult[whitedot], \tinyunit[whitedot],\tinycounit[whitedot]\}$ which has a connected graph is equal to $(\whitedot)^m_n$. Spider
composition is:
\begin{equation}\label{eq:spidercomp}
 \begin{pic}
                \node (0) at (-4, 1.25) {};
                \node  (1) at (-3.5, 1.25) {...};
                \node  (2) at (-3, 1.25) {};
                \node  (3) at (-1.5, 1.25) {};
                \node  (4) at (-1, 1.25) {...};
                \node  (5) at (-0.5, 1.25) {};
                \node  (6) at (0.75, 1.25) {};
                \node  (7) at (1.5, 1.25) {};
                \node  (8) at (2.75, 1.25) {};
                \node  (9) at (2, 0.75) {...};
                \node [style=whitedot] (10) at (-3.25, 0.5) {};
                \node [font=\footnotesize] (11) at (-2.5, 0) {...};
                \node  (12) at (0, 0) {$:=$};
                \node [style=whitedot] (13) at (1.75, 0) {};
                \node [style=whitedot] (14) at (-1.75, -0.5) {};
                \node  (15) at (2, -0.75) {...};
                \node  (16) at (-4.5, -1.25) {};
                \node  (17) at (-4, -1.25) {...};
                \node  (18) at (-3.5, -1.25) {};
                \node  (19) at (-2, -1.25) {};
                \node  (20) at (-1.5, -1.25) {...};
                \node  (21) at (-1, -1.25) {};
                \node  (22) at (0.75, -1.25) {};
                \node  (23) at (1.5, -1.25) {};
                \node  (24) at (2.75, -1.25) {};
                \draw [bend left=15] (22.center) to (13);
                \draw [bend left=15] (19.center) to (14);
                \draw [bend right=15] (10) to (2.center);
                \draw [in=-60, out=195] (14) to (10);
                \draw [in=15, out=116] (14) to (10);
                \draw [bend right=15] (18.center) to (10);
                \draw [bend left=15] (14) to (3.center);
                \draw [bend right=15] (13) to (8.center);
                \draw [bend right=15] (21.center) to (14);
                \draw [bend left=15] (23.center) to (13);
                \draw [bend right=15] (24.center) to (13);
                \draw [bend left=15] (10) to (0.center);
                \draw [bend right=15, looseness=0.75] (14) to (5.center);
                \draw [bend left=15, looseness=0.75] (16.center) to (10);
                \draw [bend left=15] (13) to (6.center);
                \draw [bend left=15] (13) to (7.center);
\end{pic}

\end{equation}
\end{proposition}

The spider rule makes it clear that every classical structure on $A$ can be used to make $A$ dual to itself (Definition~\ref{def:dual}) using cups and caps built from the white dot:
\tikzeq{tikz/2_frobcomp.tikz}

The upper and lower star operations with respect to this white dot cup and cap corresponds in \cat{FHilb} to transposition and conjugation in the white dot  basis respectively. Further it can be shown that transposition with respect to a classical structure is equivalent to the dagger when applied to its classical states:
\begin{equation}
\label{eq:dagfrob}
\begin{pic}[scale=0.75]
                \node (0) at (-5, -0.5) {};
                \node (1) at (-1.5, -0.5) {};
                \node [style={graydot}] (2) at (2, -0.5) {};
                \node [style=point] (3) at (-3.5, -0.25) {$i$};
                \node [style={gray point}] (4) at (4.5, -0.25) {$j$};
                \node (5) at (-2.5, 0.25) {$=$};
                \node (6) at (3.5, 0.25) {$=$};
                \node [style=copoint] (7) at (-1.5, 0.5) {$i$};
                \node [style={gray copoint}] (8) at (1.25, 0.5) {$j$};
                \node [style={whitedot}] (9) at (-4.25, 0.75) {};
                \node (10) at (2.75, 0.75) {};
                \node (11) at (4.5, 0.75) {};
                \draw [in=-15, out=90, looseness=1.00] (3) to (9.center);
                \draw [in=-90, out=165, looseness=1.00] (2.center) to (8);
                \draw [in=-165, out=90, looseness=1.00] (0.center) to (9.center);
                \draw [in=-90, out=15, looseness=1.00] (2.center) to (10.center);
                \draw (1.center) to (7);
                \draw (4) to (11.center);
\end{pic}

\end{equation}
\noindent This is of course not true on general states.

\section{Phases}
\label{sec:phases}
Phases for oGCTs were introduced by \cite{coecke2011interacting}.

\begin{defn}
\label{def:phases}
A \emph{phase state} for a Frobenius algebra $\whitecomonoid{A}$ is a state $\ket{\alpha}$ such that:
\begin{equation}
\label{eqn:zphasestate}
\begin{pic}[xscale=\tikzxscale, yscale=\tikzyscale]
\node [style=state] (0) at (0, -1) {$a$};
\node [style=whitedot] (1) at (0, -0) {};
\node [style=state, hflip] (2) at (-1, 1) {$a$};
\node (3) at (1, 1) {};
\node (4) at (1, 2) {};                

\draw (0) to (1);
\draw (3.center) to (4);
\draw [bend left=45, looseness=1.00] (1) to (2);
\draw [bend right=45, looseness=1.00] (1) to (3.center);
\end{pic}
\;=\;
\begin{pic}[xscale=\tikzxscale, yscale=\tikzyscale]
\draw (0,0) to (0,2);
\node [whitedot] at (0,0) {};
\end{pic}
\;=\;
\begin{pic}[xscale=-1*\tikzxscale, yscale=\tikzyscale]
\node [style=state] (0) at (0, -1) {$a$};
\node [style=whitedot] (1) at (0, -0) {};
\node [style=state, hflip] (2) at (-1, 1) {$a$};
\node (3) at (1, 1) {};
\node (4) at (1, 2) {};                

\draw (0) to (1);
\draw (3.center) to (4);
\draw [bend left=45, looseness=1.00] (1) to (2);
\draw [bend right=45, looseness=1.00] (1) to (3.center);
\end{pic}

\end{equation}
Each phase state corresponds to a \emph{phase} that is a unitary map in the following form:
\begin{equation}
\label{eqn:zphase}
\begin{tikzpicture}[xscale=1.4*\tikzxscale, yscale=1.4*\tikzyscale]
\node [whitedot] (0) at (-1, -0) {$\alpha$};
\node (1) at (-1, -1) {};
\node (2) at (-1, 1) {};
\node (3) at (0, -0) {$:=$};
\node [whitedot] (4) at (1, -0) {};
\node (5) at (1, -1) {};
\node (6) at (1, 1) {};
\node [state] (7) at (2, -0.5) {$\alpha$};

\begin{pgfonlayer}{background}
\draw [bend right=45, looseness=0.75] (7) to (4);
\draw (5.center) to (4);
\draw (4) to (6.center);
\draw (1.center) to (0);
\draw (0) to (2.center);
\end{pgfonlayer}
\end{tikzpicture}

\end{equation}
\end{defn}

\begin{proposition}[Phase groups]
Given a dagger Frobenius algebra $\whitemonoid{A}$ in an oGCT, its phases form a group under the following addition:
\begin{equation}
\begin{pic}[xscale=\tikzxscale, yscale=\tikzyscale]
\draw (0,0) to (0,2);
\node [state] at (0,0) {$ab$};
\end{pic}
\;=\;
\begin{pic}[xscale=1.2*\tikzxscale, yscale=1.1*\tikzyscale]
\node [style=whitedot] (0) at (0, 0) {};
\node [style=state] (1) at (-1, -1) {$a$};
\node [style=state] (2) at (1, -1) {$a$};
\node (3) at (0, 1) {};
\draw [bend right=45, looseness=1.00] (0) to (1);
\draw [bend left=45, looseness=1.00] (0) to (2);
\draw (0) to (3.center);
\end{pic}
\end{equation}
with unit $\tinyunit[whitedot]$. When the Frobenius algebra is part of a classical structure, then its phases form an abelian group.
\end{proposition}
\begin{proof}
\todo{Include proof or cite CQM notes?}
\end{proof}

Phases can be added to the normal form from Theorem~\ref{thm:spider} to give a decorated spider rule~\cite[Thm 7.11]{coecke2011interacting}:
\begin{equation}
\label{eq:decspider}
\begin{pic}[font=\footnotesize, scale=1.5]
                \node  (0) at (-1.5, 0.5) {};
                \node  (1) at (-1, 0.5) {\raisebox{-2mm}{...}};
                \node  (2) at (-0.5, 0.5) {};
                \node  (3) at (0.75, 0.5) {};
                \node  (4) at (1.25, 0.5) {\raisebox{-2mm}{...}};
                \node  (5) at (1.75, 0.5) {};
                \node [style=whitedot] (6) at (-1, 0) {$\alpha$};
                \node  (7) at (0, 0) {$:=$};
                \node [style=whitedot] (8) at (1.25, 0) {};
                \node [style=whitedot] (9) at (2, -0.25) {$\alpha$};
                \node  (10) at (-1.5, -0.5) {};
                \node  (11) at (-1, -0.5) {\raisebox{2mm}{...}};
                \node  (12) at (-0.5, -0.5) {};
                \node  (13) at (0.5, -0.5) {};
                \node  (14) at (1, -0.5) {\raisebox{2mm}{...}};
                \node  (15) at (1.5, -0.5) {};
                \draw [bend left=15] (10.center) to (6);
                \draw [bend right=15] (8) to (5.center);
                \draw [bend left=15] (8) to (3.center);
                \draw [in=-15, out=150, looseness=0.75] (9) to (8);
                \draw [bend right=15] (12.center) to (6);
                \draw [bend right=15] (15.center) to (8);
                \draw [bend right=15] (6) to (2.center);
                \draw [bend left=15] (6) to (0.center);
                \draw [bend left=15] (13.center) to (8);
\end{pic}

\end{equation}
with composition
\begin{equation}
\begin{pic}[font=\footnotesize, scale=1.5]
                \node  (0) at (-2.5, 0.75) {};
                \node  (1) at (-2, 0.75) {\raisebox{-2mm}{...}};
                \node  (2) at (-1.5, 0.75) {};
                \node  (3) at (-1, 0.75) {};
                \node  (4) at (-0.5, 0.75) {\raisebox{-2mm}{...}};
                \node  (5) at (0, 0.75) {};
                \node  (6) at (1, 0.75) {};
                \node  (7) at (1.5, 0.75) {\raisebox{-2mm}{...}};
                \node  (8) at (2, 0.75) {};
                \node  (9) at (-1, 0.5) {};
                \node  (10) at (0, 0.5) {};
                \node [style=whitedot] (11) at (-1.75, 0.25) {$\alpha$};
                \node  (12) at (-1.25, 0) {\rotatebox[origin=c]{45}{...}};
                \node  (13) at (0.5, 0) {$:=$};
                \node [style=whitedot] (14) at (1.5, 0) {$\!\alpha\!+\!\beta\!$};
                \node [style=whitedot] (15) at (-0.75, -0.25) {$\beta$};
                \node  (16) at (-2.5, -0.5) {};
                \node  (17) at (-1.5, -0.5) {};
                \node  (18) at (-2.5, -0.75) {};
                \node  (19) at (-2, -0.75) {\raisebox{2mm}{...}};
                \node  (20) at (-1.5, -0.75) {};
                \node  (21) at (-1, -0.75) {};
                \node  (22) at (-0.5, -0.75) {\raisebox{2mm}{...}};
                \node  (23) at (0, -0.75) {};
                \node  (24) at (1, -0.75) {};
                \node  (25) at (1.5, -0.75) {\raisebox{2mm}{...}};
                \node  (26) at (2, -0.75) {};
                \draw [in=210, out=90, looseness=1.25] (16.center) to (11);
                \draw [in=-60, out=180] (15) to (11);
                \draw [in=-15, out=120] (15) to (11);
                \draw [bend right=15] (11) to (2.center);
                \draw [in=-90, out=45] (15) to (10.center);
                \draw [in=270, out=105] (15) to (9.center);
                \draw [bend right=15] (23.center) to (15);
                \draw [bend left=15] (24.center) to (14);
                \draw [in=-75, out=90, looseness=1.25] (17.center) to (11);
                \draw (20.center) to (17.center);
                \draw [bend right=15] (14) to (8.center);
                \draw [bend left=15] (21.center) to (15);
                \draw [bend right=15] (26.center) to (14);
                \draw (9.center) to (3.center);
                \draw [bend left=15] (14) to (6.center);
                \draw (18.center) to (16.center);
                \draw (10.center) to (5.center);
                \draw [bend left=15] (11) to (0.center);
\end{pic}

\end{equation}
This normal form is a powerful simplifying tool and will often be used in the analysis of diagrams of processes in oGCTs.

\section{Complementarity}
The notion of complementary bases can also lifted to the abstract level of classical structures in oGCTs~\cite{coecke2011interacting}. This extension is perhaps best presented as emergent from a suitable generalization of mutual unbiasedness.\footnote{Our presentation in this regard is heavily influenced by the approach in \cite{coecke2015generalised}.}  This means that for two bases $\{\ket{i}\}$ and $\{\ket{j}\}$ on a $D$-dimensional Hilbert space, $\langle i|j\rangle = 1/D$ for all $i,j$. In diagrams we then have:
\tikzeq{tikz/2_mub.tikz}
as the mutual unbiasedness condition, where we have assumed that $D$ is invertible and used \eqref{eq:defdim} and \eqref{eq:unitscalar} to obtain to the right hand form.
\begin{defn}[Complementarity]
\label{def:complementarity}
In an oGCT, two classical structures \whitecomonoid{A} and \graycomonoid{A} are \emph{complementary} when the following equation holds up to a scalar:
\begin{equation}
\label{eq:complementarity}
\begin{pic}[scale=0.75]
                \node (0) at (-2, 1.75) {};
                \node (1) at (1.25, 1.5) {};
                \node [style=graydot] (2) at (-2, 1) {};
                \node [style=graydot] (3) at (1.25, 0.5) {};
                \node [style=antipode] (4) at (-1.25, 0) {$S$};
                \node (5) at (0, 0) {$=$};
                \node [style=whitedot] (6) at (1.25, -0.5) {};
                \node [style=whitedot] (7) at (-2, -1) {};
                \node (8) at (-2, -1.75) {};
                \node (9) at (1.25, -1.5) {};
                \node [style=whitedot] (10) at (-3.5, 0.5) {};
                \node [style=whitedot] (11) at (-3.5, -0.5) {};

                \draw (8.center) to (7);
                \draw [in=-90, out=15, looseness=1.00] (7) to (4);
                \draw (9.center) to (6);
                \draw [in=-165, out=165, looseness=1.00] (7) to (2);
                \draw (3) to (1.center);
                \draw (2) to (0.center);
                \draw [in=-15, out=90, looseness=1.00] (4) to (2);
                \draw (10) to (11);
\end{pic}

\qquad \mbox{where} \qquad
\begin{pic}[scale=0.75]
                \node (0) at (-1.75, 0.75) {};
                \node (1) at (2, 0.75) {};
                \node [style=whitedot] (2) at (0.5, 0.5) {};
                \node [style=antipode] (3) at (-1.75, 0) {$S$};
                \node (4) at (-0.75, 0) {$:=$};
                \node (5) at (0, 0) {};
                \node (6) at (1, 0) {};
                \node (7) at (2, 0) {};
                \node [style=graydot] (8) at (1.5, -0.5) {};
                \node (9) at (-1.75, -0.75) {};
                \node (10) at (0, -0.75) {};
                \draw (7.center) to (1.center);
                \draw (10.center) to (5.center);
                \draw (3) to (0.center);
                \draw [in=-90, out=165] (8) to (6.center);
                \draw [in=-90, out=15] (8) to (7.center);
                \draw [in=-165, out=90, looseness=1.25] (5.center) to (2);
                \draw (9.center) to (3);
                \draw [in=-15, out=90] (6.center) to (2);
\end{pic}

\end{equation}
The map $S:A\to A$ is called the \emph{antipode}.
\end{defn}
The Equation \eqref{eq:complementarity} is equivalent to mutual unbiasedness on classical structures that have enough classical points by Definition \ref{def:enoughclassicalpoints}. We show this with a quick calculation:
\tikzeq{tikz/2_complemub.tikz}
where $(*)$ is an application of Theorem \ref{thm:fund}, and the last step uses the fact that there are enough classical points.  In some literature the dimensional scalar is explicitly included in the abstract definition of strong complementarity, but, for the purposes of this thesis, it is suffices to define complementarity up to a scalar. 
%We refer to the gray states $\{\ket{j}\}$ as \emph{unbiased states} for %the $\dotonly{whitedot}$-classical structure. Similarly the white states %$\{\keti}\}$ are unbiased states for the $\dotonly{graydot}$-classical structure.

Complementarity relates the classical states of classical structures to their phase groups.
\begin{lemma}
\label{lem:phaseunbiased}
If two classical structures in an oGCT are complementary, then, up to an idempotent scalar, a state that is a classical point of one is a phase state for the other.
\end{lemma}
\begin{proof}
We prove this using diagrams, following~\cite{qcs-notes}:
\begin{equation}
\begin{pic}[xscale=-1*\tikzxscale, yscale=\tikzyscale]
\node [style=state] (0) at (0, -1) {$a$};
\node [style=graydot] (1) at (0, -0) {};
\node [style=state, hflip] (2) at (-1, 1) {$a$};
\node (3) at (1, 1) {};
\node (4) at (1, 2) {};                

\draw (0) to (1);
\draw (3.center) to (4);
\draw [bend left=45, looseness=1.00] (1) to (2);
\draw [bend right=45, looseness=1.00] (1) to (3.center);
\end{pic}
\;\stackrel{\small \mbox{Thm}~\ref{thm:spider}}{=}\;
\begin{pic}[xscale=\tikzxscale, yscale=\tikzyscale]
                \node [style=state] (0) at (-2, -0.5) {$a$};
                \node [style=graydot] (1) at (-1, 0.5) {};
                \node [style=graydot] (2) at (1, -1.5) {};
                \node [style=state, hflip] (3) at (2, -0.5) {$a$};
                \node (4) at (-1, 1.5) {};
                \node (5) at (0, -0.5) {};
                \draw [bend left=45, looseness=0.75] (0) to (1);
                \draw (1) to (4.center);
                \draw [bend right=45, looseness=1.00] (2) to (3);
                \draw [bend left=45, looseness=0.75] (1) to (5.center);
                \draw [bend right=45, looseness=0.75] (5.center) to (2);
\end{pic}
\;\stackrel{\eqref{eq:sym}}{=}\;
\begin{pic}[xscale=\tikzxscale, yscale=\tikzyscale]
                \node [style=state] (0) at (-2, -0.25) {$a$};
                \node [style=graydot] (1) at (-1, 1.5) {};
                \node [style=state, hflip] (2) at (0, -0.5) {$a$};
                \node (3) at (-1, 2.5) {};
                \node (4) at (1.75, -0.5) {};
                \node [style=graydot] (5) at (1, -1.5) {};

                \draw [bend left=45, looseness=0.75] (0) to (1);
                \draw (1) to (3.center);
                \draw [bend left=60, looseness=0.50] (1) to (4.center);
                \draw [bend right=45, looseness=1.00] (5) to (4.center);
                \draw [in=-90, out=180, looseness=1.00] (5) to (2);
\end{pic}
\qquad\qquad\qquad\quad\;\;\;
\end{equation}
\begin{equation}
\stackrel{\eqref{eq:dagfrob}}{=}
\begin{pic}[xscale=\tikzxscale, yscale=\tikzyscale]
                \node [style=state] (0) at (-3.25, -1.5) {$a$};
                \node [style=graydot] (1) at (-1.5, 0.75) {};
                \node (2) at (-1.5, 1.75) {};
                \node (3) at (1.75, -0.5) {};
                \node [style=graydot] (4) at (1, -1.5) {};
                \node [style=state] (5) at (-1.25, -1.5) {$a$};
                \node [style=whitedot] (6) at (-0.5, -0.5) {};
                \node (7) at (0.25, -1) {};
                \draw [bend left=45, looseness=0.75] (0) to (1);
                \draw (1) to (2.center);
                \draw [bend left=60, looseness=0.50] (1) to (3.center);
                \draw [bend right=45, looseness=1.00] (4) to (3.center);
                \draw [bend right=45, looseness=1.00] (6) to (5);
                \draw [bend left, looseness=1.25] (4) to (7.center);
                \draw [bend right, looseness=1.00] (7.center) to (6);
\end{pic}
\stackrel{\eqref{eq:copy}}{=}
\begin{pic}[xscale=\tikzxscale, yscale=\tikzyscale]
                \node [style=graydot] (0) at (-1.5, 0.75) {};
                \node (1) at (-1.5, 1.75) {};
                \node (2) at (1.75, -0.5) {};
                \node [style=graydot] (3) at (1, -1.5) {};
                \node [style=whitedot] (4) at (-0.5, -0.5) {};
                \node (5) at (0.25, -1) {};
                \node [style=whitedot] (6) at (-2, -1.5) {};
                \node [style=state] (7) at (-2, -2.25) {$a$};
                \node (8) at (-2.75, -0.25) {};
                \node (9) at (-1.25, -1) {};
                \draw (0) to (1.center);
                \draw [bend left=60, looseness=0.50] (0) to (2.center);
                \draw [bend right=45, looseness=1.00] (3) to (2.center);
                \draw [bend left, looseness=1.25] (3) to (5.center);
                \draw [bend right, looseness=1.00] (5.center) to (4);
                \draw (7) to (6);
                \draw [bend left, looseness=1.00] (6) to (8.center);
                \draw [bend right=45, looseness=1.00] (0) to (8.center);
                \draw [bend right, looseness=1.00] (6) to (9.center);
                \draw [bend right, looseness=1.00] (4) to (9.center);
\end{pic}
\stackrel{\eqref{eq:complementarity}}{=}
\begin{pic}
\node [state] (1) at (0,-1) {$a$};
\node [whitedot] (2) at (0,-0.5) {};
\node [graydot] (3) at (0,0.25) {};
\node (4) at (0,1) {};
\draw (1) to (2);
\draw (3) to (4);
\end{pic}
\stackrel{\eqref{eq:unitscalar}}{=}\;
\begin{pic}[xscale=\tikzxscale, yscale=\tikzyscale]
\draw (0,0) to (0,2);
\node [graydot] at (0,0) {};
\end{pic}

\end{equation}
\end{proof}
Though the converse of this lemma holds in \cat{FHilb}, i.e. each phase is also a classical point of the complementary structure, the converse does not hold in general.

%\noindent 
%Note that this is not a symmetric condition between the gray and white structures. %However, thanks to the symmetric property of the dagger-Frobenius algebras, %it is equivalent to the following alternative condition:
%\begin{equation}
%\ud(A) \,\,
\begin{tikzpicture}[xscale=1.4*\tikzxscale, yscale=-1.4*\tikzyscale]
\draw (-0.5,0.25) to (-0.5,1) node [whitedot] {} to [out=left, in=right] (-1,2) node [whitedot] {} to [out=left, in=right] (-1.5,1.5) node [graydot] {} to [out=left, in=down] (-2,2) to [out=up, in=left] (-0.75,3) node (a) [graydot] {} to [out=right, in=right] (-0.5,1);
\draw (a.center) to +(0,0.75);
\end{tikzpicture}
\quad=\quad\,\,\,
\begin{tikzpicture}[xscale=1.4*\tikzxscale, yscale=1.4*\tikzyscale]
\draw (0,0.25) to (0,1) node [graydot] {};
\draw (0,3) node [whitedot] {} to (0,3.75);
\end{tikzpicture}
%\end{equation}
%The daggers of these equations give rise to two further equivalent conditions.

\begin{defn}\label{def:coherence}
In an oGCT, two classical structures \whitecomonoid{A} and \graycomonoid{A} are \emph{coherent} when the following equations hold:
\begin{equation}
\label{eq:coherence}
 \begin{pic}[scale=0.75]
                \node (0) at (-4.5, 0.75) {};
                \node (1) at (-3.5, 0.75) {};
                \node (2) at (1.75, 0.75) {};
                \node (3) at (2.75, 0.75) {};
                \node (4) at (-2, 0.5) {};
                \node (5) at (-1.25, 0.5) {};
                \node (6) at (4.25, 0.5) {};
                \node (7) at (5, 0.5) {};
                \node [style=whitedot] (8) at (-5.25, 0.25) {};
                \node [style=graydot] (9) at (1, 0.25) {};
                \node [style=whitedot] (10) at (6.75, 0.25) {};
                \node [style=graydot] (11) at (8.25, 0.25) {};
                \node [style=whitedot] (12) at (-4, 0) {};
                \node (13) at (-3, 0) {$=$};
                \node [style=graydot] (14) at (2.25, 0) {};
                \node (15) at (3.25, 0) {$=$};
                \node (16) at (7.5, 0) {$=$};
                \node [style=graydot] (17) at (-5.25, -0.25) {};
                \node [style=whitedot] (18) at (1, -0.25) {};
                \node [style=graydot] (19) at (6.75, -0.25) {};
                \node [style=whitedot] (20) at (8.25, -0.25) {};
                \node [style=graydot] (21) at (-2, -0.5) {};
                \node [style=graydot] (22) at (-1.25, -0.5) {};
                \node [style=whitedot] (23) at (4.25, -0.5) {};
                \node [style=whitedot] (24) at (5, -0.5) {};
                \node [style=graydot] (25) at (-4, -0.75) {};
                \node [style=whitedot] (26) at (2.25, -0.75) {};

                \draw [bend left=15] (12) to (0.center);
                \draw (22) to (5.center);
                \draw [bend left=15] (14) to (2.center);
                \draw [bend right=15, looseness=0.75] (12) to (1.center);
                \draw (24) to (7.center);
                \draw (26) to (14);
                \draw (19) to (10);
                \draw (18) to (9);
                \draw (21) to (4.center);
                \draw (20) to (11);
                \draw (25) to (12);
                \draw (23) to (6.center);
                \draw (17) to (8);
                \draw [bend right=15, looseness=0.75] (14) to (3.center);
\end{pic}

\end{equation}
\end{defn}

\begin{proposition}[{\cite[Prop.~3]{coecke2015generalised}}]
\label{prop:cohere}
In \cat{FHilb} if we are given two are self-adjoint operators corresponding to complementary classical structures, then we can always construct a pair of coherent classical structures with the same classical points.
\end{proposition}
\noindent This means that in the oGCT for quantum computation, we can take complementary classical structures to be coherent without loss of generality.

The terminology for the antipode comes from the fact that complementarity requires the classical structures to almost form a Hopf algebra using this antipode. Some complementary classical structures do indeed form Hopf algebras and these ones are called strongly complementary. They will be discussed in Section~\ref{sec:strcomplFT} in detail.

\begin{example}
\label{ex:cnot}
Complementary observables allow us to construct a generalized from of the controlled-not gate in any oGCT. In \cat{FHilb} we can choose classical structures on the two-dimensional Hilbert space that correspond with the usual Z and X observables:  \begin{align*}
    \tinycomult[whitedot] : &
      \begin{array}{rcl}
        \ket{0} & \mapsto & \ket{00} \\
        \ket{1} & \mapsto & \ket{11}
      \end{array}
&
    \tinymult[graydot] : &
      \begin{array}{rcl}
        \ket{+} & \mapsto & \ket{++} \\
        \ket{-} & \mapsto & \ket{--}
      \end{array}
\\ \\
    \tinycounit[whitedot] : &
      \begin{array}{rcl}
        \ket{0} & \mapsto & 1 \\
        \ket{1} & \mapsto & 1
      \end{array}
&
    \tinycounit[graydot] : &
      \begin{array}{rcl}
        \ket{+} & \mapsto & 1 \\
        \ket{-} & \mapsto & 1
      \end{array}
\\ \\
K_{\dotonly{whitedot}} &= \{\ket{0},\ket{1}\}
& 
K_{\dotonly{graydot}} &= \{\ket{+},\ket{-}\}
\end{align*}

These classical structures are coherent and complementary and each have $\mathbb{Z}_2$ as their phase groups, which we will write as $\{0,\pi\}$ where addition is modulo $2\pi$.  By Lemma~\ref{lem:phaseunbiased}, we know that the classical points of the $\dotonly{whitedot}$-structure are phases for the $\dotonly{graydot}$-structure. In particular we have:
\begin{equation}
\label{ex:cnottypes}
\begin{pic}
\node [style=state] (0) at (1, -0) {$0$};
\node (1) at (1, 1) {};
\draw (0) to (1);
\end{pic}
\;=\;
\begin{pic}
\node [style=graydot] (0) at (1, -0) {};
\node (1) at (1, 1) {};
\draw (0) to (1);
\end{pic}
\qquad\qquad\qquad
\begin{pic}
\node [style=state] (0) at (1, -0) {$1$};
\node (1) at (1, 1) {};
\draw (0) to (1);
\end{pic}
\;=\;
\begin{pic}
\node [style=graydot] (0) at (1, -0) {$\pi$};
\node (1) at (1, 1) {};
\draw (0) to (1);
\end{pic}
\qquad\qquad\qquad
\begin{pic}[scale=0.8]
\node [style=graydot] (0) at (1, -0) {$\pi$};
\node (1) at (1, 1) {};
\node (2) at (1, -1) {};
\draw (0) to (1);
\draw (0) to (2);
\end{pic}
\;=\;
X\mbox{-gate}
\end{equation}
These structures can be used to define the controlled-not $\mbox{CNOT}:H_2\otimes H_2\to H_2\otimes H_2$ whose diagram is:
\begin{align}
\begin{pic}[dotpic]
                \node [style=graydot] (0) at (1, -0) {};
                \node [style=whitedot] (1) at (-1, -0) {};
                \node (2) at (1, -1) {};
                \node (3) at (-1, -1) {};
                \node (4) at (-1, 1) {};
                \node (5) at (1, 1) {};
                \draw (2.center) to (0);
                \draw (0) to (1);
                \draw (3.center) to (1);
                \draw (1) to (4.center);
                \draw (0) to (5.center);
\end{pic}
\end{align}
\noindent where the right system acts as the control. We can verify that this behaves as usual for qubits using Z (the $\dotonly{whitedot}$-classical structure as their computational basis).
\begin{align}
\begin{pic}[dotpic]
                \node [style=whitedot] (0) at (1, -0) {};
                \node [style=graydot] (1) at (-1, -0) {};
                \node [point] (2) at (1, -1) {$0$};
                \node (3) at (-1, -1) {};
                \node (4) at (-1, 1) {};
                \node (5) at (1, 1) {};
                \draw (2.north) to (0);
                \draw (0) to (1);
                \draw (3.center) to (1);
                \draw (1) to (4.center);
                \draw (0) to (5.center);
\end{pic}
\;&\stackrel{\eqref{ex:cnottypes}}{=}\;
\begin{pic}[dotpic]
                \node [style=whitedot] (0) at (1, -0) {};
                \node [style=graydot] (1) at (-1, -0) {};
                \node [graydot] (2) at (1, -1) {};
                \node (3) at (-1, -1) {};
                \node (4) at (-1, 1) {};
                \node (5) at (1, 1) {};
                \draw (2.north) to (0);
                \draw (0) to (1);
                \draw (3.center) to (1);
                \draw (1) to (4.center);
                \draw (0) to (5.center);
\end{pic}
\;\stackrel{\eqref{eq:coherence}}{=}\;
\begin{pic}[dotpic]
                \node [style=graydot] (0) at (0, -0) {};
                \node [style=graydot] (1) at (-1, -0) {};
                \node [graydot] (2) at (1, 0) {};
                \node (3) at (-1, -1) {};
                \node (4) at (-1, 1) {};
                \node (5) at (1, 1) {};
                \draw (2.north) to (5.center);
                \draw (0) to (1);
                \draw (3.center) to (1);
                \draw (1) to (4.center);
\end{pic}
\;\stackrel{\eqref{eq:decspider}}{=}\;
\begin{pic}[dotpic]
                \node [point] (2) at (1, 0) {$0$};
                \node (3) at (-1, -1) {};
                \node (4) at (-1, 1) {};
                \node (5) at (1, 1) {};
                \draw (2.north) to (5.center);
                \draw (3.center) to (4.center);
\end{pic}
\\
\begin{pic}[dotpic]
                \node [style=whitedot] (0) at (1, -0) {};
                \node [style=graydot] (1) at (-1, -0) {};
                \node [point] (2) at (1, -1) {$1$};
                \node (3) at (-1, -1) {};
                \node (4) at (-1, 1) {};
                \node (5) at (1, 1) {};
                \draw (2.north) to (0);
                \draw (0) to (1);
                \draw (3.center) to (1);
                \draw (1) to (4.center);
                \draw (0) to (5.center);
\end{pic}
\;&\stackrel{\eqref{ex:cnottypes}}{=}\;
\begin{pic}[dotpic]
                \node [style=whitedot] (0) at (1, -0) {};
                \node [style=graydot] (1) at (-1, -0) {};
                \node [graydot] (2) at (1, -1) {$\pi$};
                \node (3) at (-1, -1) {};
                \node (4) at (-1, 1) {};
                \node (5) at (1, 1) {};
                \draw (2.north) to (0);
                \draw (0) to (1);
                \draw (3.center) to (1);
                \draw (1) to (4.center);
                \draw (0) to (5.center);
\end{pic}
\,\stackrel{\eqref{eq:coherence}}{=}\,
\begin{pic}[dotpic]
                \node [style=graydot] (0) at (0, -0) {$\pi$};
                \node [style=graydot] (1) at (-1, -0) {};
                \node [graydot] (2) at (1.5, 0) {$\pi$};
                \node (3) at (-1, -1) {};
                \node (4) at (-1, 1) {};
                \node (5) at (1.5, 1) {};
                \draw (2.north) to (5.center);
                \draw (0) to (1);
                \draw (3.center) to (1);
                \draw (1) to (4.center);
\end{pic}
\;\stackrel{\eqref{eq:decspider}}{=}\;
\begin{pic}[dotpic]
                \node [style=graydot] (1) at (-1, -0) {$\pi$};
                \node [point] (2) at (1.5, 0) {$1$};
                \node (3) at (-1, -1) {};
                \node (4) at (-1, 1) {};
                \node (5) at (1.5, 1) {};
                \draw (2.north) to (5.center);
                \draw (3.center) to (1);
                \draw (1) to (4.center);
\end{pic}
\end{align}
This shows that on computational basis elements, the control is left unchanged while a $X$-gate (a computational bit flip) is conditionally applied.
\end{example}

This example construction motivates the following definition on any kind of system in any oGCT:
\begin{defn}
Given two two classical structures \whitecomonoid{A} and \graycomonoid{A} that are complementary and coherent, the generalized \emph{controlled-not} is the map:
\begin{equation}
\label{eq:generalizedcnot}
\sqrt{\ud(A)}\,\,
\begin{pic}[xscale=\tikzxscale, yscale=\tikzyscale]
\node (b) [graydot] at (0,0) {};
\node (w) [whitedot] at (1,1) {};
\draw (-0.75,2) to [out=down, in=left] (b.center);
\draw (b.center) to [out=right, in=left] (w.center);
\draw (w.center) to (1,2);
\draw (b.center) to (0,-1);
\draw (w.center) to [out=right, in=up] (1.75,-1);
\end{pic}
\end{equation}
where we have assumed that a scalar equal to the square root of the dimension exists.
\end{defn}
\noindent This scalar is needed to ensure the unitarity of the controlled-not gate, which we discuss in further details in Section \ref{sec:unitaryoracles}.

\section{Enriched oGCTs}
\label{sec:enrichedogcts}
Some oGCTs, such as \cat{FHilb} come equipped with linear structure on their processes. Abstractly these are enriched categories, i.e. categories where hom-sets are replaced with other objects.\footnote{A standard reference for enriched category theory is~\cite{kelly1982basic}.}

\begin{defn}
\label{def:enrichedcat}
Given a monoidal category $K$, a $K$-enriched category \cat{C} has objects $\objs{C}$ such that:
\begin{itemize}
\item For every pair of $A,B\in\objs{C}$ there is a hom-object:
\begin{align}
\cat{C}(A,B)\in \objs{K}.
\end{align}
\item For all objects $A,B,C\in\objs{C}$ composition is given by a morphism in $\arrs{K}$ of type:
\begin{align}
 $\circ_{A,B,C}:\cat{C}(B,C)\otimes\cat{C}(A,B)\to\cat{C}(C,A).
\end{align}
\end{itemize}
\end{defn}

Linear maps between Hilbert spaces themselves form a vector space, so \cat{FHilb} is in fact a $\cat{FVect}$-enriched category. Another way of saying this is that \cat{FHilb} is enriched in \cat{FVect}. As diagrams of a GCT are morphisms in a category, diagrams in an enriched category have additional operations. The \cat{FVect} enrichment of \cat{FHilb}, for example, allows us to sum diagrams, as in the following property. Here \cat{Mon} is the category of monoids and monoid homomorphisms.


\begin{defn}
\label{def:resid}
In a \cat{Mon}-enriched oGCT, a set of states $\{\ket{x}\}$ on a system $A$ forms a \emph{resolution of the identity} when:
\begin{equation}
\label{eq:ResolutionId}
\begin{tikzpicture}[node distance = 11mm]

\node (spacer) {};

\node (sum) {$\dfrac{1}{d(A)}\sum_x$};
\node (projCenter) [right of = sum, xshift = -1mm, yshift=1mm] {};
\node [point] (projTop) [above of = projCenter, yshift=-6mm] {x};
\node [copoint] (projBot) [below of = projCenter,yshift=+6mm] {x};

\node (projIn) [below of = projBot, yshift = 4mm] {};
\node (projOut) [above of = projTop, yshift = -4mm] {};

\begin{pgfonlayer}{background}
\draw[-] [out=90,in=270](projIn) to (projBot.south);
\draw[-] [out=90,in=270](projTop.north) to (projOut);
\end{pgfonlayer}

\node (equals) [right of = projCenter, xshift = 0mm] {$=$};

\node (idCenter) [right of = equals, xshift = -3mm] {};

\node (idIn) [below of = idCenter] {};
\node (idOut) [above of = idCenter] {};;

\begin{pgfonlayer}{background}
\draw[-] [out=90,in=270](idIn) to (idOut);
\end{pgfonlayer}


\end{tikzpicture}

\end{equation}
\end{defn}

\noindent Where the sum operation comes from a monoid structure of the hom-object \cat{C}(A,A). In \cat{FHilb} more specifically, this the addition is inherited  from the vector space enrichment. In fact, in Hilbert spaces all the classical states of any classical structure form a resolution of the identity. Further, we are able to link the classical structure maps to classical states in the following ways~\cite{coecke2015generalised}:
\begin{align}
\label{eq:spidersums}
\begin{pic}[xscale=\tikzxscale, yscale=\tikzyscale]
                \node (0) at (-1.75, 1) {};
                \node [style={graydot}] (1) at (-1.75, 0.25) {};
                \node (2) at (0, -0) {$=$};
                \node (3) at (-2.5, -0.5) {};
                \node (4) at (-1, -0.5) {};
                \node [style={gray copoint}] (5) at (2.5, -0.5) {$i$};
                \node [style={gray copoint}] (6) at (3.5, -0.5) {$i$};
                \node [style={gray point}] (7) at (3, 0.5) {$i$};
                \node (8) at (3, 1.25) {};
                \node (9) at (2.5, -1.25) {};
                \node (10) at (3.5, -1.25) {};
                \node [style={graydot}] (11) at (6.5, -0) {};
                \node (12) at (6.5, -1) {};
                \node (13) at (5.75, 1) {};
                \node (14) at (10.75, 1.5) {};
                \node (15) at (7.25, 1) {};
                \node [style={gray point}] (16) at (10.75, 0.75) {$i$};
                \node [style={gray point}] (17) at (11.75, 0.75) {$i$};
                \node (18) at (8, -0) {$=$};
                \node (19) at (11.75, 1.5) {};
                \node (20) at (11.25, -1.25) {};
                \node [style={gray copoint}] (21) at (11.25, -0.5) {$i$};
                \node (22) at (-1.5, -3.5) {};
                \node [style={graydot}] (23) at (-1.5, -4.25) {};
                \node (24) at (0, -4) {$=$};
                \node [style=none, yshift={-0.8 mm}] (25) at (1.25, -4) {$\underset{i}{\sum}$};
                \node (26) at (2.75, -3.25) {};
                \node [style={gray point}] (27) at (2.75, -4.25) {$i$};
                \node (28) at (6.5, -4.5) {};
                \node [style={graydot}] (29) at (6.5, -3.75) {};
                \node (30) at (8, -4) {$=$};
                \node [style={gray copoint}] (31) at (11.25, -3.75) {$i$};
                \node (32) at (11.25, -4.75) {};
                \node [style=none, yshift={-0.8 mm}] (33) at (9.5, -4) {$\underset{i}{\sum}$};
                \node [style=none, yshift={-0.8 mm}] (34) at (1.25, -0) {$\underset{i}{\sum}$};
                \node [style=none, yshift={-0.8 mm}] (35) at (9.5, -0) {$\underset{i}{\sum}$};

                \draw [in=-165, out=90, looseness=1.00] (3.center) to (1);
                \draw [in=-15, out=90, looseness=1.00] (4.center) to (1);
                \draw (1) to (0.center);
                \draw (7) to (8.center);
                \draw (10.center) to (6);
                \draw (9.center) to (5);
                \draw [in=-90, out=180, looseness=1.00] (11) to (13.center);
                \draw [in=-90, out=0, looseness=1.00] (11) to (15.center);
                \draw (12.center) to (11);
                \draw (20.center) to (21);
                \draw (17) to (19.center);
                \draw (16) to (14.center);
                \draw (23) to (22.center);
                \draw (27) to (26.center);
                \draw (28.center) to (29);
                \draw (32.center) to (31);
\end{pic}

\end{align}
Arbitrary spiders can then be written as:
\begin{align}
\begin{pic}[xscale=\tikzxscale, yscale=\tikzyscale]
                \node (0) at (-3.5, -1.25) {};
                \node (1) at (-0.75, -0) {$=$};
                \node (2) at (-3.25, 1.25) {};
                \node [style={graydot}] (3) at (-2.5, -0) {};
                \node (4) at (-1.5, -1.25) {};
                \node (5) at (1.75, -1.5) {};
                \node [yshift={-0.8 mm}] (6) at (0.5, -0) {$\underset{i}{\sum}$};
                \node [style={gray point}] (7) at (2, 0.75) {$i$};
                \node [style={gray copoint}] (8) at (1.75, -0.75) {$i$};
                \node [style={gray copoint}] (9) at (3.75, -0.75) {$i$};
                \node (10) at (3.75, -1.5) {};
                \node (11) at (2, 1.5) {};
                \node (12) at (-1.75, 1.25) {};
                \node (13) at (-2.5, 0.75) {...};
                \node (14) at (-2.5, -0.75) {...};
                \node [style={gray point}] (15) at (3.5, 0.75) {$i$};
                \node (16) at (3.5, 1.5) {};
                \node (17) at (2.75, -0.5) {...};
                \node (18) at (2.75, 0.5) {...};

                \draw [in=-165, out=90, looseness=1.00] (0.center) to (3);
                \draw [in=-15, out=90, looseness=1.00] (4.center) to (3);
                \draw [in=-90, out=165, looseness=1.00] (3) to (2.center);
                \draw (7) to (11.center);
                \draw (10.center) to (9);
                \draw (5.center) to (8);
                \draw [in=-90, out=15, looseness=1.00] (3) to (12.center);
                \draw (15) to (16.center);
\end{pic}

\end{align}

\begin{remark}
We emphasize that this connection between classical states and classical structures does not hold in general oGCTs, but merely serves to further motivate abstract constructions that generalize from \cat{FHilb}. We make use of this technique in the following section on measurements.
\end{remark}

\section{Measurements}
\label{sec:measurements}

We have already introduced a notion of post-selected measurement in Definition~\ref{def:bornrule}. This section uses Selinger's CPM (completely positive map) construction~\cite{selinger2007dagger}, to present measurement as a process that outputs classical information~\cite{coecke2012strong}. We will need to be more precise about systems and their duals here so will be more consistent about using wires decorated with the proper arrows. 

Using caps and cups, we can construct the unique ``name" of a process $\rho$ as:
\begin{equation}
\label{eq:cj}
\begin{pic}[dotpic]
                \node  (0) at (3.25, 1) {};
                \node  (1) at (-2, 1) {};
                \node  (2) at (-0.5, 1) {};
                \node [style={morphism}] (3) at (3.25, -0) {$\rho$};
                \node  (4) at (1, -0) {$:=$};
                \node  (5) at (-2, -0) {};
                \node  (6) at (-0.5, -0) {};
                \node  (7) at (3.25, -0.75) {};
                \node [style={wide point}, yshift={0.1 mm}] (8) at (-1.25, -0.5) {$\rho$};
                \node  (9) at (1.75, 1) {};
                \node  (10) at (1.75, -0.75) {};
                \node [style={morphism}] (11) at (-6.75, -0) {$\rho$};
                \node  (12) at (-6.75, 1.25) {};
                \node  (13) at (-6.75, -1.25) {};
                \node  (14) at (-5, -0) {$\leftrightarrow$};
                \draw [style=diredge] (3.north) to (0.center);
                \draw [style=diredge] (7.center) to (3.south);
                \draw [style=diredge] (1.center) to (5.center);
                \draw [style=diredge] (6.center) to (2.center);
                \draw (9.center) to (10.center);
                \draw [in=-90, out=-90, looseness=1.50] (10.center) to (7.center);
                \draw [style=diredge] (11.north) to (12.center);
                \draw [style=diredge] (13.center) to (11.south);
\end{pic}

\end{equation}
This is the usual Choi-Jamiolkowski  isomorphism for map-state duality in quantum information. The ``doubling" that occurs here, provides a natural way to represent measurements as completely positive maps. Suppose we wish to measure a system with respect to a classical structure $\graycomonoid{A}$,
whose classical states form an orthonormal basis $\{ \ket{x_i} \}$. The probability
of getting the $i$-th measurement outcome is computed using the Born
rule: 
\begin{equation}
 \mbox{Prob}(i, \rho) = \mbox{Tr}(\ketbra{x_i}{x_i} \rho) 
\end{equation}

We can write this probability distribution as a vector in the basis $\{ \ket{x_i} \}$. That is, a vector whose $i$-th entry is the probability of the $i$-th outcome:
\begin{equation}
M_{\dotonly{graydot}}(\rho) = \sum 
\mbox{Tr}(\ketbra{x_i}{x_i} \rho) \ket{x_i}
\end{equation}

So, $M$ defines a linear map from density matrices to probability distributions. Expanding this graphically, we have:
\begin{equation}
\sum_i \mbox{Tr}\left( \begin{pic}[dotpic]
                \node [style={gray copoint}] (0) at (0, -0) {$i$};
                \node [style={gray point}] (1) at (0, 1.25) {$i$};
                \node [style={morphism}] (2) at (0, -1.25) {$\rho$};
                \node (3) at (0, -2.25) {};
                \node (4) at (0, 2.25) {};

                \draw [style=diredge] (2.north) to (0);
                \draw [style=diredge] (3.center) to (2.south);
                \draw [style=diredge] (1) to (4.center);
\end{pic}
 \right) 
\begin{pic}
\node [style={gray point}] (m) at (0, 1.25) {$i$};
\draw [diredge] (m) to (0,2.25)
\end{pic}
\;\stackrel{\eqref{eq:trace}}{=}\;
\begin{pic}[dotpic]
                \node [style={gray copoint}] (0) at (-8.25, 1) {$i$};
                \node  (1) at (6.5, 1.25) {};
                \node  (2) at (-7, 0.75) {};
                \node [style={graydot}] (3) at (6.5, 0.5000002) {};
                \node [style={square box}] (4) at (-8.25, -0.25) {$\rho$};
                \node  (5) at (-0.5, 0) {$=$};
                \node  (6) at (4.25, -0) {$=$};
                \node  (7) at (-10.75, -0.25) {$\underset{i}{\sum}$};
                \node  (8) at (-5, -0.25) {$\underset{i}{\sum}$};
                \node  (9) at (5.75, -0.2499996) {};
                \node  (10) at (7.25, -0.2499996) {};
                \node [style={gray point}] (11) at (-7, -0.75) {$i$};
                \node  (12) at (-8.25, -1.5) {};
                \node [style={wide point}, yshift={0.55 mm}] (13) at (6.5, -0.7499998) {$\rho$};
                \node [style={gray point}] (14) at (-8.25, 2.25) {$i$};
                \node  (15) at (-9, 3) {};
                \node  (16) at (-9, -2.25) {};
                \node  (17) at (-9.75, 2.25) {};
                \node  (18) at (-9.75, -1.5) {};
                \node [style={gray copoint}] (19) at (-2.25, 1) {$i$};
                \node [style={gray point}] (20) at (-2.75, 2) {$i$};
                \node  (21) at (-3.25, 2.75) {};
                \node [style={square box}] (22) at (-2.25, -0.25) {$\rho$};
                \node  (23) at (-2.25, -1.5) {};
                \node  (24) at (-3.75, 2) {};
                \node  (25) at (-3.75, -1.5) {};
                \node  (26) at (-3, -2.25) {};
                \node  (27) at (-6, -0) {$=$};
                \node [style={gray point}] (28) at (-1.75, 2) {$i$};
                \node  (29) at (-1.75, 3) {};
                \node [style={graydot}] (30) at (2.25, 1.25) {};
                \node  (31) at (1.250001, 2.5) {};
                \node [style={square box}] (32) at (2.25, -0.2500002) {$\rho$};
                \node  (33) at (2.25, -1.5) {};
                \node  (34) at (0.7499998, 1.5) {};
                \node  (35) at (0.7500006, -1.5) {};
                \node  (36) at (1.5, -2.25) {}; 
                \node  (37) at (2.75, 2.25) {};

                \draw [style=diredge] (4) to (0);
                \draw [style=diredge, in=90, out=-165, looseness=1.00] (3) to (9.center);
                \draw [style=diredge] (12.center) to (4);
                \draw [style=diredge, in=-15, out=90, looseness=1.00] (10.center) to (3);
                \draw [style=diredge] (11) to (2.center);
                \draw [style=diredge] (3.center) to (1.center);
                \draw [in=90, out=180, looseness=1.00] (15.center) to (17.center);
                \draw [in=-90, out=0, looseness=1.00] (16.center) to (12.center);
                \draw [in=0, out=90, looseness=1.00] (14) to (15.center);
                \draw [style=diredge, in=90, out=-90, looseness=1.00] (17.center) to (18.center);
                \draw [in=180, out=-90, looseness=1.00] (18.center) to (16.center);
                \draw [style=diredge] (22) to (19);
                \draw [style=diredge] (23.center) to (22);
                \draw [in=90, out=180, looseness=1.00] (21.center) to (24.center);
                \draw [in=-90, out=0, looseness=1.00] (26.center) to (23.center);
                \draw [in=0, out=90, looseness=1.00] (20) to (21.center);
                \draw [style=diredge, in=90, out=-90, looseness=1.00] (24.center) to (25.center);
                \draw [in=180, out=-90, looseness=1.00] (25.center) to (26.center);
                \draw [style=diredge] (28) to (29.center);
                \draw [style=diredge] (33.center) to (32);
                \draw [in=90, out=180, looseness=1.00] (31.center) to (34.center);
                \draw [in=-90, out=0, looseness=1.00] (36.center) to (33.center);
                \draw [in=0, out=180, looseness=1.00] (30) to (31.center);
                \draw [style=diredge, in=90, out=-90, looseness=1.00] (34.center) to (35.center);
                \draw [in=180, out=-90, looseness=1.00] (35.center) to (36.center);
                \draw [style=diredge] (32) to (30);
                \draw [style=diredge, in=-90, out=0, looseness=1.00] (30) to (37.center);
\end{tikzpicture}

\end{equation}

\begin{defn}[{\cite{coecke2012strong}}]
\label{def:meas2}
  For a classical structure \graycomonoid{A}, a measurement in that classical structure is defined as
  the following map: 
  \begin{equation}
  \begin{pic}[yscale=0.8]
                \node (0) at (0.75, 0.75) {};
                \node (1) at (-1.25, 0) {$m_{\dotonly{graydot}} :=$ };
                \node [style=graydot] (2) at (0.75, 0) {};
                \node (3) at (0.25, -0.75) {};
                \node (4) at (1.25, -0.75) {};
                \draw [style=diredge, bend right] (4.center) to (2);
                \draw [style=diredge, bend right] (2) to (3.center);
                \draw [style=diredge] (2) to (0.center);
\end{pic}

  \end{equation}
\end{defn}

\begin{example}
\todo{EXAMPLE on qubits}
\end{example}

\section{Summary of oGCTs}

This chapter has introduced the main framework that is used in this thesis. This framework performs two functions:
\begin{enumerate}
\item oGCTs lift the structure of quantum computation into the general setting of operational generalised compositional theories. This allows the study of protocols like teleportation, state transfer~\cite{cqm-notes}, quantum secret sharing (Section~\ref{section_QSS}), quantum bit commitment~\cite{katriel-commitment}, Mermin non-locality tests (Chapter~\ref{chap:mermin}), and other protocols (Chapters~\ref{chap:qalg} and~\ref{chap:qDisCo}) to be handled at an abstract level and studied in different oGCTs.
\item The categorical diagrams that accompany oGCT's present a more powerful quantum circuit language. This is one that includes bases, classical information, complementarity, and phases explicitly and gives rules for manipulating these structures within the diagrams themselves. We present a summarizing table of these structures at the end of this chapter.
\end{enumerate}
Both the abstraction and diagrammatic formalism provide powerful tools throughout the results in this thesis.

%\newpage

\begin{figure}[b]
\caption{A summary of the diagrammatic elements for oGCTs above and beyond quantum circuit diagrams.}
{\renewcommand{\arraystretch}{2}\small
\begin{tabulary}{\linewidth}{|p{7cm}|c|}\hline
The \textbf{dagger} $\dagger:\cat{C}\to\cat{C}$ generalizes the adjoint and vertically flips the whole diagram. See Definition \ref{defn:dagger}.
& \begin{aligned}
\begin{tikzpicture}[xscale=\tikzxscale, yscale=\tikzyscale]

\node at (0.5, -2) {$A$};
\node (0) at (0, -2) {};
\node (1) [brnode] at (0, 0) {$f$};
\node (2) at (0, 2) {};
\node at (0.5, 2) {$B$};

\draw (0.center) to (1.south);
\draw (1.north) to (2.center);

\end{tikzpicture}
\end{aligned}
\qquad \mapsto \qquad 
\begin{aligned}
\begin{tikzpicture}[xscale=\tikzxscale, yscale=\tikzyscale]

\node at (0.5, -2) {$B$};
\node (0) at (0, -2) {};
\node (1) [trnode] at (0, 0) {$f$};
\node (2) at (0, 2) {};
\node at (0.5, 2) {$A$};

\draw (0.center) to (1.south);
\draw (1.north) to (2.center);

\end{tikzpicture}
\end{aligned}
\::=\:
\begin{aligned}
\begin{tikzpicture}[xscale=\tikzxscale, yscale=\tikzyscale]

\node at (0.5, -2) {$B$};
\node (0) at (0, -2) {};
\node (1) [morphism] at (0, 0) {$\dag{f}$};
\node (2) at (0, 2) {};
\node at (0.5, 2) {$A$};

\draw (0.center) to (1.south);
\draw (1.north) to (2.center);

\end{tikzpicture}
\end{aligned}
 \\\hline
\textbf{Scalars} $s:I\to I$. These float freely in diagrams. See Definition \ref{defn:scalar}.
& \begin{pic}[xscale={\tikzxscale}, yscale={\tikzyscale}]
\node [whitedot] (0) at (0, -1) {$s$};
\node [none] (0) at (0, -0) {};
\end{pic} \\\hline
\textbf{States} and \textbf{effects}. Definition~\ref{def:effect}
& \left(
\begin{aligned}
\begin{tikzpicture}[xscale=\tikzxscale, yscale=0.5*\tikzyscale]
\node (1) [state] at (0, -0.75) {$\psi$};
\node (2) at (0, 0.75) {};
\draw (1.center) to (2.center);
\end{tikzpicture}
\end{aligned}\right)^{\dagger}
=
\begin{aligned}
\begin{tikzpicture}[xscale=\tikzxscale, yscale=0.5*\tikzyscale]
\node (1) at (0, -0.75) {};
\node (2) [state, hflip] at (0, 0.75) {$\psi$};
\draw (1.center) to (2.center);
\end{tikzpicture}
\end{aligned} \\\hline
\textbf{Post-selected measurement}.\newline Section~\ref{sec:bornrule}.
& \langle\phi|\psi\rangle =
\begin{aligned}
\begin{tikzpicture}[xscale=\tikzxscale, yscale=0.5*\tikzyscale]
\node (1) [state] at (0, -0.75) {$\psi$};
\node (2) [state, hflip] at (0, 0.75) {$\phi$};
\draw (1.center) to (2.center);
\end{tikzpicture}
\end{aligned} \\\hline
\textbf{Bell states} and measurements, i.e. cups and caps. Definition~\ref{def:dual}.
& e_A = \,
\begin{pic}[xscale={\tikzxscale}, yscale={\tikzyscale}]
\node (0) at (-1, -0) {};
\node (1) at (1, -0) {};
\node (2) at (0, -1) {};
\draw [bend right=45, looseness=1.00] (0.center) to (2.center);
\draw [bend right=45, looseness=1.00, ->] (2.center) to (1.center);
\end{pic}
\qquad \qquad 
d_A = \,
\begin{pic}[xscale={\tikzxscale}, yscale={-\tikzyscale}]
\node (0) at (-1, -0) {};
\node (1) at (1, -0) {};
\node (2) at (0, -1) {};
\draw [bend right=45, looseness=1.00] (0.center) to (2.center);
\draw [bend right=45, looseness=1.00, ->] (2.center) to (1.center);
\end{pic}
 \\\hline
\textbf{Duals}. See \eqref{eq:dualmorph}.
& \begin{pic}[xscale={\tikzxscale}, yscale={2*\tikzyscale}]
\node (0) at (0,0) {};
\node (1) at (0,2) {};
\node [morphism] (2) at (0,1) {$f^*$};
\draw [reverse arrow=.5] (0) to (2.south);
\draw [reverse arrow=.5] (2.north) to (1);
\end{pic}
\;=\;
\begin{pic}[xscale={\tikzxscale}, yscale=1.75*\tikzyscale]
                \node  (0) at (-1, -0) {};
                \node  (1) at (1, -0) {};
                \node  (2) at (0, -1) {};
                \node  (3) at (2, 1) {};
                \node  (4) at (1, -0) {};
                \node  (5) at (3, -0) {};
                \node  (6) at (3, -1) {};
                \node  (7) at (-1, 1) {};
                \node [style=morphism] (8) at (1, -0) {f};
                \draw [bend right=45, looseness=1.00] (0.center) to (2.center);
                \draw [bend right=45, looseness=1.00] (2.center) to (8.south);
                \draw [bend left=45, looseness=1.00] (8.north) to (3.center);
                \draw [bend left=45, looseness=1.00] (3.center) to (5.center);
                \draw [reverse arrow=.5] (6.center) to (5.center);
                \draw [reverse arrow=.5] (0.center) to (7.center);
\end{pic}
 \\\hline
\textbf{Classical structures} i.e. generalized observables (Def.~\ref{def:classicalstruct}). Their \textbf{classical states} (Def. \ref{def:copyables}) act as ``basis elements". 
&  \begin{pic}
        \node  (0) at (-2, 1) {};
        \node  (1) at (-0.5, 1) {};
        \node  (2) at (0, 1) {};
        \node  (3) at (1, 1) {};
        \node  (4) at (1.5, 1) {};
        \node  (5) at (-2, 0.75) {};
        \node  (6) at (-1.5, 0.75) {};
        \node  (7) at (-0.5, 0.75) {};
        \node [style=whitedot] (8) at (1.25, 0.75) {};
        \node  (9) at (-1, 0.5) {...};
        \node  (10) at (1, 0.5) {\small \rotatebox[origin=c]{45}{...}};
        \node [style=whitedot] (11) at (0.75, 0.25) {};
        \node [style=whitedot] (12) at (-1.25, 0) {};
        \node [anchor=east] (13) at (0, 0) {$:=$};
        \node [style=whitedot] (14) at (0.75, -0.25) {};
        \node  (15) at (-1, -0.5) {...};
        \node  (16) at (1, -0.5) {\small \rotatebox[origin=c]{-45}{...}};
        \node  (17) at (-2, -0.75) {};
        \node  (18) at (-1.5, -0.75) {};
        \node  (19) at (-0.5, -0.75) {};
        \node [style=whitedot] (20) at (1.25, -0.75) {};
        \node  (21) at (-2, -1) {};
        \node  (22) at (-0.5, -1) {};
        \node  (23) at (0, -1) {};
        \node  (24) at (1, -1) {};
        \node  (25) at (1.5, -1) {};

        \draw [bend left=15] (18.center) to (12);
        \draw (14) to (11);
        \draw (11) to (2.center);
        \draw [style=small braceedge] (22.center) to node[wire label, inner sep=3 pt]{$n$} (21.center);
        \draw (8) to (4.center);
        \draw (23.center) to (14);
        \draw (24.center) to (20);
        \draw (8) to (3.center);
        \draw [bend left=15] (12) to (5.center);
        \draw [bend right=15] (19.center) to (12);
        \draw [bend left=15] (17.center) to (12);
        \draw [bend right=15] (12) to (7.center);
        \draw (25.center) to (20);
        \draw [bend left=15] (12) to (6.center);
        \draw [style=small braceedge] (0.center) to node[wire label, inner sep=3 pt]{$m$} (1.center);
\end{pic}
 \\\hline
\textbf{Dimension} of a system. Def.~\ref{def:dimension}
& d(A) \;= \;
\begin{pic}[xscale={\tikzxscale}, yscale={0.5*\tikzyscale}]
\node (t) [whitedot] at (0,1) {};
\node (b) [whitedot] at (0,-1) {};
\draw (t) to (b.center);
\end{pic} \\\hline
\textbf{Phases}: A group of states for each classical structure. Section~\ref{sec:phases}.
& \begin{pic}[font=\footnotesize, scale=1.5]
                \node  (0) at (-2.5, 0.75) {};
                \node  (1) at (-2, 0.75) {\raisebox{-2mm}{...}};
                \node  (2) at (-1.5, 0.75) {};
                \node  (3) at (-1, 0.75) {};
                \node  (4) at (-0.5, 0.75) {\raisebox{-2mm}{...}};
                \node  (5) at (0, 0.75) {};
                \node  (6) at (1, 0.75) {};
                \node  (7) at (1.5, 0.75) {\raisebox{-2mm}{...}};
                \node  (8) at (2, 0.75) {};
                \node  (9) at (-1, 0.5) {};
                \node  (10) at (0, 0.5) {};
                \node [style=whitedot] (11) at (-1.75, 0.25) {$\alpha$};
                \node  (12) at (-1.25, 0) {\rotatebox[origin=c]{45}{...}};
                \node  (13) at (0.5, 0) {$:=$};
                \node [style=whitedot] (14) at (1.5, 0) {$\!\alpha\!+\!\beta\!$};
                \node [style=whitedot] (15) at (-0.75, -0.25) {$\beta$};
                \node  (16) at (-2.5, -0.5) {};
                \node  (17) at (-1.5, -0.5) {};
                \node  (18) at (-2.5, -0.75) {};
                \node  (19) at (-2, -0.75) {\raisebox{2mm}{...}};
                \node  (20) at (-1.5, -0.75) {};
                \node  (21) at (-1, -0.75) {};
                \node  (22) at (-0.5, -0.75) {\raisebox{2mm}{...}};
                \node  (23) at (0, -0.75) {};
                \node  (24) at (1, -0.75) {};
                \node  (25) at (1.5, -0.75) {\raisebox{2mm}{...}};
                \node  (26) at (2, -0.75) {};
                \draw [in=210, out=90, looseness=1.25] (16.center) to (11);
                \draw [in=-60, out=180] (15) to (11);
                \draw [in=-15, out=120] (15) to (11);
                \draw [bend right=15] (11) to (2.center);
                \draw [in=-90, out=45] (15) to (10.center);
                \draw [in=270, out=105] (15) to (9.center);
                \draw [bend right=15] (23.center) to (15);
                \draw [bend left=15] (24.center) to (14);
                \draw [in=-75, out=90, looseness=1.25] (17.center) to (11);
                \draw (20.center) to (17.center);
                \draw [bend right=15] (14) to (8.center);
                \draw [bend left=15] (21.center) to (15);
                \draw [bend right=15] (26.center) to (14);
                \draw (9.center) to (3.center);
                \draw [bend left=15] (14) to (6.center);
                \draw (18.center) to (16.center);
                \draw (10.center) to (5.center);
                \draw [bend left=15] (11) to (0.center);
\end{pic}
 \\\hline
\textbf{Complementarity} between classical structures. Definition~\ref{def:complementarity}.
& \begin{pic}[scale=0.75]
                \node (0) at (-2, 1.75) {};
                \node (1) at (1.25, 1.5) {};
                \node [style=graydot] (2) at (-2, 1) {};
                \node [style=graydot] (3) at (1.25, 0.5) {};
                \node [style=antipode] (4) at (-1.25, 0) {$S$};
                \node (5) at (0, 0) {$=$};
                \node [style=whitedot] (6) at (1.25, -0.5) {};
                \node [style=whitedot] (7) at (-2, -1) {};
                \node (8) at (-2, -1.75) {};
                \node (9) at (1.25, -1.5) {};
                \node [style=whitedot] (10) at (-3.5, 0.5) {};
                \node [style=whitedot] (11) at (-3.5, -0.5) {};

                \draw (8.center) to (7);
                \draw [in=-90, out=15, looseness=1.00] (7) to (4);
                \draw (9.center) to (6);
                \draw [in=-165, out=165, looseness=1.00] (7) to (2);
                \draw (3) to (1.center);
                \draw (2) to (0.center);
                \draw [in=-15, out=90, looseness=1.00] (4) to (2);
                \draw (10) to (11);
\end{pic}
 \\\hline
\textbf{Measurement} by a classical structure. Definition~\ref{def:meas2}.
& \begin{pic}[yscale=0.8]
                \node (0) at (0.75, 0.75) {};
                \node (1) at (-1.25, 0) {$m_{\dotonly{graydot}} :=$ };
                \node [style=graydot] (2) at (0.75, 0) {};
                \node (3) at (0.25, -0.75) {};
                \node (4) at (1.25, -0.75) {};
                \draw [style=diredge, bend right] (4.center) to (2);
                \draw [style=diredge, bend right] (2) to (3.center);
                \draw [style=diredge] (2) to (0.center);
\end{pic}
 \\\hline
\end{tabulary}
}

\end{figure}



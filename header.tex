% Header adapted by Will Zeng in Spring 2015 from 
% contributions of Bob Coecke, Aleks Kissinger, 
% Jamie Vicary, Stefano Gogioso, and Dan Marsden
%

% Packages
\usepackage{docmute}
%\usepackage{a4wide}
%\usepackage[hyperindex]{hyperref}
\usepackage{enumerate,multirow}
\usepackage{amsmath}
\usepackage{amsthm,amsfonts,amssymb,mathtools,xspace}
\usepackage{tikz}
\usetikzlibrary{decorations.pathreplacing}
\usetikzlibrary{decorations.markings}
\usetikzlibrary{calc}
\usetikzlibrary{shapes.geometric}
\usepackage{etoolbox}
\usepackage{datetime}
\usepackage{tabulary}
\usepackage{array}
%\usepackage[usenames,dvipsnames]{xcolor}
\usepackage{subcaption}
\usepackage{bbm}
\usepackage{listings}
\lstset{language=Mathematica}
\lstset{basicstyle={\sffamily\footnotesize},
  numbers=left,
  numberstyle=\tiny\color{gray},
  numbersep=5pt,
  breaklines=true,
  captionpos={t},
  frame={lines},
  rulecolor=\color{black},
  framerule=0.5pt,
  columns=flexible,
  tabsize=2
}
%\usepackage{booktabs}
\usepackage{float}
%\usetikzlibrary{arrows,shapes,backgrounds,positioning}

% Graphics Path
\graphicspath{ {./images/} }

% Make hyphenation less likely
\pretolerance=2000

% Section Commands
% Subsection definition
\def\customsubsection{\subsection*}
% Chapter Abstracts
\newenvironment{chapabstract}{
    \begin{center}
      \bfseries Chapter Abstract
    \end{center}
}


% Theorem environments
\theoremstyle{plain}
\newtheorem{theorem}{Theorem}[section] % number by section
\newtheorem{axiom}[theorem]{Axiom}
\newtheorem{lemma}[theorem]{Lemma}
\newtheorem{corollary}[theorem]{Corollary}
\newtheorem{proposition}[theorem]{Proposition}
\newtheorem{conjecture}[theorem]{Conjecture}
\theoremstyle{definition}
\newtheorem{defn}[theorem]{Definition}
\newtheorem{examples}[theorem]{Examples}
\newtheorem{example}[theorem]{Example}
\newtheorem{remark}[theorem]{Remark}
%\newtheorem*{question}{Question}

% Picture environments
%\newenvironment{pic}[1][]
%{\begin{aligned}\begin{tikzpicture}[#1]\begin{pgfonlayer}{background}\begin{scope}}
%{\end{scope}\end{pgfonlayer}\end{tikzpicture}\end{aligned}}
\newenvironment{pic}[1][]
{\begin{aligned}\begin{tikzpicture}[#1]}
{\end{tikzpicture}\end{aligned}}
\newcommand{\edges}[1][]%
{%\end{scope}\end{pgfonlayer}\begin{pgfonlayer}{foreground}\begin{scope}[#1]
}

\makeatletter
\def\calign@preamble{%
   &\hfil\strut@
    \setboxz@h{\@lign$\m@th\displaystyle{##}$}%
    \ifmeasuring@\savefieldlength@\fi
    \set@field
    \hfil
    \tabskip\alignsep@
}
\let\cmeasure@\measure@
\patchcmd\cmeasure@{\divide\@tempcntb\tw@}{}{}{}
\patchcmd\cmeasure@{\divide\@tempcntb\tw@}{}{}{}
\patchcmd\cmeasure@{\ifodd\maxfields@
  \global\advance\maxfields@\@ne
  \fi}{}{}{}
\newenvironment{calign}
{%
  \let\align@preamble\calign@preamble
  \let\measure@\cmeasure@
  \align
}
{%
  \endalign
}
\makeatother

% Starred sections
\makeatletter
\newcommand{\starsection}[1]{%
  \removelastskip
  \vskip 3ex\goodbreak
  \refstepcounter{section}%
  \noindent
  \begingroup
  \leavevmode\Large\bfseries\raggedright
  \thesection*\quad%
  #1
  \par
  \endgroup
  \vskip 2ex\nobreak
  \addcontentsline{toc}{section}{%
    \protect\numberline{\thesection*}%
    #1}%
}
\makeatother
% Start at chapter 0
%\setcounter{chapter}{-1}
% Table of Contents depth
\setcounter{secnumdepth}{2}

% Matrices
\renewcommand\matrix[1]
{\hspace{-1.5pt}\left( \hspace{-2pt} \raisebox{-0pt}{$\begin{array}{@{\extracolsep{-5.5pt}\,}cccc}#1\end{array}$}  \hspace{0pt} \right)\hspace{-1.5pt}}
\newcommand\fmatrix[1]
{\hspace{-1.5pt}\left( \hspace{-2pt} \raisebox{-0pt}{\footnotesize$\begin{array}{@{\extracolsep{-5.5pt}\,}cccc}#1\end{array}$}  \hspace{0pt} \right)\hspace{-1.5pt}}
\newcommand\tinymatrix[1]
{\left( \hspace{-2pt} \renewcommand\thickspace{\kern2pt} \scriptstyle\begin{smallmatrix} #1 \end{smallmatrix} \hspace{-2pt} \right)}
\newcommand\foplus{\!\oplus\!}

% Useful commands
\newcommand\ignore[1]{}
\newcommand\superequals[1]{\ensuremath{\stackrel {\makebox[0pt]{\ensuremath{\scriptstyle #1}}}{=}}}
\newcommand\equalref[1]{\superequals{\eqref{#1}}}
\newcommand\equalreftwo[2]{\ensuremath{\stackrel{\makebox[0pt]{\ensuremath{\stackrel{\scriptstyle\eqref{#1}}{\scriptstyle\eqref{#2}}}}}{=}}}

% Layers
\pgfdeclarelayer{foreground}
\pgfdeclarelayer{background}
\pgfsetlayers{main,foreground,background}
\tikzset{smallbox/.style={draw, fill=white, minimum height=0.45cm, minimum width=0.45cm, inner sep=-100pt}}

% Styles
\makeatletter
\pgfkeys{%
  /tikz/on layer/.code={
    \pgfonlayer{#1}\begingroup
    \aftergroup\endpgfonlayer
    \aftergroup\endgroup
  },
  /tikz/node on layer/.code={
    \gdef\node@@on@layer{%
      \setbox\tikz@tempbox=\hbox\bgroup\pgfonlayer{#1}\unhbox\tikz@tempbox\endpgfonlayer\egroup}
    \aftergroup\node@on@layer
  },
  /tikz/end node on layer/.code={
    \endpgfonlayer\endgroup\endgroup
  }
}
\def\node@on@layer{\aftergroup\node@@on@layer}
\makeatother\def\thickness{0.7pt}
\tikzstyle{oldmorphism}=[minimum width=30pt, minimum height=16pt, draw, font=\small, inner sep=0pt, fill=white, line width=\thickness]
\tikzstyle{cross}=[preaction={draw=white, -, line width=10pt}]
%\tikzstyle{braid}=[preaction={draw=white, -, line width=8pt}, line width=\thickness]
\tikzstyle{braid}=[double=black, line width=3*\thickness, double distance=\thickness, white]
\tikzstyle{string}=[line width=\thickness]
\tikzstyle{scalar}=[circle, inner sep=0pt, minimum width=15pt, draw, line width=\thickness]
\tikzstyle{dot}=[circle, draw=black, fill=black!25, inner sep=.4ex, line width=\thickness, node on layer=foreground]
\tikzstyle{blackdot}=[circle, draw=black, fill=black!100, inner sep=.4ex, line width=\thickness, node on layer=foreground]
\tikzstyle{graydot}=[circle, draw=black, fill=gray!40!white, inner sep=.4ex, line width=\thickness, node on layer=foreground]
\tikzstyle{whitedot}=[circle, draw=black, fill=white, inner sep=0.4ex, line width=\thickness, node on layer=foreground]
\tikzstyle{altwhitedot}=[circle, draw=black, fill=-red, inner sep=0.4ex, line width=\thickness, node on layer=foreground]
\tikzstyle{altblackdot}=[circle, draw=black, fill=red!72.5375!blue, inner sep=0.4ex, line width=\thickness, node on layer=foreground]
\tikzstyle{mixedmorphism}=[morphism, minimum width=30pt, minimum height=16pt, draw, font=\small, inner sep=0pt, fill=white, line width=\thickness,rounded corners=1ex]
\tikzstyle{thick}=[line width=\thickness]
\tikzstyle{tiny}=[font=\tiny]

% Arrows
\tikzset{arrow/.style={decoration={
    markings,
    mark=at position #1 with \arrow{thickarrow}},
    postaction=decorate}
}
\tikzset{reverse arrow/.style={decoration={
    markings,
    mark=at position #1 with \arrow{reversethickarrow}},
    postaction=decorate}
}

% Keys
\newif\ifblack\pgfkeys{/tikz/black/.is if=black}
\newif\ifwedge\pgfkeys{/tikz/wedge/.is if=wedge}
\newif\ifvflip\pgfkeys{/tikz/vflip/.is if=vflip}
\newif\ifhflip\pgfkeys{/tikz/hflip/.is if=hflip}
\newif\ifhvflip\pgfkeys{/tikz/hvflip/.is if=hvflip}
\newif\ifconnectnw\pgfkeys{/tikz/connect nw/.is if=connectnw}
\newif\ifconnectne\pgfkeys{/tikz/connect ne/.is if=connectne}
\newif\ifconnectsw\pgfkeys{/tikz/connect sw/.is if=connectsw}
\newif\ifconnectse\pgfkeys{/tikz/connect se/.is if=connectse}
\newif\ifconnectn\pgfkeys{/tikz/connect n/.is if=connectn}
\newif\ifconnects\pgfkeys{/tikz/connect s/.is if=connects}
\newif\ifconnectnwf\pgfkeys{/tikz/connect nw >/.is if=connectnwf}
\newif\ifconnectnef\pgfkeys{/tikz/connect ne >/.is if=connectnef}
\newif\ifconnectswf\pgfkeys{/tikz/connect sw >/.is if=connectswf}
\newif\ifconnectsef\pgfkeys{/tikz/connect se >/.is if=connectsef}
\newif\ifconnectnf\pgfkeys{/tikz/connect n >/.is if=connectnf}
\newif\ifconnectsf\pgfkeys{/tikz/connect s >/.is if=connectsf}
\newif\ifconnectnwr\pgfkeys{/tikz/connect nw </.is if=connectnwr}
\newif\ifconnectner\pgfkeys{/tikz/connect ne </.is if=connectner}
\newif\ifconnectswr\pgfkeys{/tikz/connect sw </.is if=connectswr}
\newif\ifconnectser\pgfkeys{/tikz/connect se </.is if=connectser}
\newif\ifconnectnr\pgfkeys{/tikz/connect n </.is if=connectnr}
\newif\ifconnectsr\pgfkeys{/tikz/connect s </.is if=connectsr}
\tikzset{keylengthnw/.initial=\connectheight}
\tikzset{keylengthn/.initial =\connectheight}
\tikzset{keylengthne/.initial=\connectheight}
\tikzset{keylengthsw/.initial=\connectheight}
\tikzset{keylengths/.initial =\connectheight}
\tikzset{keylengthse/.initial=\connectheight}
\tikzset{connect nw length/.style={connect nw=true, keylengthnw={#1}}}
\tikzset{connect n length/.style ={connect n =true, keylengthn ={#1}}}
\tikzset{connect ne length/.style={connect ne=true, keylengthne={#1}}}
\tikzset{connect sw length/.style={connect sw=true, keylengthsw={#1}}}
\tikzset{connect s length/.style ={connect s =true, keylengths ={#1}}}
\tikzset{connect se length/.style={connect se=true, keylengthse={#1}}}
\tikzset{connect nw < length/.style={connect nw <=true, keylengthnw={#1}}}
\tikzset{connect n < length/.style ={connect n <=true,  keylengthn ={#1}}}
\tikzset{connect ne < length/.style={connect ne <=true, keylengthne={#1}}}
\tikzset{connect sw < length/.style={connect sw <=true, keylengthnw={#1}}}
\tikzset{connect s < length/.style ={connect s <=true,  keylengths ={#1}}}
\tikzset{connect se < length/.style={connect se <=true, keylengthse={#1}}}
\tikzset{connect nw > length/.style={connect nw >=true, keylengthnw={#1}}}
\tikzset{connect n > length/.style ={connect n >=true,  keylengthn ={#1}}}
\tikzset{connect ne > length/.style={connect ne >=true, keylengthne={#1}}}
\tikzset{connect sw > length/.style={connect sw >=true, keylengthsw={#1}}}
\tikzset{connect s > length/.style ={connect s >=true,  keylengths ={#1}}}
\tikzset{connect se > length/.style={connect se >=true, keylengthse={#1}}}

% Lengths
\newlength\morphismheight
\setlength\morphismheight{0.7cm}
\newlength\minimummorphismwidth
\setlength\minimummorphismwidth{0.2cm}
\newlength\stateheight
\setlength\stateheight{0.6cm}
\newlength\minimumstatewidth
\setlength\minimumstatewidth{0.89cm}
\newlength\connectheight
\setlength\connectheight{0.5cm}
\tikzset{width/.initial=\minimummorphismwidth}

% Custom arrowhead
\makeatletter
\pgfarrowsdeclare{thickarrow}{thickarrow}
{
  \pgfutil@tempdima=-0.84pt%
  \advance\pgfutil@tempdima by-1.3\pgflinewidth%
  \pgfutil@tempdimb=-1.7pt%
  \advance\pgfutil@tempdimb by.625\pgflinewidth%
  \pgfarrowsleftextend{+\pgfutil@tempdima}
  \pgfarrowsrightextend{+\pgfutil@tempdimb}
}
{
  \pgfmathparse{\pgfgetarrowoptions{thickarrow}}%
  \pgfsetlinewidth{1.25 pt}
  \pgfutil@tempdima=0.28pt%
  \advance\pgfutil@tempdima by.3\pgflinewidth%
  \pgfsetlinewidth{0.8\pgflinewidth}
  \pgfsetdash{}{+0pt}
  \pgfsetroundcap
  \pgfsetroundjoin
  \pgfpathmoveto{\pgfqpoint{-3\pgfutil@tempdima}{4\pgfutil@tempdima}}
  \pgfpathcurveto
  {\pgfqpoint{-2.75\pgfutil@tempdima}{2.5\pgfutil@tempdima}}
  {\pgfqpoint{0pt}{0.25\pgfutil@tempdima}}
  {\pgfqpoint{0.75\pgfutil@tempdima}{0pt}}
  \pgfpathcurveto
  {\pgfqpoint{0pt}{-0.25\pgfutil@tempdima}}
  {\pgfqpoint{-2.75\pgfutil@tempdima}{-2.5\pgfutil@tempdima}}
  {\pgfqpoint{-3\pgfutil@tempdima}{-4\pgfutil@tempdima}}
  \pgfusepathqstroke
}
\pgfarrowsdeclare{reversethickarrow}{reversethickarrow}
{
  \pgfutil@tempdima=-0.84pt%
  \advance\pgfutil@tempdima by-1.3\pgflinewidth%
  \pgfutil@tempdimb=0.2pt%
  \advance\pgfutil@tempdimb by.625\pgflinewidth%
  \pgfarrowsleftextend{+\pgfutil@tempdima}
  \pgfarrowsrightextend{+\pgfutil@tempdimb}
}
{
  \pgftransformxscale{-1}
  \pgfmathparse{\pgfgetarrowoptions{thickarrow}}%
  \ifpgfmathunitsdeclared%
    \pgfmathparse{\pgfmathresult pt}%
  \else%
    \pgfmathparse{\pgfmathresult*\pgflinewidth}%
  \fi%
  \let\thickness=\pgfmathresult
  \pgfsetlinewidth{1.25 pt}
  \pgfutil@tempdima=0.28pt%
  \advance\pgfutil@tempdima by.3\pgflinewidth%
  \pgfsetlinewidth{0.8\pgflinewidth}
  \pgfsetdash{}{+0pt}
  \pgfsetroundcap
  \pgfsetroundjoin
  \pgfpathmoveto{\pgfqpoint{-3\pgfutil@tempdima}{4\pgfutil@tempdima}}
  \pgfpathcurveto
  {\pgfqpoint{-2.75\pgfutil@tempdima}{2.5\pgfutil@tempdima}}
  {\pgfqpoint{0pt}{0.25\pgfutil@tempdima}}
  {\pgfqpoint{0.75\pgfutil@tempdima}{0pt}}
  \pgfpathcurveto
  {\pgfqpoint{0pt}{-0.25\pgfutil@tempdima}}
  {\pgfqpoint{-2.75\pgfutil@tempdima}{-2.5\pgfutil@tempdima}}
  {\pgfqpoint{-3\pgfutil@tempdima}{-4\pgfutil@tempdima}}
  \pgfusepathqstroke
}
\makeatother

% Shapes
\makeatletter
\pgfdeclareshape{ground}
{
    \savedanchor\centerpoint
    {
        \pgf@x=0pt
        \pgf@y=0pt
    }
    \anchor{center}{\centerpoint}
    \anchorborder{\centerpoint}
    % \saveddimen\myscale
    % {
    %   \pgfkeysgetvalue{/pgf/scale}{\minwidth}
    %   \pgf@x=\minwidth
    % }
    \anchor{north}
    {
        \pgf@x=0pt
        \pgf@y=0.5*0.33*\stateheight
    }
    \anchor{south}
    {
        \pgf@x=0pt
        \pgf@y=0pt
    }
    \saveddimen\overallwidth
    {
        \pgfkeysgetvalue{/pgf/minimum width}{\minwidth}
        \pgf@x=\minimumstatewidth
        \ifdim\pgf@x<\minwidth
            \pgf@x=\minwidth
        \fi
    }
    \backgroundpath
    {
        \begin{pgfonlayer}{foreground}
        \pgfsetstrokecolor{black}
        \pgfsetlinewidth{1.25pt}
        \ifhflip
            \pgftransformyscale{-1}
        \fi
        \pgftransformscale{0.5}
        \pgfpathmoveto{\pgfpoint{-0.5*\overallwidth}{0}}
        \pgfpathlineto{\pgfpoint{0.5*\overallwidth}{0}}
        \pgfpathmoveto{\pgfpoint{-0.33*\overallwidth}{0.33*\stateheight}}
        \pgfpathlineto{\pgfpoint{0.33*\overallwidth}{0.33*\stateheight}}
        \pgfpathmoveto{\pgfpoint{-0.16*\overallwidth}{0.66*\stateheight}}
        \pgfpathlineto{\pgfpoint{0.16*\overallwidth}{0.66*\stateheight}}
        \pgfpathmoveto{\pgfpoint{-0.02*\overallwidth}{\stateheight}}
        \pgfpathlineto{\pgfpoint{0.02*\overallwidth}{\stateheight}}
        \pgfusepath{stroke}
        \end{pgfonlayer}
    }
}
\tikzset{forward arrow style/.style={every to/.style, decoration={
    markings,
    mark=at position 0.5 with \arrow{thickarrow}},
    postaction=decorate}}
\tikzset{reverse arrow style/.style={every to/.style, decoration={
    markings,
    mark=at position 0.5 with \arrow{reversethickarrow}},
    postaction=decorate}}
\pgfdeclareshape{morphism}
{
    \savedanchor\centerpoint
    {
        \pgf@x=0pt
        \pgf@y=0pt
    }
    \anchor{center}{\centerpoint}
    \anchorborder{\centerpoint}
    \saveddimen\savedlengthnw
    {
        \pgfkeysgetvalue{/tikz/keylengthnw}{\len}
        \pgf@x=\len
    }
    \saveddimen\savedlengthn
    {
        \pgfkeysgetvalue{/tikz/keylengthn}{\len}
        \pgf@x=\len
    }
    \saveddimen\savedlengthne
    {
        \pgfkeysgetvalue{/tikz/keylengthne}{\len}
        \pgf@x=\len
    }
    \saveddimen\savedlengthsw
    {
        \pgfkeysgetvalue{/tikz/keylengthsw}{\len}
        \pgf@x=\len
    }
    \saveddimen\savedlengths
    {
        \pgfkeysgetvalue{/tikz/keylengths}{\len}
        \pgf@x=\len
    }
    \saveddimen\savedlengthse
    {
        \pgfkeysgetvalue{/tikz/keylengthse}{\len}
        \pgf@x=\len
    }
    \saveddimen\overallwidth
    {
        \pgfkeysgetvalue{/tikz/width}{\minwidth}
        \pgf@x=\wd\pgfnodeparttextbox
        \ifdim\pgf@x<\minwidth
            \pgf@x=\minwidth
        \fi
    }
    \savedanchor{\upperrightcorner}
    {
        \pgf@y=.5\ht\pgfnodeparttextbox
        \advance\pgf@y by -.5\dp\pgfnodeparttextbox
        \pgf@x=.5\wd\pgfnodeparttextbox
    }
    \anchor{north}
    {
        \pgf@x=0pt
        \pgf@y=0.5\morphismheight
    }
    \anchor{north east}
    {
        \pgf@x=\overallwidth
        \multiply \pgf@x by 2
        \divide \pgf@x by 5
        \pgf@y=0.5\morphismheight
    }
    \anchor{east}
    {
        \pgf@x=\overallwidth
        \divide \pgf@x by 2
        \advance \pgf@x by 5pt
        \pgf@y=0pt
    }
    \anchor{west}
    {
        \pgf@x=-\overallwidth
        \divide \pgf@x by 2
        \advance \pgf@x by -5pt
        \pgf@y=0pt
    }
    \anchor{north west}
    {
        \pgf@x=-\overallwidth
        \multiply \pgf@x by 2
        \divide \pgf@x by 5
        \pgf@y=0.5\morphismheight
    }
    \anchor{connect nw}
    {
        \pgf@x=-\overallwidth
        \multiply \pgf@x by 2
        \divide \pgf@x by 5
        \pgf@y=0.5\morphismheight
        \advance\pgf@y by \savedlengthnw
    }
    \anchor{connect ne}
    {
        \pgf@x=\overallwidth
        \multiply \pgf@x by 2
        \divide \pgf@x by 5
        \pgf@y=0.5\morphismheight
        \advance\pgf@y by \savedlengthne
    }
    \anchor{connect sw}
    {
        \pgf@x=-\overallwidth
        \multiply \pgf@x by 2
        \divide \pgf@x by 5
        \pgf@y=-0.5\morphismheight
        \advance\pgf@y by -\savedlengthsw
    }
    \anchor{connect se}
    {
        \pgf@x=\overallwidth
        \multiply \pgf@x by 2
        \divide \pgf@x by 5
        \pgf@y=-0.5\morphismheight
        \advance\pgf@y by -\savedlengthse
    }
    \anchor{connect n}
    {
        \pgf@x=0pt
        \pgf@y=0.5\morphismheight
        \advance\pgf@y by \savedlengthn
    }
    \anchor{connect s}
    {
        \pgf@x=0pt
        \pgf@y=-0.5\morphismheight
        \advance\pgf@y by -\savedlengths
    }
    \anchor{south east}
    {
        \pgf@x=\overallwidth
        \multiply \pgf@x by 2
        \divide \pgf@x by 5
        \pgf@y=-0.5\morphismheight
    }
    \anchor{south west}
    {
        \pgf@x=-\overallwidth
        \multiply \pgf@x by 2
        \divide \pgf@x by 5
        \pgf@y=-0.5\morphismheight
    }
    \anchor{south}
    {
        \pgf@x=0pt
        \pgf@y=-0.5\morphismheight
    }
    \anchor{text}
    {
        \upperrightcorner
        \pgf@x=-\pgf@x
        \pgf@y=-\pgf@y
    }
    \backgroundpath
    {
        \pgfsetstrokecolor{black}
        \pgfsetlinewidth{\thickness}
        \begin{scope}
                \ifhflip
                    \pgftransformyscale{-1}
                \fi
                \ifvflip
                    \pgftransformxscale{-1}
                \fi
                \ifhvflip
                    \pgftransformxscale{-1}
                    \pgftransformyscale{-1}
                \fi
                \pgfpathmoveto{\pgfpoint
                    {-0.5*\overallwidth-5pt}
                    {0.5*\morphismheight}}
                \pgfpathlineto{\pgfpoint
                    {0.5*\overallwidth+5pt}
                    {0.5*\morphismheight}}
                \ifwedge
                    \pgfpathlineto{\pgfpoint
                        {0.5*\overallwidth + 15pt}
                        {-0.5*\morphismheight}}
                \else
                    \pgfpathlineto{\pgfpoint
                        {0.5*\overallwidth + 5pt}
                        {-0.5*\morphismheight}}
                \fi
                \pgfpathlineto{\pgfpoint
                    {-0.5*\overallwidth-5pt}
                    {-0.5*\morphismheight}}
                \pgfpathclose
                \pgfusepath{stroke}
        \end{scope}
        \ifconnectnw
            \pgfpathmoveto{\pgfpoint
                {-0.4*\overallwidth}
                {0.5*\morphismheight}}
            \pgfpathlineto{\pgfpoint
                {-0.4*\overallwidth}
                {0.5*\morphismheight+\savedlengthnw}}
            \pgfusepath{stroke}
        \fi
        \ifconnectne
            \pgfpathmoveto{\pgfpoint
                {0.4*\overallwidth}
                {0.5*\morphismheight}}
            \pgfpathlineto{\pgfpoint
                {0.4*\overallwidth}
                {0.5*\morphismheight+\savedlengthne}}
            \pgfusepath{stroke}
        \fi
        \ifconnectsw
            \pgfpathmoveto{\pgfpoint
                {-0.4*\overallwidth}
                {-0.5*\morphismheight}}
            \pgfpathlineto{\pgfpoint
                {-0.4*\overallwidth}
                {-0.5*\morphismheight-\savedlengthsw}}
            \pgfusepath{stroke}
        \fi
        \ifconnectse
            \pgfpathmoveto{\pgfpoint
                {0.4*\overallwidth}
                {-0.5*\morphismheight}}
            \pgfpathlineto{\pgfpoint
                {0.4*\overallwidth}
                {-0.5*\morphismheight-\savedlengthse}}
            \pgfusepath{stroke}
        \fi
        \ifconnectn
            \pgfpathmoveto{\pgfpoint
                {0pt}
                {0.5*\morphismheight}}
            \pgfpathlineto{\pgfpoint
                {0pt}
                {0.5*\morphismheight+\savedlengthn}}
            \pgfusepath{stroke}
        \fi
        \ifconnects
            \pgfpathmoveto{\pgfpoint
                {0pt}
                {-0.5*\morphismheight}}
            \pgfpathlineto{\pgfpoint
                {0pt}
                {-0.5*\morphismheight-\savedlengths}}
            \pgfusepath{stroke}
        \fi
        \ifconnectnwf
            \draw [forward arrow style] (-0.4*\overallwidth,0.5*\morphismheight)
                to (-0.4*\overallwidth,0.5*\morphismheight+\savedlengthnw);
        \fi
        \ifconnectnef
            \draw [forward arrow style] (0.4*\overallwidth,0.5*\morphismheight)
                to (0.4*\overallwidth,0.5*\morphismheight+\savedlengthne);
        \fi
        \ifconnectswf
            \draw [forward arrow style] (-0.4*\overallwidth,-0.5*\morphismheight-\savedlengthsw)
                to (-0.4*\overallwidth,-0.5*\morphismheight);
        \fi
        \ifconnectsef
            \draw [forward arrow style] (0.4*\overallwidth,-0.5*\morphismheight-\savedlengthse)
                to (0.4*\overallwidth,-0.5*\morphismheight);
        \fi
        \ifconnectnf
            \draw [forward arrow style] (0,0.5*\morphismheight)
                to (0,0.5*\morphismheight+\savedlengthn);
        \fi
        \ifconnectsf
            \draw [forward arrow style] (0,-0.5*\morphismheight-\savedlengths)
                to (0,-0.5*\morphismheight);
        \fi
        \ifconnectnwr
            \draw [reverse arrow style] (-0.4*\overallwidth,0.5*\morphismheight)
                to (-0.4*\overallwidth,0.5*\morphismheight+\savedlengthnw);
        \fi
        \ifconnectner
            \draw [reverse arrow style] (0.4*\overallwidth,0.5*\morphismheight)
                to (0.4*\overallwidth,0.5*\morphismheight+\savedlengthne);
        \fi
        \ifconnectswr
            \draw [reverse arrow style] (-0.4*\overallwidth,-0.5*\morphismheight-\savedlengthsw)
                to (-0.4*\overallwidth,-0.5*\morphismheight);
        \fi
        \ifconnectser
            \draw [reverse arrow style] (0.4*\overallwidth,-0.5*\morphismheight-\savedlengthse)
                to (0.4*\overallwidth,-0.5*\morphismheight);
        \fi
        \ifconnectnr
            \draw [reverse arrow style] (0,0.5*\morphismheight)
                to (0,0.5*\morphismheight+\savedlengthn);
        \fi
        \ifconnectsr
            \draw [reverse arrow style] (0,-0.5*\morphismheight-\savedlengths)
                to (0,-0.5*\morphismheight);
        \fi
    }
}
\tikzset{trapnode/.style={draw, trapezium, trapezium stretches=true, minimum width=1cm, minimum height=0.6cm}}
\tikzset{brnode/.style={trapnode, trapezium left angle=90, trapezium right angle=70}}
\tikzset{trnode/.style={trapnode, trapezium left angle=90, trapezium right angle=110}}
\tikzset{blnode/.style={trapnode, trapezium left angle=70, trapezium right angle=90}}
\tikzset{tlnode/.style={trapnode, trapezium left angle=110, trapezium right angle=90}}
\pgfdeclareshape{swish right}
{
    \savedanchor\centerpoint
    {
        \pgf@x=0pt
        \pgf@y=0pt
    }
    \anchor{center}{\centerpoint}
    \anchorborder{\centerpoint}
    \anchor{north}
    {
        \pgf@x=\minimummorphismwidth
        \divide\pgf@x by 5
        \pgf@y=\morphismheight
        \divide\pgf@y by 2
        \advance\pgf@y by \connectheight
    }
    \anchor{south}
    {
        \pgf@x=-\minimummorphismwidth
        \divide\pgf@x by 5
        \pgf@y=-\morphismheight
        \divide\pgf@y by 2
        \advance\pgf@y by -\connectheight
    }
    \backgroundpath
    {
        \pgfsetstrokecolor{black}
        \pgfsetlinewidth{\thickness}
        \pgfpathmoveto{\pgfpoint
            {-0.2*\minimummorphismwidth}
            {-0.5*\morphismheight-\connectheight}}
        \pgfpathcurveto
            {\pgfpoint{-0.2*\minimummorphismwidth}{0pt}}
            {\pgfpoint{0.2*\minimummorphismwidth}{0pt}}
            {\pgfpoint
                {0.2*\minimummorphismwidth}
                {0.5*\morphismheight+\connectheight}}
        \pgfusepath{stroke}
    }
}
\pgfdeclareshape{swish left}
{
    \savedanchor\centerpoint
    {
        \pgf@x=0pt
        \pgf@y=0pt
    }
    \anchor{center}{\centerpoint}
    \anchorborder{\centerpoint}
    \anchor{north}
    {
        \pgf@x=-\minimummorphismwidth
        \divide\pgf@x by 5
        \pgf@y=\morphismheight
        \divide\pgf@y by 2
        \advance\pgf@y by \connectheight
    }
    \anchor{south}
    {
        \pgf@x=\minimummorphismwidth
        \divide\pgf@x by 5
        \pgf@y=-\morphismheight
        \divide\pgf@y by 2
        \advance\pgf@y by -\connectheight
    }
    \backgroundpath
    {
        \pgfsetstrokecolor{black}
        \pgfsetlinewidth{\thickness}
        \pgfpathmoveto{\pgfpoint
            {0.2*\minimummorphismwidth}
            {-0.5*\morphismheight-\connectheight}}
        \pgfpathcurveto
            {\pgfpoint{0.2*\minimummorphismwidth}{0pt}}
            {\pgfpoint{-0.2*\minimummorphismwidth}{0pt}}
            {\pgfpoint
                {-0.2*\minimummorphismwidth}
                {0.5*\morphismheight+\connectheight}}
        \pgfusepath{stroke}
    }
}
\pgfdeclareshape{state}
{
    \savedanchor\centerpoint
    {
        \pgf@x=0pt
        \pgf@y=0pt
    }
    \anchor{center}{\centerpoint}
    \anchorborder{\centerpoint}
    \saveddimen\overallwidth
    {
        \pgf@x=3\wd\pgfnodeparttextbox
        \ifdim\pgf@x<\minimumstatewidth
            \pgf@x=\minimumstatewidth
        \fi
    }
    \savedanchor{\upperrightcorner}
    {
        \pgf@x=.5\wd\pgfnodeparttextbox
        \pgf@y=.5\ht\pgfnodeparttextbox
        \advance\pgf@y by -.5\dp\pgfnodeparttextbox
    }
    \anchor{A}
    {
        \pgf@x=-\overallwidth
        \divide\pgf@x by 4
        \pgf@y=0pt
    }
    \anchor{B}
    {
        \pgf@x=\overallwidth
        \divide\pgf@x by 4
        \pgf@y=0pt
    }
    \anchor{text}
    {
        \upperrightcorner
        \pgf@x=-\pgf@x
        \ifhflip
            \pgf@y=-\pgf@y
            \advance\pgf@y by 0.4\stateheight
        \else
            \pgf@y=-\pgf@y
            \advance\pgf@y by -0.4\stateheight
        \fi
    }
    \backgroundpath
    {
%        \begin{pgfonlayer}{foreground}
        \pgfsetstrokecolor{black}
        \pgfsetlinewidth{\thickness}
        \pgfpathmoveto{\pgfpoint{-0.5*\overallwidth}{0}}
        \pgfpathlineto{\pgfpoint{0.5*\overallwidth}{0}}
        \ifhflip
            \pgfpathlineto{\pgfpoint{0}{\stateheight}}
        \else
            \pgfpathlineto{\pgfpoint{0}{-\stateheight}}
        \fi
        \pgfpathclose
        \ifblack
            \pgfsetfillcolor{black!50}
            \pgfusepath{fill,stroke}
        \else
            \pgfusepath{stroke}
        \fi
%        \end{pgfonlayer}
    }
}

\makeatother

% Little pictures
\newcommand{\tinycomult}[1][dot]{
\smash{\raisebox{-1.8pt}{\hspace{-5pt}\ensuremath{\begin{pic}[scale=0.34,string]
    \node (0) at (0,0) {};
    \node[#1, inner sep=1.5pt] (1) at (0,0.55) {};
    \node (2) at (-0.5,1) {};
    \node (3) at (0.5,1) {};
    \draw (0.center) to (1.center);
    \draw (1.center) to [out=left, in=down, out looseness=1.5] (2.center);
    \draw (1.center) to [out=right, in=down, out looseness=1.5] (3.center);
\end{pic}
}\hspace{-3pt}}}}
\newcommand{\tinycounit}[1][dot]{
\smash{\raisebox{-1.8pt}{\ensuremath{\hspace{-3pt}\begin{pic}[scale=0.34,string]
        \node (0) at (0,0) {};
        \node (1) at (0,1) {};
        \node[#1, inner sep=1.5pt] (d) at (0,0.55) {};
        \draw (0.center) to (d.center);
    \end{pic}
    \hspace{-1pt}}}}}
\newcommand{\tinymult}[1][dot]{
\smash{{\hspace{-5pt}\ensuremath{\begin{aligned}\begin{tikzpicture}[scale=0.36,thick,yscale=-1]
    \node (0) at (0,0) {};
    \node (2) at (-0.5,1) {};
    \node (3) at (0.5,1) {};
    \draw (0.center) to (0,0.55);
    \draw (0,0.55) to [out=left, in=down, out looseness=1.5] (2.center);
    \draw (0,0.55) to [out=right, in=down, out looseness=1.5] (3.center);
    \node[#1, inner sep=1.5pt] (1) at (0,0.55) {};
\end{tikzpicture}\end{aligned}
}\hspace{-3pt}}}}
\newcommand{\tinyunit}[1][dot]{
\smash{{\ensuremath{\hspace{-3pt}\begin{aligned}\begin{tikzpicture}[scale=0.4,thick,yscale=-1]
        \node (0) at (0,0) {};
        \node (1) at (0,1) {};
        \draw (0.center) to (0,0.55);
        \node[#1, inner sep=1.5pt] (d) at (0,0.55) {};
    \end{tikzpicture}\end{aligned}
    \hspace{-1pt}}}}}
\newcommand{\tinyid}{\raisebox{-2pt}{\ensuremath{\hspace{-4pt}\begin{pic}[scale=0.4,string]
        \node (0) at (0,0) {};
        \node (1) at (0,1) {};
        \draw[string] (0.center) to (1.center);
     \end{pic}
     \hspace{-2pt}}}}
\newcommand{\tinyhandle}[1][dot]{\raisebox{-2pt}{\ensuremath{\hspace{-3pt}\begin{pic}[scale=0.4,string]
        \node (0) at (0,0) {};
        \node[dot, inner sep=1.0pt] (1) at (0,0.3) {};
        \node[dot, inner sep=1.0pt] (2) at (0,0.7) {};
        \node (3) at (0,1) {};
        \draw (0.center) to (1.center);
        \draw (2.center) to (3.center);
        \draw[in=180, out=180, looseness=2] (1.center) to (2.center);
        \draw[in=0, out=0, looseness=2] (1.center) to (2.center);
\end{pic}\hspace{-1pt}}}}
\newcommand\whitemonoid[1]{\ensuremath{(#1,\tinymult[whitedot],\tinyunit[whitedot])}}
\newcommand\blackmonoid[1]{\ensuremath{(#1,\tinymult[blackdot],\tinyunit[blackdot])}}
\newcommand\graymonoid[1]{\ensuremath{(#1,\tinymult[graydot],\tinyunit[graydot])}}
\newcommand\whitecomonoid[1]{\ensuremath{(#1,\tinycomult[whitedot],\tinycounit[whitedot])}}
\newcommand\blackcomonoid[1]{\ensuremath{(#1,\tinycomult[blackdot],\tinycounit[blackdot])}}
\newcommand\graycomonoid[1]{\ensuremath{(#1,\tinycomult[graydot],\tinycounit[graydot])}}

% Shortcuts
% From Will
\newcommand{\ketbra}[2]{\left|#1\right\rangle\!\left\langle #2\right|}
\newcommand{\braketb}[2]{\left\langle #1 | #2 \right\rangle}
\newcommand\cat[1]{\ensuremath{\mathbf{#1}}}
\newcommand{\objs}[1]{\mbox{Ob}(\cat{#1})}
\newcommand{\arrs}[1]{\mbox{Arr}(\cat{#1})}
\newcommand{\idm}[1]{\mbox{id}_{#1}}
\newcommand{\Hsp}{\mathcal{H}}
\newcommand{\dsmc}{$\dagger$-SMC~}
\newcommand{\dcc}{$\dagger$-compact category~}
\newcommand{\isom}{\cong} % Isomorphism
\newcommand{\iso}{\cong} % Another definition for isomorphism
\newcommand{\LtwoSym}{\operatorname{L}^2} % Symbol for L2 spaces
\newcommand{\Ltwo}[1]{\LtwoSym[#1]} % L2 space over space #1
\newcommand{\tikzeq}[1]{
  \begin{equation}\input{#1}\end{equation}
}
\tikzstyle{copoint}=[regular polygon,regular polygon sides=3,draw=black,scale=0.75,inner sep=-0.5pt,minimum width=7mm,fill=white]
\tikzstyle{point}=[regular polygon,regular polygon sides=3,draw=black,scale=0.75,inner sep=-0.5pt,minimum width=7mm,fill=white,regular polygon rotate=180]
\tikzstyle{gray point}=[point,fill=gray!40!white]
\tikzstyle{gray copoint}=[copoint,fill=gray!40!white]
\tikzstyle{antipode}=[whitedot,inner sep=0.3mm,font=\footnotesize]

%% Sets, groups and algebra
        \newcommand{\Powerset}[1]{\mathcal{P}(#1)} % Powerset of #1
        \newcommand{\naturals}{\mathbb{N}} % Set of natural numbers
        \newcommand{\integers}{\mathbb{Z}} % Set of interer numbers
        \newcommand{\rationals}{\mathbb{Q}} % Set of rational numbers
        \newcommand{\reals}{\mathbb{R}} % Set of real numbers
        \newcommand{\complexs}{\mathbb{C}} % Set of complex numbers
        \newcommand{\integersMod}[1]{\mathbb{Z}_{#1}} % Set/group/ring of integers mod #1
        \newcommand{\modclass}[2]{#1 \; (\text{mod } #2)} % Equivalence class of integers mod #2 corresponding to representative #1
        \newcommand{\restrict}[2]{\left. #1 \right\vert_{#2}} % Restriction of function/morphism #1 to subset/subobject #2
        \newcommand{\domain}[1]{\operatorname{dom}#1} % Domain of function/morphism #1
        \newcommand{\inject}{\hookrightarrow} % Set-theoretic injection
        \newcommand{\Irreps}[1]{\operatorname{Irr}[#1]} % Set of irreps for group #1

% From Bob/Aleks 
\newcommand{\name}[1]{\ensuremath{\ulcorner #1 \urcorner}}
\newcommand{\coname}[1]{\ensuremath{\llcorner #1 \lrcorner}}
\newcommand{\op}{\ensuremath{\mathrm{op}}}
\newcommand{\tuple}[2]{\ensuremath{\left(\begin{smallmatrix} #1\\#2 \end{smallmatrix}\right)}}%{\ensuremath{\protect\langle #1,#2 \protect\rangle}}
\DeclareMathOperator{\Tr}{Tr}
\DeclareMathOperator{\dom}{dom}
\DeclareMathOperator{\cod}{cod}
\renewcommand\dag[1]{\ensuremath{#1^\dagger}}
\newcommand\lecture[2]{\section{#1}}
\newcommand\id[1][]{\ensuremath{\mathrm{id}_{#1}}}
\def\totimes{{\textstyle \bigotimes}}
\newcommand\sxleftarrow[1]{\xleftarrow{\smash{#1}}}
\newcommand\sxrightarrow[1]{\xrightarrow{\smash{#1}}}
\newcommand\xto[1]{\xrightarrow{#1}}
\newcommand\sxto[1]{\sxrightarrow{#1}}
\renewcommand{\-}[0]{\nobreakdash-\hspace{0pt}}
\DeclareMathOperator{\Ob}{Ob}
\newcommand\ket[1]{\ensuremath{| #1 \rangle}}
\newcommand\bra[1]{\ensuremath{\langle #1 |}}
\newcommand{\braket}[2]{\langle #1 \vert #2 \rangle} % Inner product of bra %labelled #1 with ket labelled #2
\newcommand\C{\ensuremath{\mathbb{C}}}
\newcommand{\inprod}[2]{\ensuremath{\langle #1\hspace{0.5pt}|\hspace{0.5pt}#2 \rangle}}
\newcommand{\defined}{\ensuremath{\mathop{\downarrow}}}
\newcommand\Hom{\ensuremath{\textrm{Hom}}}
\newcommand\pdag{{\phantom{\dagger}}}
\DeclareMathOperator{\CP}{CPM}
\DeclareMathOperator{\CPstar}{CP^*}
\newcommand{\proj}{\ensuremath{p}}
\newcommand{\inj}{\ensuremath{i}}
\DeclareMathOperator{\Prob}{Prob}
\newcommand{\ie}{\textit{i.e.}\xspace}
\newcommand{\eg}{\textit{e.g.}\xspace}
\newcommand\ud{\ensuremath{\mathrm{d}}}
\newcommand{\todo}[1]{{\color{red}#1}}

%NEW
\tikzstyle{dotpic}=[scale=0.5]
\newcommand{\dotonly}[1]{%
\,\begin{tikzpicture}[dotpic,yshift=-0.3mm]
\node [#1] (a) at (0,0) {};
\end{tikzpicture}\,}
\newcommand{\whitedot}{\dotonly{whitedot}}
\tikzstyle{braceedge}=[decorate,decoration={brace,amplitude=2mm,raise=-1mm}]
\tikzstyle{small braceedge}=[decorate,decoration={brace,amplitude=1mm,raise=-1mm}]
\tikzstyle{wire label}=[font=\tiny, auto]
\newcommand{\lildot}[1]{%
\,\begin{tikzpicture}[dotpic,yshift=-0.15mm]
\node [#1,inner sep=0.4mm,minimum width=0pt,minimum height=0pt] (a) at (0,0) {};
\end{tikzpicture}\,}
\tikzstyle{diredge}=[->]
\tikzstyle{wide point}=[fill=white,draw=black,shape=isosceles triangle,shape border rotate=-90,isosceles triangle stretches=true,inner sep=1pt,minimum width=1.5cm,minimum height=4mm]
\tikzstyle{square box}=[rectangle,fill=white,draw=black,minimum height=5mm,minimum width=5mm,font=\small]
\newcommand{\RelCategory}{\cat{Rel}}
\newcommand{\fRelCategory}{\cat{FRel}}
\newcommand{\timemultSym}{\tinymult[whitedot]}
\newcommand{\timeunitSym}{\tinyunit[whitedot]}
\newcommand{\timecomultSym}{\tinycomult[whitedot]}
\newcommand{\timecounitSym}{\tinycounit[whitedot]}
\newcommand{\antipodeSym}{\hbox{\begin{tikzpicture} [scale=1,transform shape] %% DO NOT CHANGE

\def\deltax{0.1} %% CAN BE CHANGED
\def\deltay{0.4} %% DO NOT CHANGE

\path[use as bounding box] (-\deltax,-0.1) rectangle (\deltax,\deltay);

\node [draw, diamond, scale=0.5] (mult) at (0,0) {};
\node (mult_label_in) at (0,-\deltay) {};
\node (mult_label_out) at (0,+\deltay) {};
\draw[-] (mult_label_in) to (mult);
\draw[-] (mult) to (mult_label_out);

%\draw (current bounding box.south west) rectangle (current bounding box.north %east);
\end{tikzpicture}
}\!} % Antipode (group inverse)
\newcommand{\timediagSym}{\tinycomult[blackdot]}
\newcommand{\trivialcharSym}{\tinycounit[blackdot]}
\newcommand{\scpair}{(\dotonly{whitedot},\dotonly{blackdot})}
\newcommand{\XdotSym}{\dotonly{blackdot}}
\newcommand{\ZdotSym}{\dotonly{whitedot}}
\newcommand{\fdHilbCategory}{\cat{FHilb}}
\newcommand{\XmultSym}{\tinymult[blackdot]}
\newcommand{\XunitSym}{\tinyunit[blackdot]}
\newcommand{\XcomultSym}{\tinycomult[blackdot]}
\newcommand{\XcounitSym}{\tinycounit[blackdot]}
\newcommand{\ZmultSym}{\tinymult[whitedot]}
\newcommand{\ZunitSym}{\tinyunit[whitedot]}
\newcommand{\ZcomultSym}{\tinycomult[whitedot]}
\newcommand{\ZcounitSym}{\tinycounit[whitedot]}
\newcommand{\tempSymLabel}{}
\newcommand{\effectSym}[1]{
  \renewcommand{\tempSymLabel}{#1}
  \hbox{\begin{tikzpicture} [scale=1.2,transform shape] %% DO NOT CHANGE

\def\deltax{0.4} %% CAN BE CHANGED
\def\deltay{0.7} %% DO NOT CHANGE

\path[use as bounding box] (-\deltax,-0.2) rectangle (\deltax,\deltay);

\node [state, wedge, hflip, scale=0.8] (mult) at (0,0) {$x$};
\node (mult_label_in) at (0,-1*\deltay) {};
\draw[-] (mult_label_in) to (mult.south);

%\draw (current bounding box.south west) rectangle (current bounding box.north east);
\end{tikzpicture}
}
}
\newcommand{\groupStateSym}[1]{
  \renewcommand{\tempSymLabel}{#1}
  \hbox{\begin{tikzpicture} [scale=1.2,transform shape] %% DO NOT CHANGE

\def\deltax{0.3} %% CAN BE CHANGED
\def\deltay{0.7} %% DO NOT CHANGE

\path[use as bounding box] (-\deltax,-\deltay) rectangle (\deltax,\deltay);

\node [point, fill=\classicalStructColour] (mult) at (0,0) {\tempSymLabel};
\node (mult_label_out) at (0,+\deltay) {};
\draw[-] (mult) to (mult_label_out);

%\draw (current bounding box.south west) rectangle (current bounding box.north east);
\end{tikzpicture}
}
}
\newcommand{\multCharacterEffectSym}[1]{
  \renewcommand{\tempSymLabel}{#1}
  \hbox{\begin{pic} [scale=1.2,transform shape] %% DO NOT CHANGE

\def\deltax{0.1} %% CAN BE CHANGED
\def\deltay{0.5} %% DO NOT CHANGE

\path[use as bounding box] (-\deltax,-0.8) rectangle (\deltax,\deltay);

\node [morphism, wedge, fill=black, scale=0.4] (mult) at (0,0) {\tempSymLabel};
\node (mult_label_out) at (0,-\deltay) {};
\draw[-] (mult.south) to (mult_label_out);

%\draw (current bounding box.south west) rectangle (current bounding box.north east);
\end{pic}
}
}   
\newcommand{\linearFunctionalSym}[1]{
  \renewcommand{\tempSymLabel}{#1}
  \hbox{\begin{tikzpicture} [scale=1.2,transform shape] %% DO NOT CHANGE

\def\deltax{0.3} %% CAN BE CHANGED
\def\deltay{0.7} %% DO NOT CHANGE

\path[use as bounding box] (-\deltax,-\deltay) rectangle (\deltax,\deltay);

\node [kpointdag] (mult) at (0,0) {\tempSymLabel};
\node (mult_label_in) at (0,-\deltay) {};
\draw[-] (mult_label_in) to (mult);

%\draw (current bounding box.south west) rectangle (current bounding box.north east);
\end{tikzpicture}
}
}
\newcommand{\characterSym}[1]{
  \renewcommand{\tempSymLabel}{#1}
  \hbox{\begin{tikzpicture} [scale=1,transform shape] %% DO NOT CHANGE

\def\deltax{0.3} %% CAN BE CHANGED
\def\deltay{0.7} %% DO NOT CHANGE

\path[use as bounding box] (-\deltax,-\deltay) rectangle (\deltax,\deltay);

\node [mapdag, fill=\classicalStructColour] (mult) at (0,0) {\tempSymLabel};
\node (mult_label_in) at (0,-1*\deltay) {};
\draw[-] (mult_label_in) to (mult);

%\draw (current bounding box.south west) rectangle (current bounding box.north east);
\end{tikzpicture}}
}
\newcommand{\tensor}{\otimes}
%% General purpose space names 
\newcommand{\SpaceH}{\mathcal{H}}
\newcommand{\SpaceA}{\mathcal{A}}
\newcommand{\SpaceB}{\mathcal{B}}
\newcommand{\SpaceG}{\mathcal{G}}
\newcommand{\suchthat}[2]{\left\{#1 \: \text{ s.t. } \: #2\right\}} % Set of elements #1 such that condition #2 holds 
% =======================
% = PARALLELAGRAM BOXES =
% =======================

\makeatletter
\newcommand{\boxshape}[3]{%
\pgfdeclareshape{#1}{
\inheritsavedanchors[from=rectangle] % this is nearly a rectangle
\inheritanchorborder[from=rectangle]
\inheritanchor[from=rectangle]{center}
\inheritanchor[from=rectangle]{north}
\inheritanchor[from=rectangle]{south}
\inheritanchor[from=rectangle]{west}
\inheritanchor[from=rectangle]{east}
% ... and possibly more
\backgroundpath{% this is new
% store lower right in xa/ya and upper right in xb/yb
\southwest \pgf@xa=\pgf@x \pgf@ya=\pgf@y
\northeast \pgf@xb=\pgf@x \pgf@yb=\pgf@y

\@tempdima=#2
\@tempdimb=#3

\pgfpathmoveto{\pgfpoint{\pgf@xa - 5pt + \@tempdima}{\pgf@ya}}
\pgfpathlineto{\pgfpoint{\pgf@xa - 5pt - \@tempdima}{\pgf@yb}}
\pgfpathlineto{\pgfpoint{\pgf@xb + 5pt + \@tempdimb}{\pgf@yb}}
\pgfpathlineto{\pgfpoint{\pgf@xb + 5pt - \@tempdimb}{\pgf@ya}}
\pgfpathlineto{\pgfpoint{\pgf@xa - 5pt + \@tempdima}{\pgf@ya}}
\pgfpathclose
}
}}

\boxshape{NEbox}{0pt}{5pt}
\boxshape{SEbox}{0pt}{-5pt}
\boxshape{NWbox}{5pt}{0pt}
\boxshape{SWbox}{-5pt}{0pt}
\boxshape{EBox}{-3pt}{3pt}
\boxshape{WBox}{3pt}{-3pt}
\makeatother

\tikzstyle{cloud}=[shape=cloud,draw,minimum width=1.5cm,minimum height=1.5cm]

\tikzstyle{map}=[draw,shape=NEbox,inner sep=2pt,minimum height=6mm,fill=white]
\tikzstyle{dashedmap}=[draw,dashed,shape=NEbox,inner sep=2pt,minimum height=6mm,fill=white]
\tikzstyle{mapdag}=[draw,shape=SEbox,inner sep=2pt,minimum height=6mm,fill=white]
\tikzstyle{mapadj}=[draw,shape=SEbox,inner sep=2pt,minimum height=6mm,fill=white]
\tikzstyle{maptrans}=[draw,shape=SWbox,inner sep=2pt,minimum height=6mm,fill=white]
\tikzstyle{mapconj}=[draw,shape=NWbox,inner sep=2pt,minimum height=6mm,fill=white]

\def\swangle{-145}
\def\seangle{-35}
\def\nwangle{145}
\def\neangle{35}

% Tikz Scaling
\def\tikzyscale{0.5}
\def\tikzxscale{0.5}

\ignore{
\usepackage{mathtools}
\tikzstyle{dot}=[inner sep=0.7mm,minimum width=0pt,minimum
height=0pt,fill=black,draw=black,shape=circle]
\tikzstyle{black dot}=[dot,fill=black]
\tikzstyle{white dot}=[dot,fill=white]
\tikzstyle{gray dot}=[dot,fill=gray!40!white]
}


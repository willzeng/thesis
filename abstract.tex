\begin{abstract}
Quantum information brings theories of physics and computer science together in a synthesis that challenges both of their basic intuitions. In this thesis, we show that adopting a unified and general language for process theories advances both foundations and practical applications of quantum information.

Our first set of results analyzes quantum algorithms with this abstract structure. To this end, we contribute new constructions of the Fourier transform and Pontryagin duality in dagger symmetric monoidal categories. We then use this setting to study generalized unitary oracles and give a new quantum blackbox algorithm for the identification of group homomorphisms, solving the GROUPHOMID problem. In the remaining section, we construct a novel model of quantum blackbox algorithms in non-deterministic classical computation.

Our second set of results concern quantum foundations. We complete work begun by Coecke et al.~\cite{coecke2012strong, coecke2011phase} by definitively connecting the Mermin non-locality of a process theory with a simple algebraic condition on that theory's phase groups. This result allows us to offer new experimental tests for Mermin non-locality and new protocols for quantum secret sharing.

In our final chapter, we exploit the shared process theoretic structure of quantum information and distributional compositional linguistics. We propose a quantum algorithm adapted from~\cite{wiebe2014quantum} to classify sentences by meaning.  The clarity of the process theoretic setting allows us to recover a speedup that is lost in the naive application of the algorithm.

The main mathematical tools used in this thesis are group theory (esp. Fourier theory on finite groups), monoidal category theory, and categorical algebra.
\end{abstract}
